%\textbf{Practice Exercises}
\textbf{Problem Set}

\vspce

\begin{enumerate}[label = \Alph*. ]
\item Name the point that has the coordinates. 

\begin{enumerate}[label = \arabic*. ]
\begin{multicols}{3}

%#1
\item \hspce  (7, -8)
\vspce
%#2
\item \hspce  (-7, 3)
\vspce
%#3
\item \hspce  (0, 8)
\vspce
%#4
\item \hspce  (-6, -9)
\vspce
%#5
\item \hspce  (2, 9)

\end{multicols}
\end{enumerate}

\item Write the coordinates of each point. 

\begin{enumerate}[label = \arabic*. ]
\begin{multicols}{3}

\item \hspce C
\item \hspce A
\item \hspce F
\item \hspce D
\item \hspce H
\end{multicols} 
\end{enumerate} 

\item Determine the  quadrant or axis where each point is located. 

\begin{enumerate}[label = \arabic*. ]
\begin{multicols}{3}
\item \hspce H
\item \hspce G
\item \hspce F
\item \hspce E
\item \hspce D
\end{multicols}
\end{enumerate} 

\end{enumerate} 

%\newpage 

%\begin{center}

\begin{tikzpicture}[%scale=3.5, 
scale=1.5, place/.style={circle, fill=black, inner sep=0pt, outer sep=0pt, minimum size=3pt}, 
point-label/.style={fill=white, circle,inner sep=0pt}
]

\begin{axis}[
    xmin=-11,xmax=11,
    ymin=-11,ymax=11,
    grid=both,
    axis lines=middle,
    minor tick num=4,
    enlargelimits={abs=0.5},
    axis line style={latex-latex},
    ticklabel style={font=\tiny,fill=white, inner sep=0pt},
    xlabel style={at={(ticklabel* cs:1)},anchor=north west},
    ylabel style={at={(ticklabel* cs:1)},anchor=south west}
]

\coordinate (O) at (0,0);
\node[fill=white,circle,inner sep=0pt] (O-label) at ($(O)+(135:10pt)$) {\normalsize $O$};

\node (A) at (2,9) [place] {}; 
\node[point-label] (A-label) at ($(A)+(115:7pt)$) {\normalsize $A$};

\node (B) at (-7,3) [place] {}; 
\node[point-label] (B-label) at ($(B)+(115:7pt)$) {\normalsize $B$};

\node (C) at (-6,-9) [place] {}; 
\node[point-label] (C-label) at ($(C)+(-135:7pt)$) {\normalsize $C$};

\node (D) at (9,0) [place] {}; 
\node[point-label] (D-label) at ($(D)+(115:7pt)$) {\normalsize $D$};

\node (E) at (7,-8) [place] {}; 
\node[point-label] (E-label) at ($(E)+(115:7pt)$) {\normalsize $E$};

\node (F) at (0,8) [place] {}; 
\node[point-label] (F-label) at ($(F)+(15:7pt)$) {\normalsize $F$};

\node (G) at (-8,0) [place] {}; 
\node[point-label] (G-label) at ($(G)+(115:7pt)$) {\normalsize $G$};

\node (H) at (-9,-7) [place] {}; 
\node[point-label] (H-label) at ($(H)+(115:7pt)$) {\normalsize $H$};

\end{axis}

\end{tikzpicture}
\end{center} 

 