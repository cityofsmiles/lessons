\begin{center}
\textbf{Properties of Parallel Lines Cut by a Transversal 
}
\end{center}

\vspce

\textbf{Definitions} \\
\textbf{Parallel Lines:} lines that do not intersect and are coplanar

\textbf{Transversal:} a line that intersects two or more lines in a plane at different points

\textbf{Alternate Exterior Angles:} two exterior angles found on different sides of a transversal

\textbf{Alternate Interior Angles:} two interior angles found on different sides of a transversal

\textbf{Corresponding Angles:} pair of interior and exterior angles found on the same side of a transversal

\vspce 

\textbf{Parallel Postulates}
\begin{enumerate}[label = \arabic*. ]
%1
\item If there is a line and a point not on the line, then there is exactly one line through the point that is
parallel to the given line.
%2
\item If two parallel lines are cut by a transversal, then the corresponding angles are equal. 
\end{enumerate}   

\vspce 

\textbf{Theorems}
\begin{enumerate}[label = \arabic*. ]
%\begin{multicols}{2}
%1
\item If two parallel lines are cut by a transversal, then alternate interior angles are equal.
%2
\item If two parallel lines are cut by a transversal, then alternate exterior angles are equal.
%3
\item If two parallel lines are cut by a transversal, then interior angles on the same side of the transversal are supplementary.
%4
\item If two parallel lines are cut by a transversal, then exterior angles on the same side of the transversal are supplementary.
%\end{multicols} 
\end{enumerate}  





%If two lines are parallel to the same line, then they are parallel to each other.

%\vspce 



%\vspce 

%If two lines are cut by a transversal and a pair of alternate interior angles is congruent, then the lines are parallel.

%\vspce 

%If two lines are cut by a transversal and a pair of corresponding angles is congruent, then the lines are parallel.




%\vspce

%In a plane, if a transversal is perpendicular to two lines, then they are parallel. 

%\vspce

%If two lines are cut by a transversal so that the interior angles on the same side of the transversal are supplementary, then the lines are parallel.
 

