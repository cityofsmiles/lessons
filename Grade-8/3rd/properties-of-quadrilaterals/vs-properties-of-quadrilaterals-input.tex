\begin{center}
\textbf{Properties of Quadrilaterals
}
\end{center}

\vspace*{1ex}

\begin{center}
%\vspace*{-2ex}
\scalebox{0.8}{
\noindent\begin{minipage}{\textwidth}
{

\textbf{Polygon:} a geometric figure formed by three or more coplanar segments called sides.  The sides that have
a common endpoint are non-collinear. Each side intersects exactly two other sides only at their endpoints
called vertices.

\textbf{Diagonal:} a segment that connects any two non-consecutive vertices of a polygon

\textbf{Quadrilateral:} a four-sided polygon

\textbf{Types of Quadrilaterals} 
\begin{enumerate}[label = \arabic*. ]
%1
\item Trapezium: a quadrilateral with no pair of parallel sides
%2
\item Trapezoid: a quadrilateral with exactly one pair of parallel sides. If the nonparallel sides are congruent, the trapezoid is isosceles.
%3
\item Parallelogram: a special quadrilateral with both pairs of opposite sides parallel

\begin{enumerate}[label = \alph*. ]
%1
\item Rectangle: has four right angles 
%2
\item Rhombus: has four congruent sides
%3
\item Square: has four 
congruent angles and four congruent sides
\end{enumerate}   
\end{enumerate}   

\textbf{Theorems on Parallelograms}
\begin{enumerate}[label = \arabic*. ]
%\begin{multicols}{2}
%1
\item Opposite sides of a parallelogram are congruent.
%2
\item Opposite angles of a parallelogram are congruent.
%3
\item Consecutive angles in a parallelogram are supplementary.
%4
\item The diagonals of a parallelogram bisect each other.
%\end{multicols} 
\end{enumerate}  
}
\end{minipage}}
\end{center}  

%\begin{center}

%\noindent\begin{minipage}{\textwidth}
%{\fontsize{20}{22}\fontfamily{pnc}\selectfont{
\begin{center}
%\vspace*{-2ex}
\scalebox{0.8}{
\noindent\begin{minipage}{\textwidth}
{
\begin{tabularx}{\textwidth}{|>{\hsize=1.8\hsize}Y|>{\hsize=0.9\hsize}Y|>{\hsize=0.8\hsize}Y|>{\hsize=0.7\hsize}Y|>{\hsize=0.8\hsize}Y|}
\hline
\textbf{Property} & \textbf{Rectangle} & \textbf{Rhombus} & \textbf{Square} & \textbf{All Parallelograms} \\
\hline
\end{tabularx} 

\par\vskip1pt

\begin{tabularx}{\textwidth}{|>{\hsize=1.8\hsize}X|>{\hsize=0.9\hsize}Y|>{\hsize=0.8\hsize}Y|>{\hsize=0.7\hsize}Y|>{\hsize=0.8\hsize}Y|}
\hline
\textbf{The Sides} & & & &\\
\hline
Opposite sides are parallel. & $\checked$ & $\checked$ & $\checked$ & $\checked$\\
\hline
Opposite sides are congruent. & $\checked$ & $\checked$ & $\checked$ & $\checked$\\
\hline
All sides are congruent. & & $\checked$ & $\checked$ &\\
\hline
\end{tabularx} 

\par\vskip1pt

\begin{tabularx}{\textwidth}{|>{\hsize=1.8\hsize}X|>{\hsize=0.9\hsize}Y|>{\hsize=0.8\hsize}Y|>{\hsize=0.7\hsize}Y|>{\hsize=0.8\hsize}Y|}
\hline
\textbf{The Angles } & & & &\\
\hline
Sum of angles is $360 \degree $ & $\checked$ & $\checked$ & $\checked$ & $\checked$\\
\hline
Opposite angles are $\cong$ .  & $\checked$ & $\checked$ & $\checked$ & $\checked$\\
\hline
All four angles are right. & $\checked$ & & $\checked$ &\\
\hline
Consecutive angles are supplementary. & $\checked$ & $\checked$ & $\checked$ & $\checked$\\
\hline
\end{tabularx} 

\par\vskip1pt

\begin{tabularx}{\textwidth}{|>{\hsize=1.8\hsize}X|>{\hsize=0.9\hsize}Y|>{\hsize=0.8\hsize}Y|>{\hsize=0.7\hsize}Y|>{\hsize=0.8\hsize}Y|}
\hline
\textbf{The Diagonals} & & & &\\
\hline
Diagonals bisect each other. & $\checked$ & $\checked$ & $\checked$ & $\checked$\\
\hline
Diagonals are congruent.  & $\checked$ & & $\checked$ &\\
\hline
Diagonals are perpendicular.  & & $\checked$ & $\checked$ &\\
\hline
Diagonals bisect opposite $\angle s$.  & & $\checked$ & $\checked$ &\\
\hline
\end{tabularx} 
}
\end{minipage}}
\end{center}  
%}}
%\end{minipage}
%\end{center} 


