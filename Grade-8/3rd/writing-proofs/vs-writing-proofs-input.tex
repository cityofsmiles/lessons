
\begin{center}
\textbf{Writing Proofs 
}
\end{center}

\vspce

Proof: a form of logical reasoning in which each statement is organized and backed up by the reasons 
\vspce 

Postulate: a statement that is accepted without proof 

\vspce 

Theorem: a statement that is accepted after it is proved deductively

\vspce 

Ways of Writing Proofs 
\begin{enumerate}[label = \arabic*. ]
\item \hspce Flow-Chart Proof
\vspce
\item \hspce Two-Column Proof
\vspce
\item \hspce Paragraph Form Proof
\end{enumerate} 

\vspce 
\newpage
Definitions: 
\begin{enumerate}[label = \arabic*. ]
\item Definition of Betweenness: If $B$ is between $\overline{AC}$, then $\overline{AC}=AB + BC $.
\item Midpoint: If $B$ is the midpoint of $\overline{AC}$, then $AB = BC$.
\item Segment Bisector: If a line, ray or another segment bisects the segment $AB$ at $X$, then $AX \cong BX$.
\item Angle Bisector: If $\overrightarrow{BD}$ bisects $\angle{ABC} $, then $\angle{ABD} \cong \angle{DBC} $. 
\item Right Angle: If $\angle{A}$ is a right angle, then $m\angle{A}= 90\degree $.
\item Perpendicular Line Segments: If $\overline{AB}\perp \overline{AC}$, then $\angle{BAC}$ is a right angle.
\item Complementary Angles: If $\angle{A}$ and $\angle{B}$ are complementary angles, then $m\angle{A} + m\angle{B} = 90\degree$.
\item Supplementary Angles: If $\angle{A}$ and $\angle{B}$ are supplementary angles, then $m\angle{A} + m\angle{B} = 180\degree$.
\item Linear Pair: If two angles are adjacent such that two of the rays are opposite, then they form a linear pair.
\item Definition of Congruent Segments: If $\overline{AB}\cong\overline{CD}$, then $AB = CD$. 
\item Definition of Congruent Angles: If $\angle{A} \cong \angle{B}$, then $m\angle{A} = m\angle{B}$. 
\end{enumerate} 

\vspce 

Properties: 
\begin{enumerate}[label = \arabic*. ]
\item Addition Property of Equality: If $a = b$, then $a + c = b + c$.
\item Subtraction Property of Equality: If $a = b$, then $a - c = b - c$.
\item Multiplication Property of Equality: If $a = b$, then $ac = bc$.
\item Division Property of Equality: If $a = b$ and $c\neq 0$, then $\displaystyle \frac{a}{c} = \displaystyle \frac{b}{c}$.
\item Reflexive Property of Equality: If $a$ is any real number, then $a = a$.
\item Symmetric Property:  If $a=b$,then $b = a$.
\item Transitive Property: If $a=b$ and $b=c$, then $a=c$.
\item Substitution Property: If $a + b = c$ and $b = x$, then $a + x = c$.
\end{enumerate} 








