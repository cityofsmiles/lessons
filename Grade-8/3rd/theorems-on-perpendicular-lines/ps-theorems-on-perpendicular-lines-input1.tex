\textbf{Practice Exercises}

\vspce

The adjoining figure consists of 3 coplanar lines passing through $O$ with $\overline{AB} \perp \overline{CD}$. Determine each statement as true or false.

%\begin{center}
%\vspace*{2ex}
%\scalebox{1}{
%\noindent\begin{minipage}{\textwidth}
{\begin{enumerate}[label = \arabic*. ]
%\begin{multicols}{2}
%1
\item	$m\angle{AOC}=90\degree$ 
\item	$m\angle{FOD}= m\angle{AOD} - m\angle{AOF}$ 
\item	$\angle{COF}$ is an acute angle
\item	$\overleftrightarrow{EF} \perp \overleftrightarrow{CD}$ 
\item	$\overrightarrow{OC}$ and $\overrightarrow{OF}$ are opposite rays 
\item	$\angle{AOF}$ and $\angle{AOD}$ are adjacent angles
\item	$\angle{AOD}$ is a right angle
\item	$\angle{AOC}$ and $\angle{AOD}$ are congruent\\ adjacent supplementary angles.
\item	The exterior sides of $\angle{AOF}$ and $\angle{FOD}$ lie in perpendicular lines.
\item	$\overrightarrow{OB} \perp \overrightarrow{OD} $
%\end{multicols} 
\end{enumerate}}
%\end{minipage}}
%\end{center}  

\vspace*{-14.5em}\hspace*{18em}\begin{tikzpicture}

\coordinate (O) at (0,0);

\node[anchor=south east, inner sep=2pt, rotate=0] (O-label) at (O) {$O$}; 

\foreach \ang/\label/\dir in {0/D/south,45/F/north west,90/A/west,180/C/south,225/E/north west,270/B/west} {

\coordinate (\label) at ($(O) + (\ang:2cm)$); 

\draw[line width=0.3mm, ->, >={Latex[round]}] (O) -- (\label);  

\node[anchor=\dir, inner sep=2pt, rotate=0] (label-\label) at ($(O)+ (\ang:1.6cm)$) {$\label$};
}

\end{tikzpicture} 
 
\vspace*{2em}


     





