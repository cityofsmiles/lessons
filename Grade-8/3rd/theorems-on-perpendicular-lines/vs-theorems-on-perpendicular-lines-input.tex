\begin{center}
\textbf{Theorems on Perpendicular Lines}
\end{center}

\vspace*{1ex}

%\begin{center}
%\vspace*{-2ex}
\scalebox{0.9}{
\noindent\begin{minipage}{\textwidth}
{

%\end{center} 
\textbf{Perpendicular Lines:} two lines that meet to form congruent adjacent angles

\vspce 

\textbf{Parallel Lines:} lines that lie in the same plane and have no point in common

\vspce 

\textbf{Theorems on Perpendicular Lines:}
\begin{enumerate}[label = \arabic*. ]
%1
\item If two lines are perpendicular, then they form four right angles.
%2
\item If two lines meet to form a right angle, the lines are perpendicular.
%3
\item In a plane, through a given point of a line, there is exactly one line perpendicular to the line.
\end{enumerate}   
}
\end{minipage}}




