\def\figdir{/storage/emulated/0/Documents/documents/latex/1920/Grade-8/3rd/inequalities-in-one-triangle/f}


\begin{enumerate}[label = \arabic*. ]
%D
\item[D. ] \hspce Give the range of the possible length of the third side of $\triangle ABC$. 

\vspace*{5ex}\hspace*{17em}\begin{tikzpicture}[remember picture, overlay] 

\def\len1{\lenA}

\draw[line width=0.3mm] (0,0) coordinate (b)  -- (50:0.6*\len1) coordinate (a) node[midway, anchor=south east, inner sep=0.07*\leninnersep, rotate=0] (c-mid) {$ c$} -- (0:\len1) coordinate (c) node[midway, anchor=south west, inner sep=0.07*\leninnersep, rotate=0] (b-mid) {$ b$} -- cycle node[midway, anchor=north, inner sep=0.07*\leninnersep, rotate=0] (a-mid) {$ a$}; 

\node[anchor=south, inner sep=0.07*\leninnersep, rotate=0] (a-label) at (a) {$ A$};

\node[anchor=east, inner sep=0.07*\leninnersep, rotate=0] (b-label) at (b) {$ B$};

\node[anchor=west, inner sep=0.07*\leninnersep, rotate=0] (c-label) at (c) {$ C$}; 

\end{tikzpicture} 
\vspace*{-7ex}

$
\begin{array}{llll}
1. \phantom{i} & a=9 & \phantom{mn} & b=7 \\
2. &	b=12 &	& c=29\\
3. &	a=15.2 &	& b=19.8 \\
4. & a=128.25 & & c=74.5 \\
5. &	a=3 \displaystyle \frac{2}{5} &	& c=2 \displaystyle \frac{1}{2}\\
\end{array}
$

%E
\item[E. ] \hspce Find the measure of the indicated angle. 
\begin{enumerate}[label = \arabic*. ]
%1
\item $m\angle{1} = 48\degree$; $m\angle{2} = 46\degree$; $ m\angle{4} = \blank$
%2
\item $m\angle{1} = 29\degree$; $ m\angle{2} = 52\degree$; $ m\angle{4} = \blank$
%3
\item $m\angle{2} = 18\degree$; $ m\angle{4} = 127\degree$; $ m\angle{1} = \blank$
%4
\item $m\angle{4} = 109\degree$; $ m\angle{1} = 86\degree$; $ m\angle{2} = \blank$
%5 
\item $m\angle{1} = (x+5)\degree$; $ m\angle{2} = (2x+51)\degree$; $ m\angle{4} = (5x-10)\degree$; $ x = \blank$
\end{enumerate}  
\end{enumerate}   

%\begin{center}

\vspace*{-19ex}\hspace*{22em}\begin{tikzpicture}

\def\len2{\lenA}

\draw[line width=0.3mm] (0,0) coordinate (1) node[yshift=-0.07*\leninnersep, inner sep=5pt, rotate=0, anchor=south west] (1-label) {$ 1$} -- (70:\len2) coordinate (2) node[xshift=2pt, inner sep=6pt, rotate=0, anchor=north] (2-label) {$ 2$} -- (0:1.2*\len2) coordinate (3) edge[-Latex] (0:1.7*\len2) node[yshift=-4pt, xshift=-2pt, inner sep=5pt, rotate=0, anchor=south east] (3-label) {$ 3$} node[yshift=0pt, xshift=-2pt, inner sep=2pt, rotate=0, anchor=south west] (4-label) {$ 4$}  -- cycle;  

\end{tikzpicture} 
\vspace*{4ex}

%\end{center} 