\def\figdir{/storage/emulated/0/Documents/documents/latex/1920/Grade-8/3rd/triangle-congruence-postulates/f}


%\textbf{Practice Exercises}
\textbf{Problem Set}

\vspce
A. Fill in the blanks then indicate the congruence postulate used. 

%\begin{center}
%\vspace*{2ex}
%\scalebox{0.5}{
%\noindent\begin{minipage}{0.3\textwidth}
{\begin{enumerate}[label = \arabic*. ]
\begin{multicols}{3}
%1
\item \begin{tikzpicture}

\def\len1{\lenB}

\draw[line width=0.3mm] (0,0) coordinate (u)  -- (40:0.7*\len1) coordinate (d) -- (0:1.2*\len1) coordinate (t) -- (-40:0.7*\len1) coordinate (s) -- cycle -- (t) ;  

\node[anchor=east, inner sep=2pt, rotate=0] (u-label) at (u) {$ U$}; 

\node[anchor=south, inner sep=2pt, rotate=0] (d-label) at (d) {$ D$}; 

\node[anchor=west, inner sep=2pt, rotate=0] (t-label) at (t) {$ T$}; 

\node[anchor=north, inner sep=2pt, rotate=0] (s-label) at (s) {$ S$}; 

\pic [draw, line width=0.3mm, angle radius=0.15*\len1] {angle=d--t--u}; 

\pic [draw, line width=0.3mm, angle radius=0.13*\len1] {angle=u--t--s}; 

\pic [draw, line width=0.3mm, angle radius=0.14*\len1] {angle=s--u--t}; 

\pic [draw, line width=0.3mm, angle radius=0.16*\len1] {angle=s--u--t}; 

\pic [draw, line width=0.3mm, angle radius=0.13*\len1] {angle=t--u--d}; 

\pic [draw, line width=0.3mm, angle radius=0.15*\len1] {angle=t--u--d}; 

\end{tikzpicture} 
%2
\item \begin{tikzpicture}

\def\len2{0.8*\lenB}

\draw[line width=0.3mm] (0,0) coordinate (q)  -- (85:0.6*\len2) coordinate (r) -- (0:1.3*\len2) coordinate (s) --++ (-95:0.6*\len2) coordinate (d) -- cycle --(s) ;  

\node[anchor=east, inner sep=2pt, rotate=0] (q-label) at (q) {$ Q$}; 

\node[anchor=south, inner sep=2pt, rotate=0] (r-label) at (r) {$ R$}; 

\node[anchor=west, inner sep=2pt, rotate=0] (s-label) at (s) {$ S$}; 

\node[anchor=north, inner sep=2pt, rotate=0] (d-label) at (d) {$ D$}; 

\coordinate (tick1) at ($(q)!0.5!(r)$) {};

\coordinate (tick2) at ($(s)!0.5!(d)$) {};  

\tikzset{twotick/.pic={\draw[line width=0.3mm] ($(-0.02*\len2, 0)+(0, 0.06*\len2)$) -- ($(-0.02*\len2, 0)-(0, 0.06*\len2)$) ($(0.02*\len2, 0)+(0, 0.06*\len2)$) -- ($(0.02*\len2, 0)-(0, 0.06*\len2)$) ; }}

\pic[rotate=85] at (tick1) [pic type = twotick];  

\pic[rotate=85] at (tick2) [pic type = twotick];   

\pic [draw, line width=0.3mm, angle radius=0.15*\len2] {angle=s--q--r}; 

\pic [draw, line width=0.3mm, angle radius=0.15*\len2] {angle=q--s--d}; 

\end{tikzpicture} 
%3
\item \begin{tikzpicture}%[remember picture, overlay] 

\def\len3{\lenB}

\draw[line width=0.3mm] (0,0) coordinate (v)  -- (130:0.8*\len3) coordinate (u) -- (180:0.4*\len3) coordinate (w) -- (0:0.8*\len3) coordinate (k) -- (-50:0.4*\len3) coordinate (m) -- cycle;  

\node[anchor=south west, inner sep=2pt, rotate=0] (v-label) at (v) {$ V$}; 

\node[anchor=south, inner sep=7pt, rotate=0] (u-label) at (u) {$ U$}; 

\node[anchor=north east, inner sep=2pt, rotate=0] (w-label) at (w) {$ W$}; 

\node[anchor=west, inner sep=2pt, rotate=0] (k-label) at (k) {$ K$}; 

\node[anchor=north, inner sep=2pt, rotate=0] (m-label) at (m) {$ M$}; 

\coordinate (tick1) at ($(u)!0.5!(v)$) {}; 

\coordinate (tick2) at ($(k)!0.5!(v)$) {};  

\tikzset{twotick/.pic={\draw[line width=0.3mm] ($(-0.02*\len3, 0)+(0, 0.06*\len3)$) -- ($(-0.02*\len3, 0)-(0, 0.06*\len3)$) ($(0.02*\len3, 0)+(0, 0.06*\len3)$) -- ($(0.02*\len3, 0)-(0, 0.06*\len3)$) ; }}

\pic[rotate=130] at (tick1) [pic type = twotick];

\pic[rotate=0] at (tick2) [pic type = twotick];   

\pic [draw, line width=0.3mm, angle radius=0.15*\len3] {angle=w--u--v}; 

\pic [draw, line width=0.3mm, angle radius=0.15*\len3] {angle=v--k--m}; 

\end{tikzpicture} 
%4
\item \begin{tikzpicture}

\def\len4{\lenB}

\draw[line width=0.3mm] (0,0) coordinate (a)  -- (180:0.6*\len4) coordinate (b) -- (90:0.9*\len4) coordinate (c) -- cycle;  

\node[anchor=north, inner sep=2pt, rotate=0] (a-label) at (a) {$ A$}; 

\node[anchor=north, inner sep=2pt, rotate=0] (b-label) at (b) {$ B$}; 

\node[anchor=south, inner sep=7pt, rotate=0] (c-label) at (c) {$ C$}; 

\draw[line width=0.3mm, xshift=0.8*\len4] (0,0) coordinate (j)  -- (180:0.6*\len4) coordinate (k) -- (90:0.9*\len4) coordinate (l) -- cycle;

\node[anchor=north, inner sep=2pt, rotate=0] (j-label) at (j) {$ J$}; 

\node[anchor=north, inner sep=2pt, rotate=0] (k-label) at (k) {$ K$}; 

\node[anchor=south, inner sep=7pt, rotate=0] (l-label) at (l) {$ L$}; 

\coordinate (tick1) at ($(a)!0.5!(b)$) {}; 

\coordinate (tick2) at ($(j)!0.5!(k)$) {};  

\tikzset{onetick/.pic={\draw[line width=0.3mm] ($(0,0)+(0, 0.06*\len4)$) -- ($(0,0)-(0, 0.06*\len4)$) ; }}

\pic[rotate=0] at (tick1) [pic type = onetick]; 

\pic[rotate=0] at (tick2) [pic type = onetick];   

\coordinate (tick3) at ($(a)!0.5!(c)$) {};  

\coordinate (tick4) at ($(l)!0.5!(j)$) {};  

\tikzset{twotick/.pic={\draw[line width=0.3mm] ($(-0.02*\len4, 0)+(0, 0.06*\len4)$) -- ($(-0.02*\len4, 0)-(0, 0.06*\len4)$) ($(0.02*\len4, 0)+(0, 0.06*\len4)$) -- ($(0.02*\len4, 0)-(0, 0.06*\len4)$) ; }}

\pic[rotate=90] at (tick3) [pic type = twotick];

\pic[rotate=90] at (tick4) [pic type = twotick];  

\begin{scope} [rotate=90]
\draw[line width=0.3mm] (a) rectangle ++(0.13*\len4,0.13*\len4) node[transform shape]{};
\end{scope} 

\begin{scope} [rotate=90]
\draw[line width=0.3mm] (j) rectangle ++(0.13*\len4,0.13*\len4) node[transform shape]{};
\end{scope} 

\end{tikzpicture} 
%5
\item \begin{tikzpicture}

\def\len5{1.1*\lenB}

\draw[line width=0.3mm] (0,0) coordinate (f)  -- (95:0.6*\len5) coordinate (o) -- (180:0.5*\len5) coordinate (h)  -- (0:0.5*\len5) coordinate (d) -- (-85:0.6*\len5) coordinate (g) -- cycle;  

\node[anchor=south west, inner sep=2pt, rotate=0] (f-label) at (f) {$ F$};

\node[anchor=south, inner sep=2pt, rotate=0] (o-label) at (o) {$ O$};

\node[anchor=east, inner sep=2pt, rotate=0] (h-label) at (h) {$ H$};

\node[anchor=west, inner sep=2pt, rotate=0] (d-label) at (d) {$ D$};

\node[anchor=north, inner sep=2pt, rotate=0] (g-label) at (g) {$ G$};

\coordinate (tick1) at ($(o)!0.5!(f)$) {};  

\coordinate (tick2) at ($(f)!0.5!(g)$) {};  

\tikzset{twotick/.pic={\draw[line width=0.3mm] ($(-0.02*\len5, 0)+(0, 0.06*\len5)$) -- ($(-0.02*\len5, 0)-(0, 0.06*\len5)$) ($(0.02*\len5, 0)+(0, 0.06*\len5)$) -- ($(0.02*\len5, 0)-(0, 0.06*\len5)$) ; }}

\pic[rotate=95] at (tick1) [pic type = twotick]; 

\pic[rotate=95] at (tick2) [pic type = twotick];   

\coordinate (tick3) at ($(h)!0.5!(f)$) {};  

\coordinate (tick4) at ($(f)!0.5!(d)$) {};  

\tikzset{threetick/.pic={\draw[line width=0.3mm] ($(-0.03*\len5, 0)+(0, 0.06*\len5)$) -- ($(-0.03*\len5, 0)-(0, 0.06*\len5)$) ($(0,0)+(0, 0.06*\len5)$) -- ($(0,0)-(0, 0.06*\len5)$)  ($(0.03*\len5, 0)+(0, 0.06*\len5)$) -- ($(0.03*\len5, 0)-(0, 0.06*\len5)$) ; }}

\pic[rotate=0] at (tick3) [pic type = threetick]; 

\pic[rotate=0] at (tick4) [pic type = threetick];   

\end{tikzpicture} 
%6
\item \begin{tikzpicture}

\def\len6{1.1*\lenB}

\draw[line width=0.3mm] (0,0) coordinate (v)  -- (180:0.5*\len6) coordinate (w) -- (90:\len6) coordinate (x) --++ (0:0.5*\len6) coordinate (k) -- cycle -- (x) ; 

\node[anchor=north, inner sep=2pt, rotate=0] (v-label) at (v) {$ V$};

\node[anchor=north east, inner sep=2pt, rotate=0] (w-label) at (w) {$ W$};

\node[anchor=south, inner sep=2pt, rotate=0] (x-label) at (x) {$ X$};

\node[anchor=south west, inner sep=2pt, rotate=0] (k-label) at (k) {$ K$};

\begin{scope} [rotate=90]
\draw[line width=0.3mm] (v) rectangle ++(0.13*\len6,0.13*\len6) node[transform shape]{};
\end{scope} 

\begin{scope} [rotate=-90]
\draw[line width=0.3mm] (x) rectangle ++(0.13*\len6,0.13*\len6) node[transform shape]{};
\end{scope} 

\coordinate (tick1) at ($(w)!0.5!(v)$) {};  

\coordinate (tick2) at ($(x)!0.5!(k)$) {};  

\tikzset{onetick/.pic={\draw[line width=0.3mm] ($(0,0)+(0, 0.06*\len6)$) -- ($(0,0)-(0, 0.06*\len6)$) ; }}

\pic[rotate=0] at (tick1) [pic type = onetick];

\pic[rotate=0] at (tick2) [pic type = onetick];   

\end{tikzpicture} 
%7
%\item \begin{tikzpicture}

\def\len7{\lenB}

\draw[line width=0.3mm] (0,0) coordinate (d)  -- (-55:0.7*\len7) coordinate (e) -- (0:0.7*\len7) coordinate (f) -- cycle; 

\node[anchor=east, inner sep=2pt, rotate=0] (d-label) at (d) {$ D$};

\node[anchor=north, inner sep=2pt, rotate=0] (e-label) at (e) {$ E$};

\node[anchor=west, inner sep=2pt, rotate=0] (f-label) at (f) {$ F$};

\draw[line width=0.3mm, xshift=0.7*\len7, yshift=0.5*\len7, remember picture, overlay]  (0,0) coordinate (l) -- (0:0.7*\len7) coordinate (j) --++ (235:0.7*\len7) coordinate (k) -- cycle; 

\node[anchor=west, inner sep=2pt, rotate=0] (j-label) at (j) {$ J$};

\node[anchor=north, inner sep=2pt, rotate=0] (k-label) at (k) {$ K$};

\node[anchor=east, inner sep=2pt, rotate=0] (l-label) at (l) {$ L$};

\pic [draw, line width=0.3mm, angle radius=0.15*\len7] {angle=e--d--f};

\pic [draw, line width=0.3mm, angle radius=0.15*\len7] {angle=l--j--k};

\pic [draw, line width=0.3mm, angle radius=0.15*\len7] {angle=f--e--d};

\pic [draw, line width=0.3mm, angle radius=0.12*\len7] {angle=f--e--d};

\pic [draw, line width=0.3mm, angle radius=0.15*\len7] {angle=j--k--l};

\pic [draw, line width=0.3mm, angle radius=0.12*\len7] {angle=j--k--l};

\coordinate (tick1) at ($(d)!0.5!(e)$) {};  

\coordinate (tick2) at ($(j)!0.5!(k)$) {};  

\tikzset{onetick/.pic={\draw[line width=0.3mm] ($(0,0)+(0, 0.06*\len7)$) -- ($(0,0)-(0, 0.06*\len7)$) ; }}

\pic[rotate=-55] at (tick1) [pic type = onetick]; 

\pic[rotate=55] at (tick2) [pic type = onetick];   

\end{tikzpicture} 
%8
%\item \begin{tikzpicture}

\def\len8{\lenB}

\draw[line width=0.3mm, remember picture, overlay] (0,0) coordinate (m)  -- (60:0.5*\len8) coordinate (l) -- (0:0.7*\len8) coordinate (k) -- cycle; 

\node[anchor=east, inner sep=2pt, rotate=0] (m-label) at (m) {$ M$};

\node[anchor=south, inner sep=2pt, rotate=0] (l-label) at (l) {$ L$};

\node[anchor=west, inner sep=2pt, rotate=0] (k-label) at (k) {$ K$};

\draw[line width=0.3mm, rotate=-90, xshift=-1*\len8,
yshift=0.9*\len8
] (0,0) coordinate (d)  -- (60:0.5*\len8) coordinate (i) -- (0:0.7*\len8) coordinate (r) -- cycle; 

\node[anchor=south, inner sep=2pt, rotate=0] (d-label) at (d) {$ D$};

\node[anchor=west, inner sep=2pt, rotate=0] (i-label) at (i) {$ I$};

\node[anchor=north, inner sep=2pt, rotate=0] (r-label) at (r) {$ R$};

\pic [draw, line width=0.3mm, angle radius=0.15*\len8] {angle=k--m--l}; 

\pic [draw, line width=0.3mm, angle radius=0.15*\len8] {angle=r--d--i}; 

\coordinate (tick1) at ($(l)!0.5!(m)$) {};  

\coordinate (tick2) at ($(d)!0.5!(i)$) {};  

\tikzset{onetick/.pic={\draw[line width=0.3mm] ($(0,0)+(0, 0.06*\len8)$) -- ($(0,0)-(0, 0.06*\len8)$) ; }}

\pic[rotate=60] at (tick1) [pic type = onetick];

\pic[rotate=-30] at (tick2) [pic type = onetick]; 

\coordinate (tick3) at ($(k)!0.5!(m)$) {};  

\coordinate (tick4) at ($(d)!0.5!(r)$) {};  

\tikzset{twotick/.pic={\draw[line width=0.3mm] ($(-0.02*\len8, 0)+(0, 0.06*\len8)$) -- ($(-0.02*\len8, 0)-(0, 0.06*\len8)$) ($(0.02*\len8, 0)+(0, 0.06*\len8)$) -- ($(0.02*\len8, 0)-(0, 0.06*\len8)$) ; }}

\pic[rotate=0] at (tick3) [pic type = twotick];

\pic[rotate=90] at (tick4) [pic type = twotick];    

\end{tikzpicture} 

\end{multicols} 
\end{enumerate}}
%\end{minipage}}
%\end{center}  

    
\begin{enumerate}[label = \arabic*. ]
\begin{multicols}{3}

%1
\item $\overline{TU} \cong$ \blank \\
$\angle{DUT} \cong$ \blank \\
$\angle{DTU} \cong$ \blank \\
$\triangle{DUT} \cong$ \blank \\
Postulate: \blank
%2
\item $\overline{QS} \cong$ \blank \\
$\overline{QR} \cong$ \blank \\
$\angle{RQS} \cong$ \blank \\
$\triangle{RQS} \cong$ \blank \\
Postulate: \blank
%3
\item $\overline{UV} \cong$ \blank \\
$\angle{U} \cong$ \blank \\
$\angle{UVW} \cong$ \blank \\
$\triangle{UVW} \cong$ \blank \\
Postulate: \blank
%4
\item $\overline{AB} \cong$ \blank \\
$\overline{AC} \cong$ \blank \\
$\angle{A} \cong$ \blank \\
$\triangle{ABC} \cong$ \blank \\
Postulate: \blank
%5
\item $\overline{FO} \cong$ \blank \\
$\overline{HF} \cong$ \blank \\
$\angle{HFO} \cong$ \blank \\
$\triangle{HFO} \cong$ \blank \\
Postulate: \blank
%6
\item $\overline{KX} \cong$ \blank \\
$\overline{XV} \cong$ \blank \\
$\angle{KXV} \cong$ \blank \\
$\triangle{KXV} \cong$ \blank \\
Postulate: \blank

\end{multicols} 
\end{enumerate}   
%}}


