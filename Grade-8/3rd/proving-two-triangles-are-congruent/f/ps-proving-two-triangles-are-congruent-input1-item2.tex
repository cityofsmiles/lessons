\begin{tikzpicture}

\def\len2{\lenA}

\coordinate (e) at (0,0);

\coordinate (d) at ($(e) + (0:0.8*\len2)$); 

\coordinate (c) at ($(e) + (40:0.6*\len2)$);

\coordinate (a) at ($(e) + (180:0.8*\len2)$);

\coordinate (b) at ($(e) -(40:0.6*\len2)$); 

\draw[line width=0.3mm] (a) -- (b) -- (c) -- (d) -- cycle; 

\node[anchor=east, inner sep=0.07*\leninnersep, rotate=0] (a-label) at (a) {$ A$}; 

\node[anchor=north, inner sep=0.07*\leninnersep, rotate=0] (b-label) at (b) {$ B$}; 

\node[anchor=south, inner sep=0.07*\leninnersep, rotate=0] (c-label) at (c) {$ C$}; 

\node[anchor=west, inner sep=0.07*\leninnersep, rotate=0] (d-label) at (d) {$ D$}; 

\node[anchor=north, inner sep=0.07*\leninnersep, rotate=0, xshift=3pt] (e-label) at (e) {$ E$}; 

\end{tikzpicture} 


Given: $E$ is the midpoint of segments $AD$ and $BC$. \\
Prove: $\triangle AEB \cong \triangle DEC$\\
Proof: 
\begin{center}

\noindent\begin{minipage}{27em}

\begin{tabularx}{\textwidth}{|Y|Y|}
\hline
\textbf{Statements} & \textbf{Reasons}  \\
\hline
\end{tabularx} 

\par\vskip1pt

\begin{tabularx}{\textwidth}{|X|X|}
\hline
1. $E$ is the midpoint of segments $AD$ and $BC$. & 1.  \\
\hline
2. $\overline{AE} \cong \overline{DE} $ & 2.  \\
\hline
3. $\angle{AEB} \cong \angle{DEC}$ & 3.  \\
\hline
4. $\overline{BE} \cong \overline{CE} $ & 4.  \\
\hline
5. $\triangle AEB \cong \triangle DEC$ & 5.  \\
\hline
\end{tabularx} 
\end{minipage}

\end{center} 

