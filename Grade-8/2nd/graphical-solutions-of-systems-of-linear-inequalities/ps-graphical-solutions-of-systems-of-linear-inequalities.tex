% cd /storage/emulated/0/Documents/documents/latex/1920/Grade-8/2nd/graphical-solutions-of-systems-of-linear-inequalities && pdflatex ps-graphical-solutions-of-systems-of-linear-inequalities.tex && termux-open ps-graphical-solutions-of-systems-of-linear-inequalities.pdf

% cd /storage/emulated/0/Documents/documents/latex/1920/Grade-8/2nd/graphical-solutions-of-systems-of-linear-inequalities && clean-tex ps-graphical-solutions-of-systems-of-linear-inequalities-input1.tex


% cd /storage/emulated/0/Documents/documents/latex/1920/Grade-8/2nd/graphical-solutions-of-systems-of-linear-inequalities && convert -density 600 ps-graphical-solutions-of-systems-of-linear-inequalities.pdf -crop 2200x1700 -quality 100 -verbose ps-graphical-solutions-of-systems-of-linear-inequalities%02d.png

%2480.5x3508 portrait 2x2 2550x3300
%3508x2480.5 landscape 2x2 3300x2550 
%1653.7x2338.7 portrait 3x3 1700x2200
%landscape 3x3 2200x1700

% cd /storage/emulated/0/Documents/documents/latex/1819/grade10/visual/4th/graphical-solutions-of-systems-of-linear-inequalities && while inotifywait -e close_write ps-graphical-solutions-of-systems-of-linear-inequalities*.tex; do touch /storage/emulated/0/Android/data/com.termux/files/launch-termux.txt && printf '1' > /storage/emulated/0/Android/data/com.termux/files/launch-termux.txt && pdflatex ps-graphical-solutions-of-systems-of-linear-inequalities.tex && termux-open ps-graphical-solutions-of-systems-of-linear-inequalities.pdf; done

% cd /host-rootfs/storage/emulated/0/Documents/documents/latex/1819/grade10/visual/4th/graphical-solutions-of-systems-of-linear-inequalities && while inotifywait -e close_write ps-graphical-solutions-of-systems-of-linear-inequalities*.tex; do pdflatex ps-graphical-solutions-of-systems-of-linear-inequalities.tex  && printf "/storage/emulated/0/Documents/documents/latex/1819/grade10/visual/4th/graphical-solutions-of-systems-of-linear-inequalities/ps-graphical-solutions-of-systems-of-linear-inequalities.pdf" > /host-rootfs/storage/emulated/0/GNURoot/home/Scripts/file-to-launch.txt; done


\documentclass[10pt]{article}
\usepackage[letterpaper, landscape, right=0.25in, left=0.4in, top=0.25in, bottom=0.25in]{geometry}
\usepackage{xcolor}
\usepackage{anyfontsize}
\usepackage{enumitem}
\usepackage{multicol}
\usepackage{bm} 
\usepackage{amsmath}
%\usepackage{amsfonts,dsfont}% for \mathds 
\usepackage{tabularx} 
\usepackage{gensymb}
\usepackage{multirow}
\usepackage{graphicx, tipa}
\usepackage{tikz}
\usetikzlibrary{angles,quotes}
\usepackage{pgfplots} 
\usetikzlibrary{calc}
\pgfplotsset{compat=newest}
\usetikzlibrary{arrows.meta}
\usetikzlibrary{intersections}
\usetikzlibrary{decorations.pathreplacing}
\usepackage{flafter}
\usepackage{amsmath,amssymb,cancel,units}
\usepackage{microtype} % nicer output 
\usepackage{hfoldsty} % nicer output 
\usepackage{fixltx2e} 
\usepackage{mathptmx}
%\usepackage{booktabs}
\usepackage{numprint}
\usepackage[utf8]{inputenc} 
\usepackage[T1]{fontenc}
%\usepackage{siunitx} 
%\sisetup{detect-all}
\usepackage{stackengine}
 
\newcommand\pesos{\stackengine{-1.28ex}{P}{\stackengine{-1.2ex}{$-$}{$-$}{O}{c}{F}{F}{S}}{O}{c}{F}{T}{S}} 


\def\radA{3.6cm}

\def\radB{3.6cm}

%\def\thirdrad{8cm}

\pagenumbering{gobble}
%\linespread{0.9}
\newcommand{\vspce}{\vspace{0.75ex}}
\newcommand{\hspce}{\hspace{0.5em}}
\newcommand{\blank}{\underline{\hspace{2em}}}%{\rule{1em}{0.15ex}}
\newcommand{\arc}[1]{{% 
\setbox9=\hbox{#1}% 
\ooalign{\resizebox{\wd9}{\height}{\texttoptiebar{\phantom{A}}}\cr#1}}}

\newcolumntype{C}{ >{\centering\arraybackslash} X}




\begin{document}
\boldmath
{\fontsize{37}{39}\fontfamily{pnc}\selectfont {

\textbf{Practice Exercises}

\vspce

A. Identify whether each ordered pair is a solution to the given system of linear inequality. Write \emph{YES} if it is or \emph{NO} if it is not.
\begin{multicols}{2}
\begin{enumerate}[label = \arabic*. ]
%1
\item $\left\{\begin{array}{rcl}
5x+y & > & 3 \\
y & \leq & x-4 \\
\end{array}\right.$
\begin{enumerate}[label = \alph*. ]
%1
\item \hspce (-1, 2)
%2
\item \hspce (0, 0)
%3
\item \hspce (-3, 2) 
\end{enumerate}   

\vfill
\columnbreak

%2
\item $\left\{\begin{array}{ccl}
2x+5y & < & 10 \\
3x-4y & \geq & -8 \\
\end{array}\right.$

\begin{enumerate}[label = \alph*. ]
%1
\item \hspce (2, 1)
%2
\item \hspce (2, 0)
%3
\item \hspce $\big(\displaystyle \frac{1}{2}, 2\big)$
\end{enumerate}  

\end{enumerate}  
\end{multicols} 


%}} 

\newpage

%{\fontsize{38}{40}\fontfamily{pnc}\selectfont {

B. Solve each system of inequality graphically. 
\begin{enumerate}[label = \arabic*. ]
\begin{multicols}{2}
%1
\item $\left\{\begin{array}{rcl}
y & > & x+3 \\
y & \leq & -x+1 \\
\end{array}\right.$
%2
\item $\left\{\begin{array}{rcl}
y & > & -2x+5 \\
y & < & \dfrac{1}{4}x \\
\end{array}\right.$
%3
\item $\left\{\begin{array}{rcl}
x+y & \leq & 6 \\
x-y & > & 8 \\
\end{array}\right.$
%4
\item $\left\{\begin{array}{rcl}
x-2y & \geq & 10 \\
2x+y & \leq & -4 \\
\end{array}\right.$

\end{multicols} 
\end{enumerate}  

\newpage

%{\fontsize{38}{40}\fontfamily{pnc}\selectfont {

\textbf{Problem Set}

\vspce

Solve each system of inequality graphically. 
\begin{enumerate}[label = \arabic*. ]
\begin{multicols}{2}
%1
\item $\left\{\begin{array}{rcl}
x-y & \geq & 5 \\
2x+3y & \leq & 12 \\
\end{array}\right.$
%2
\item $\left\{\begin{array}{rcl}
x+y & \geq & 7 \\
3x-y & \leq & 10 \\
\end{array}\right.$
%3
\item $\left\{\begin{array}{rcl}
2x-y & \geq & -2 \\
y & < & x+4 \\
\end{array}\right.$
%4
\item $\left\{\begin{array}{rcl}
y & > & 2x-9 \\
y & < & 4x+1 \\
\end{array}\right.$

\end{multicols} 
\end{enumerate}  

}}

\newpage

{\fontsize{35}{38}\fontfamily{pnc}\selectfont {

\input{ps-graphical-solutions-of-systems-of-linear-inequalities-sol}

}}


\end{document}