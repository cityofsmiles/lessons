4. $
\begin{array}{cccccc}
\redcheck 	&	\redcheck 	& \redcheck &	\redcheck 	&&	\redcheck 	\\

	&	\textbf{Coin} 	&&	\textbf{Die} 	&&	\textbf{Outcomes } 	\\

\tikz[remember picture]{ \node[anchor=base,inner sep=0pt] (p) {}; }	&	\tikz[remember picture, overlay]{ \node[anchor=base,inner sep=0pt] (h) {H}; \draw[line width=1pt] (p) -- (h);} 		&&	\tikz[remember picture, overlay]{ \node[anchor=base,inner sep=0pt] (h1) {1}; \draw[line width=1pt] (h) -- (h1);} 	&&	H1	\\

	&		&&	\tikz[remember picture, overlay]{ \node[anchor=base,inner sep=0pt] (h2) {2}; \draw[line width=1pt] (h) -- (h2);} 	&&	H2	\\

	&		&&	\tikz[remember picture, overlay]{ \node[anchor=base,inner sep=0pt] (h3) {3}; \draw[line width=1pt] (h) -- (h3);} 	&&	H3	\\

	&		&&	\tikz[remember picture, overlay]{ \node[anchor=base,inner sep=0pt] (h4) {4}; \draw[line width=1pt] (h) -- (h4);} 	&&	H4	\\

	&		&&	\tikz[remember picture, overlay]{ \node[anchor=base,inner sep=0pt] (h5) {5}; \draw[line width=1pt] (h) -- (h5);} 	&&	H5	\\

	&		&&	\tikz[remember picture, overlay]{ \node[anchor=base,inner sep=0pt] (h6) {6}; \draw[line width=1pt] (h) -- (h6);} 	&&	H6	\\

	&		\tikz[remember picture, overlay]{ \node[anchor=base,inner sep=0pt] (t) {T}; \draw[line width=1pt] (p) -- (t);} 		&&	\tikz[remember picture, overlay]{ \node[anchor=base,inner sep=0pt] (t1) {1}; \draw[line width=1pt] (t) -- (t1);} 	&&	T1	\\

	&		&&	\tikz[remember picture, overlay]{ \node[anchor=base,inner sep=0pt] (t2) {2}; \draw[line width=1pt] (t) -- (t2);} 	&&	T2	\\

	&		&&	\tikz[remember picture, overlay]{ \node[anchor=base,inner sep=0pt] (t3) {3}; \draw[line width=1pt] (t) -- (t3);} 	&&	T3	\\

	&		&&	\tikz[remember picture, overlay]{ \node[anchor=base,inner sep=0pt] (t4) {4}; \draw[line width=1pt] (t) -- (t4);} 	&&	T4	\\

	&		&&	\tikz[remember picture, overlay]{ \node[anchor=base,inner sep=0pt] (t5) {5}; \draw[line width=1pt] (t) -- (t5);} 	&&	T5	\\

	&		&&	\tikz[remember picture, overlay]{ \node[anchor=base,inner sep=0pt] (t6) {6}; \draw[line width=1pt] (t) -- (t6);} 	&&	T6	\\

\end{array} 
$
12 \redcheck  
possible outcomes \redcheck 

$2 \times 6 \redcheck  = 
12$ \redcheck 
 possible outcomes \redcheck 
 
 5. $
\begin{array}{cccccc}
 \redcheck 	&	\redcheck 	& \redcheck &	\redcheck 	&&	\redcheck 	\\

	&	\textbf{Color} 	&&	\textbf{Type} 	&&	\textbf{Outcomes } 	\\

	\tikz[remember picture]{ \node[anchor=base,inner sep=0pt] (p) {}; } &	\tikz[remember picture, overlay]{ \node[anchor=base,inner sep=0pt] (r) {R}; \draw[line width=1pt] (p) -- (r);} 	&&	\tikz[remember picture, overlay]{ \node[anchor=base,inner sep=0pt] (r5) {5}; \draw[line width=1pt] (r) -- (r5);} 	&&	R, 5	\\

	&		&&	\tikz[remember picture, overlay]{ \node[anchor=base,inner sep=0pt] (r7) {7}; \draw[line width=1pt] (r) -- (r7);} 	&&	R, 7	\\

	&	\tikz[remember picture, overlay]{ \node[anchor=base,inner sep=0pt] (y) {Y}; \draw[line width=1pt] (p) -- (y);} 	&&	\tikz[remember picture, overlay]{ \node[anchor=base,inner sep=0pt] (y5) {5}; \draw[line width=1pt] (y) -- (y5);} 	&&	Y, 5	\\

	&		&&	\tikz[remember picture, overlay]{ \node[anchor=base,inner sep=0pt] (y7) {7}; \draw[line width=1pt] (y) -- (y7);} 	&&	Y, 7	\\

	&	\tikz[remember picture, overlay]{ \node[anchor=base,inner sep=0pt] (g) {G}; \draw[line width=1pt] (p) -- (g);} 	&&	\tikz[remember picture, overlay]{ \node[anchor=base,inner sep=0pt] (g5) {5}; \draw[line width=1pt] (g) -- (g5);} 	&&	G, 5	\\

	&		&&	\tikz[remember picture, overlay]{ \node[anchor=base,inner sep=0pt] (g7) {7}; \draw[line width=1pt] (g) -- (g7);} 	&&	G, 7	\\

	&	\tikz[remember picture, overlay]{ \node[anchor=base,inner sep=0pt] (b) {B}; \draw[line width=1pt] (p) -- (b);} 	&&	\tikz[remember picture, overlay]{ \node[anchor=base,inner sep=0pt] (b5) {5}; \draw[line width=1pt] (b) -- (b5);} 	&&	B, 5	\\

	&		&&	\tikz[remember picture, overlay]{ \node[anchor=base,inner sep=0pt] (b7) {7}; \draw[line width=1pt] (b) -- (b7);} 	&&	B, 7	\\

\end{array} 
$

8 \redcheck  
possible outcomes \redcheck 

$4 \times 2 \redcheck  = 
8$ \redcheck 
 possible outcomes \redcheck 
 
 B. 
 \begin{enumerate}[label = \arabic*. ]

\item % A restaurant offers four sizes of pizza, two types of crust, and eight toppings. How many possible combinations of pizza with one topping are there? 
$ 4\times 2\times  8\redcheck 
=64 $ \redcheck 
possible combinations \redcheck 

\item  %How many ways can 5 paintings be lined up on a wall? 
$5 \times 4 \times 3 \times 2 \times 1 \redcheck 
=120 $ \redcheck 
ways \redcheck 

\item  %Rob has 4 shirts, 3 pairs of pants, and 2 pairs of shoes that all coordinate. How many outfits can you put together? 
$4 \times 3 \times 2 \redcheck 
=24 $ \redcheck 
outfits \redcheck 

\item  %Grace loves to eat salad! How many salads can she put together if she can pick out one type of lettuce from 2 choices, one vegetable from 4 choices and one dressing from 7 choices? 
$2 \times 4 \times 7  \redcheck 
= 56 $ \redcheck 
salads \redcheck 

\item  %Motorcycle license plates have 2 letters followed by 4 numbers. 
\begin{enumerate}[label = \alph*. ]
%a
\item %If the same letter or number can be repeated, how many can be made?
$ 26 \times 26 \times 10 \times 10 \times 10 \times 10 \redcheck 
= 6,760,000$ \redcheck \\
license plates  \redcheck 

%b
\item% If the same letter CANNOT be repeated, how many can be made?
$ 26 \times 25 \times 10 \times 9 \times 8 \times 7   \redcheck 
= 3,276,000 $ \redcheck 
license plates  \redcheck 

%c
%\item 
\end{enumerate} 
 
\item % How many 5-digit numbers can be formed (using 0 – 9)? 
$9 \times 10 \times 10 \times 10 \times 10 \redcheck 
= 90,000$ \redcheck 
numbers \redcheck 

\item % How many 5-digit numbers can be formed if each one uses all the digits 0, 1, 2, 3, 4 without repetition? 
$ 4\times 4 \times 3 \times 2 \times 1 \redcheck 
=96 $ \redcheck 
numbers \redcheck 

\item%  In how many ways can 6 bicycles be parked in a row? 
$6 \times 5 \times 4 \times 3 \times 2 \times 1 \redcheck 
=720 $ \redcheck 
ways \redcheck 

\end{enumerate} 
 
 