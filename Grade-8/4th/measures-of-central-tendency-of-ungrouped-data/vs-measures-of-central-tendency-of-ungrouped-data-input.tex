\begin{center}
\textbf{Measures of Central Tendency of Ungrouped Data 
}
\end{center}

\vspace*{1ex}

\textbf{Mean (Arithmetic Mean)}
\begin{itemize}
\item[-] the most commonly used measure of central tendency

\item[-] the sum of measures $x$ divided by the number $N$ of measures in a variable
\end{itemize}  

\vspce 

To find the mean of an ungrouped data, use the formula
\[
\bar{x} = \displaystyle \frac{\Sigma x}{N} 
\] 

\begin{center}

\begin{tabular}{llcl}

where & $\Sigma x$ & = & the summation of $x$ \\

& $N$ & = & number of values of $x$ \\

\end{tabular} 
\end{center} 

\vspce 

\textbf{Median}: the middle value in a set of data arranged according to size/magnitude (either increasing or decreasing) 

\vspce 

To find the median of an ungrouped data, use the formula
\[
\tilde{x} = \displaystyle \frac{1}{2}(N+1)th
\] 

\vspce 

\textbf{Mode}
\begin{itemize} 
\item[-] the measure or value which occurs most frequently in a set of data
\item[-] the value with the greatest frequency
\end{itemize}  

To find the mode for a set of data:
\begin{enumerate}[label = \arabic*. ]
%1
\item select the measure that appears most often in the set; 
%2
\item if two or more measures appear the same number of times, then each of these 
values is a mode; and
%3
\item if every measure appears the same number of times, then the set of data has no 
mode.
\end{enumerate}   

