\begin{center}
\textbf{Experimental and Theoretical Probability
}
\end{center}

\vspace*{1ex}

\emph{Relative frequency}:  the frequency of the outcome expressed as percentage of the total number of trials

\vspce 

\emph{Experimental Probability}: 
\begin{itemize} 
\item a probability based on an experiment written as a ratio
\item compares the number of events that happen to the number of trials 
\end{itemize}  

\[
P(E) = \displaystyle \frac{\text{number of events}}{\text{total number of trials}} 
\] 

\vspce

\emph{Theoretical Probability}: a probability based on how the reality of an event will happen, written as the ratio of the number of ways the event can occur to the total number of possible outcomes in a sample space.
\[
P(E) = \displaystyle \frac{\text{number of ways the event can happen}}{\text{total number of possible outcomes}} 
\] 

\vspce 

If the number of trials becomes bigger, the theoretical probability is the value which the experimental probability approaches.

