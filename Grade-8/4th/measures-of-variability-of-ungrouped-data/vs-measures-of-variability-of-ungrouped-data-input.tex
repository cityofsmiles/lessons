\begin{center}
\textbf{Measures of Variability of Ungrouped Data}
\end{center}

\vspace*{1ex}

\textbf{Measures of Dispersion or Variability} refer to the spread  of the values  about the mean. 

\textbf{Ways of Measuring Variability}

1. \textbf{Range:} the difference between the largest and smallest values in that dataset. It is the simplest measure of variability. 

$$ R = HV - LV $$

\begin{center}
\begin{tabular}{rrcl}

where & $R$ & = & the range\\

& $HV$ & = &the highest value\\

& $LV$ & = & the lowest value\\

\end{tabular} 
\end{center} 


2. \textbf{Mean Absolute Deviation (MAD):} the dispersion of a set of data about the average of these data 

$$ MAD = \dfrac{\sum | x - \overline{x}|}{N} $$

\begin{center}
\begin{tabular}{rrcl}

where & $MAD$ & = & the mean absolute deviation \\
& $x$ & = & the individual score \\
& $\overline{x} $ & = & the mean\\

& $N$ & = & the number of scores\\

\end{tabular} 
\end{center} 


3. \textbf{Variance: } the average squared difference of the values from the mean. 

$$ \sigma^2 = \dfrac{\sum (x - \mu)^2 }{N} \text{ or } s^2 = \dfrac{\sum (x - \overline{x})^2 }{n} $$

\begin{center}
\begin{tabular}{rrcl}

where & $\sigma^2 $ & = & the population variance \\
& $s^2 $ & = & the sample variance\\
& $\mu$ & = & the population mean \\
& $\overline{x} $ & = & the sample mean\\
& $N$ & = & the number of scores in the population \\
& $n$ & = & the number of scores in the sample\\
\end{tabular} 
\end{center} 

4. \textbf{Standard Deviation: } the standard or typical difference between each data point and the mean. It is just the square root of the variance. 

$$ \sigma = \sqrt{\dfrac{\sum (x - \mu)^2 }{N}} \text{ or } s = \sqrt{\dfrac{\sum (x - \overline{x})^2 }{n}} $$

\begin{center}
\begin{tabular}{rrcl}

where & $\sigma $ & = & the population standard deviation \\
& $s$ & = & the sample standard deviation \\

\end{tabular} 
\end{center} 



