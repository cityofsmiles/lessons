\textbf{Practice Exercises}

\vspce

Solve each of the following.
%\begin{center}
%\vspace*{2ex}
%\scalebox{1}{
%\noindent\begin{minipage}{\textwidth}
{\begin{enumerate}[label = \arabic*. ]
%\begin{multicols}{2}
%1
\item There are 5 questions on a True or False quiz. If all the questions are answered, in how many different ways can the quiz be completed?
%2
\item Each of the 5 questions on a multiple-choice quiz has 4 possible choices. If all the questions are answered, in how many ways can the quiz be completed?
%3
\item How many three-digit odd numbers can be formed from the digits 0, 1, 3, 5, 6 and 8 if each digit can be used only once?
%4
\item Using the digits 2, 4, 5, 8, and 9 and not repeating any of the digits: 
\begin{enumerate}[label = \arabic*. ]
%\setcounter{enumi}{0}
%\begin{multicols}{2}
%1
\item how many 3-digit numbers can be formed?
%2
\item how many 3-digit numbers less than 700 can be formed?
%3
\item how many 3-digit even numbers can be formed?
%4
\item how many 3-digit numbers are multiples of 5?
%5 
\item how many 3-digit numbers greater than 900 can be formed? 
%\end{multicols} 
\end{enumerate} 
%5
\item A tambiolo contains 15 balls numbered 1 to 15. A ball is drawn from the tambiolo. Find the number of ways each event can occur:
\begin{enumerate}[label = \alph*. ]
%1
\item The ball drawn is odd or even. 
%2
\item The ball is numbered greater than 7 or less than 5.
%3
\item The ball has a number which is less or equal to 10.
\end{enumerate} 
%7
\item A family of 2 boys and 3 girls are to sit in a row for a picture taking.
\begin{enumerate}[label = \alph*. ]
%a
\item Find the number of ways they can be seated.
%b
\item How many ways are there if the boys and the girls are each to sit together? 
\end{enumerate} 

%\end{multicols} 
\end{enumerate}}
%\end{minipage}}
%\end{center} 



