\noindent \begin{tabularx}{\textwidth}{LYR}

\multirow[c]{3}{15em}{
\begin{tikzpicture}[remember picture, overlay]
\node (shs) at (180:-2em) {\includegraphics[width=0.75in]{/host-rootfs/storage/emulated/0/Documents/documents/latex/1920/shs.png}};
\end{tikzpicture}
}
&
\textbf{SAUYO HIGH SCHOOL} 
&
\multirow[c]{3}{15em}{
\begin{tikzpicture}[remember picture, overlay]
\node (deped) at (0:10.5em) {\includegraphics[width=0.75in]{/host-rootfs/storage/emulated/0/Documents/documents/latex/1920/deped2.png}};
\end{tikzpicture}
}
\\

& MATHEMATICS DEPARTMENT & \\

& S.Y. 2019 -- 2020 & \\

& \textbf{Lesson Plan for Mathematics \Grade}&\\
\end{tabularx}

\vspace*{2em}

\noindent \begin{tabularx}{\textwidth}{LR}
Module title: \Module
&
Grade Level: Grade \Grade 
\\
Date: \Date
& 
Designed by: \Teacher 
\\
\end{tabularx}

\vspace*{2.5ex}

\begin{enumerate}[label = \textbf{\Roman*. }]
\item \textbf{Learning Competencies/Objectives }

\begin{enumerate}[label = \Alph*. ]
%1
\item Content Standard: \ContentStandard
%2
\item Performance Standard: \PerformanceStandard
%3
\item Learning Competency: \LearningCompetency
\end{enumerate}   

At the end of a 50-minute period, 80\% of the Grade \Grade students should be able to do the following with at least 75\% accuracy:
	\begin{enumerate}[label = \alph*. ]
	\Objectives 
	\end{enumerate}
	
\item \textbf{Subject Matter}
	\begin{enumerate}[label = \Alph*. ]
	\item Topic: \Topic
	\item Reference: \Reference
	\item Materials: \Materials
	\end{enumerate}

\item	 \textbf{Procedure}
	\begin{enumerate}[label = \Alph*. ]
	\item Daily routine
		\begin{enumerate}[label = \arabic*. ]
		\item Cleaning and arranging of chairs
		\item Greeting 
		\item Checking of assignment 
		\item Drill: Flashcards showing the operations on signed numbers 
		\item Review: \ReviewTopic
	%	\item Motivation: \Motivation
		\end{enumerate}
		
	\item Lesson Proper 
		\begin{enumerate}[label = \arabic*. ]
		\item Direct instruction: The teacher describes the main concepts of the lesson. \DirectInstruction
		\item Demonstration: The teacher shows how to solve the first item in the Practice Exercises. 
		\item Practice Exercises and Boardwork: \\%(See at the end.) 
		Answer the following problems in your notebook. \\
		{\PE}
			
		\item Generalization: 	Let the students answer the following questions. 
\begin{enumerate}[label = \arabic*. ]
\Generalization
\end{enumerate}

		\end{enumerate}

%\newpage

	\item Application: Problem Set\\ %(See at the end.) 
	In a sheet of paper, answer the following problems. \\
	{\PS}%\\

	\end{enumerate}

\end{enumerate}

%\PE 

%\vspace{1ex}
%\newpage

%\PS

\vfill

\begin{flushright}
Prepared by:\\*[2.5ex]
\Teacher \\
Teacher I
\end{flushright} 

\vspace*{3.5ex}

\begin{flushright}
Checked by:\\*[2.5ex]
\Checker
\end{flushright} 
	