 

\def \LessonDayA {Arithmetic Series}

\def \LearningCompetenciesDayA {}

\def \ObjectivesDayA {
\item Derive the formula in computing arithmetic series; 
\item Compute the number of terms of a given arithmetic series; and, 
\item Display enjoyment and independence in solving problems.
}

\def \TeachersGuideDayA {pp. 31--41}

\def \LMPagesDayA {pp. 19--25}

\def \TextbookPagesDayA {pp. 26--35}

\def \AdditionalMaterialsDayA {}

\def \OtherResourcesDayA {Flashcards}

\def \ReviewDayA {\begin{center}
\textbf{Arithmetic Series}
\end{center}

\vspce

\textbf{Arithmetic Series:} the indicated sum of the terms of an  arithmetic sequence 

\vspce

If the first term and the $n^{th}$ term are given, then 

$$\displaystyle S_{n}=\frac{n}{2}(a_{1}+a_{n})$$

\vspce

If the $n^{th}$ term is not  given, then 

$$\displaystyle S_{n}=\frac{n}{2}[2a_{1}+(n-1)d]$$

}

\def \PurposeDayA {The purpose of this lesson is to enable the students to solve real life problems involving arithmetic series.}  

\def \ExamplesDayA {Some examples of arithmetic series are the following: 
\begin{enumerate}[label = \arabic*. ]
%1
\item The sum of even integers from 5 to 11 is 24.
%2
\item $1 + 3 + 5 + 7 = 16$
%3
%\item 
\end{enumerate}   
}

\def \PracticeOneDayA {\textbf{Practice Exercises}

\vspce

A. Find the sum of each arithmetic sequence. 
\begin{enumerate}
\item \hspce 2, 5, 8,... to 8 terms
\item \hspce -11, -7, -3,... to 23 terms
\item \hspce Sum of odd integers from 1 to 100
\item \hspce Sum of the  integers between  50 and 200 which are divisible by 5
\end{enumerate}
}

\def \PracticeTwoDayA {B. In each arithmetic series, find the specified unknown. 
\begin{enumerate}
\item \hspce $S_{n}$ = 90, $a_{1}$=10, $a_{n}$=26, $n$=? 
\item \hspce $S_{n}$ = 1,800, $a_{n}$=185, $n$=18, $a_{1}$=? 
\item \hspce $S_{n}$ = 119, $a_{1}$=5, $d$=4, $n$=? 
\item \hspce $a_{10}$ = 27.5, $d$=3, $a_{1}$=?, $S_{n}$=?

\end{enumerate}

}

\def \MasteryDayA {\textbf{Problem Set}

\vspce

A. Find the sum of each arithmetic sequence. 
\begin{enumerate}
\item \hspce 3, 5, 7,... to 31 terms
\item \hspce 10, -2, -14,... to 17 terms
\item \hspce Sum of even integers from 10 to 90
\item \hspce Sum of the  integers between  2 and 100 which are divisible by 3
\end{enumerate}

\vspce

B. In each arithmetic series, find the specified unknown. 
\begin{enumerate}
\item \hspce $S_{n}$ = 50, $a_{1}$=4, $a_{n}$=16, $n$=? 
%\item \hspce $S_{n}$ = 195, $a_{n}$=33, $d$=3, $a_{1}$=? no solution! 
\item \hspce $S_{n}$ = -15, $a_{1}$=12, $d$=-3, $n$=? 
\item \hspce Sum of even integers between  20 and 80
\end{enumerate}
}

\def \ApplicationDayA { Let the students answer the following questions: 
\begin{enumerate}[label = \arabic*. ]
%1
\item In what real life situations or problems can we observe some examples of arithmetic series? 
%2
\item How can you apply your knowledge of arithmetic series in solving these real life problems? 
%3
%\item 
\end{enumerate}   
}

\def \GeneralizationDayA {Let the students answer the following questions: 
\begin{enumerate}[label = \arabic*. ]
%1
\item In your own words, what are arithmetic series? 
%2
\item How do we solve problems involving arithmetic series? 
%3
%\item 
\end{enumerate}   
}

\def \EvaluationDayA {
\begin{center}
\textbf{Quiz \#3}
\end{center} 

Math Time pp. 12
\begin{enumerate}[label = \Alph*. ]
\begin{multicols}{2}

%A
\item \phantom{a}
\begin{enumerate}[label = \arabic*. ]
%1
\item \#8
%2
\item \#11
%3
\item \#14
\item \#15
\end{enumerate}   


%B
\item \phantom{a}
\begin{enumerate}[label = \arabic*. ]
%1
\item \#1
%2
\item \#6
%3
\item \#7
\item \#10

\end{enumerate}   


\end{multicols} 
\end{enumerate}   
}

\def \RemediationDayA {}

\def \LessonDayB {Geometric Sequences}

\def \LearningCompetenciesDayB {}

\def \ObjectivesDayB {
\item Apply the steps in finding the rule of a given geometric sequence; 
\item Generate the next terms of a geometric sequence; and, 
\item Display determination and interest in solving problems.
}

\def \TeachersGuideDayB {pp. 42--50}

\def \LMPagesDayB {pp. 26--30}

\def \TextbookPagesDayB {pp. 36--42}

\def \AdditionalMaterialsDayB {}

\def \OtherResourcesDayB {Flashcards}

\def \ReviewDayB {\begin{center}
\textbf{Geometric Sequences}
\end{center}

\vspace*{1ex}

\textbf{Geometric Sequence:} a sequence in which each term after the first is obtained by multiplying the preceding term by a fixed nonzero constant 

\vspce

\textbf{Common Ratio ($r$):} the fixed constant 

\vspce

To  find  any  term  in  a  geometric sequence, use

%\centering \scalebox{1.5}{\color{black} 
$$a_{n} = a_{1}r^{n - 1}$$ 
%}


}

\def \PurposeDayB {The purpose of this lesson is to enable the students to solve real life problems involving geometric sequences.}  

\def \ExamplesDayB {Some examples of geometric sequences are the following: 
\begin{enumerate}[label = \arabic*. ]
%1
\item The sum of even integers from 5 to 11 is 24.
%2
\item $1 + 3 + 5 + 7 = 16$
%3
%\item 
\end{enumerate}   
}

\def \PracticeOneDayB {\textbf{Practice Exercises}

\vspce

\begin{enumerate}[label = \Alph*. ]
\item \hspce Find the common ratio and the next three terms of each geometric sequence.   
	\begin{enumerate}[label = \arabic*. ]
	

	\item \hspce $2, 6, 18, 54,\ldots$
	\item \hspce $\displaystyle \frac{1}{8}, \frac{1}{4}, \frac{1}{2},\ldots $
	\item \hspce $4, 12, 36,\ldots$
	\item \hspce $0.02, 0.2, 2,\ldots$
	\item \hspce $3x^{3}, 6x^{5}, 12x^{7},\ldots $
	 
	\end{enumerate}


\item \hspce Find the specified term of each geometric sequence. 
	\begin{enumerate}[label = \arabic*. ]
	\item \hspce $3, 6, 12,\ldots \hspce  a_{7}$
	\item \hspce $4, 20, 100,\ldots \hspce  a_{8}$
	\item \hspce $7, -7, 7,\ldots \hspce  a_{17}$
	\item \hspce $3, 1.2, 0.48,\ldots \hspce  a_{10}$
	\item \hspce $\displaystyle 1, \frac{3}{2}, \frac{9}{4},\ldots \hspce  a_{11}$
	\end{enumerate}

\end{enumerate}}

\def \PracticeTwoDayB {Solve each problem completely. 
	\begin{enumerate}[label = \arabic*. ]
	\item \hspce The first term of a geometric sequence is 8, and the second term is 4. Find the fifth term. 

	\item \hspce The first term of a geometric sequence is 3, and the third term is $\frac{4}{3}$. Find the fifth term. 

	\item \hspce The common ratio in a geometric sequence is $\frac{2}{5}$ and the fourth term is $\frac{5}{2}$. Find the third term. 

	\item \hspce Which term of the geometric sequence 2, 6, 18,... is 118098?
	\item \hspce The second and fifth terms of a geometric sequence are 10 and 1250, respectively.  Is 31,250 a term of this sequence? If so, which term is it?
	\end{enumerate}
}

\def \MasteryDayB {\textbf{Problem Set}

\vspce

\begin{enumerate}[label = \Alph*. ]
\item \hspce Find the common ratio and the next three terms of each geometric sequence.   
	\begin{enumerate}[label = \arabic*. ]
	
	\item \hspce $4, 8, 16, 32,\ldots$
	\item \hspce $\displaystyle \frac{4}{9}, \frac{4}{3}, 4,\ldots $
	\item \hspce $1, -5, 25,\ldots$
	\item \hspce $-5, -0.5, -0.05,\ldots$
	\item \hspce $x, 5x^{2}y, 25x^{3}y^{2},\ldots $
	 
	\end{enumerate}

\item \hspce Find the specified term of each geometric sequence. 
	\begin{enumerate}[label = \arabic*. ]
	\item \hspce $64, -32, 16,\ldots \hspce  a_{7}$
	\item \hspce $2, -10, 50,\ldots \hspce  a_{8}$
	\item \hspce $2, -6, 18,\ldots \hspce  a_{13}$
	\item \hspce $3, 1.2, 0.48,\ldots \hspce  a_{10}$
	\item \hspce $\displaystyle  \frac{1}{16}, \frac{1}{8}, \frac{1}{4},\ldots \hspce  a_{9}$
	\end{enumerate}


\item \hspce Solve each problem completely. 
	\begin{enumerate}[label = \arabic*. ]
	\item \hspce The first term of a geometric sequence is -2, and the third term is $-\frac{1}{2}$. Find the fifth term. 

	\item \hspce The common ratio in a geometric sequence is $\frac{2}{3}$ and the fourth term is 1. Find the third term. 

	\item \hspce The common ratio in a geometric sequence is $\frac{3}{4}$ and the fifth term is $\frac{81}{16}$. Find the first three terms.

	\item \hspce Which term of the geometric sequence 3, 6, 12,... is 768?
	\item \hspce The common ratio in a geometric sequence is $\frac{3}{2}$ and the fifth term is 1. Find the first three terms.

	\end{enumerate}
\end{enumerate}}

\def \ApplicationDayB { Let the students answer the following questions: 
\begin{enumerate}[label = \arabic*. ]
%1
\item In what real life situations or problems can we observe some examples of geometric sequences? 
%2
\item How can you apply your knowledge of geometric sequences in solving these real life problems? 
%3
%\item 
\end{enumerate}   
}

\def \GeneralizationDayB {Let the students answer the following questions: 
\begin{enumerate}[label = \arabic*. ]
%1
\item In your own words, what are geometric sequences? 
%2
\item How do we solve problems involving geometric sequences? 
%3
%\item 
\end{enumerate}   
}

\def \EvaluationDayB {
\input{/host-rootfs/storage/emulated/0/Documents/documents/latex/1920/Grade-10/1st/geometric-sequences/qz-geometric-sequences-input} 
}

\def \RemediationDayB {}

\def \LessonDayC {Finite Geometric Series}

\def \LearningCompetenciesDayC {}

\def \ObjectivesDayC {
\item Describe  the steps in finding the sum of the terms of a finite geometric sequence; 
\item Compute  the sum of the terms of a finite geometric sequence; and, 
\item Display independence  and willingness  in solving problems.
}

\def \TeachersGuideDayC {pp. 51--62}

\def \LMPagesDayC {pp. 31--37}

\def \TextbookPagesDayC {pp. 43--52}

\def \AdditionalMaterialsDayC {}

\def \OtherResourcesDayC {Flashcards}

\def \ReviewDayC {\begin{center}
\textbf{Finite Geometric Series}
\end{center}

\vspace*{1ex}

\textbf{Finite Geometric Series:} the indicated sum of the terms of a geometric sequence 

\vspce

If the last term $a_n$ is not given, use

%\begin{center}
$$\displaystyle S_n = \frac{a_1(1-r^n)}{1-r},\hspce r\neq1$$
%\end{center}

\vspce

If the last term $a_n$ is given, use

%\begin{center}
$$\displaystyle S_n = \frac{a_1-a_{n}r}{1-r}, \hspce r\neq1$$
%\end{center}

}

\def \PurposeDayC {The purpose of this lesson is to enable the students to solve real life problems involving finite geometric series.}  

\def \ExamplesDayC {Some examples of finite geometric series are the following: 
\begin{enumerate}[label = \arabic*. ]
%1
\item The sum of even integers from 5 to 11 is 24.
%2
\item $1 + 3 + 5 + 7 = 16$
%3
%\item 
\end{enumerate}   
}

\def \PracticeOneDayC {\textbf{Practice Exercises}

\vspce

A. Find the indicated sum of the following geometric series.   

\begin{enumerate}[label = \arabic*. ]

%begin{multicols}{2}


	\item \hspce $1+4+16+\ldots S_6$
	\item \hspce $2+4+8+16+\ldots S_{10}$
	\item \hspce $2+6+18+\ldots S_7$
	\item \hspce $(-9)+6+(-4)+\ldots S_8$
	\item \hspce $2+2\sqrt{2}+4+\ldots S_{10}$
	
%end{multicols} 
\end{enumerate}}

\def \PracticeTwoDayC {B. Find each specified term.  

\begin{enumerate}[label = \arabic*. ]
%

\item \hspce $S_5 = \displaystyle \frac{31}{4}; \hspce r=\frac{1}{2}; \hspce a_1=? $
\item \hspce $\displaystyle S_8 = 2,550; \hspce r=2;\hspce a_1=? $
\item \hspce $\displaystyle S_7 = 7,651;\hspce  r=3;\hspce a_1=? $
\item \hspce $\displaystyle S_{10} = 51,150; \hspce r=2;\hspce a_1=? $
\item \hspce $S_6 = 126;\hspce \displaystyle r=-\frac{1}{2};\hspce a_6=? $

%
\end{enumerate}
}

\def \MasteryDayC {\textbf{Problem Set }

\vspce

\begin{enumerate}[label = \Alph*. ]
\item Find the indicated sum of the following geometric series.   
  
\begin{enumerate}[label = \arabic*. ]
%begin{multicols}{2}


	\item \hspce $9+6+4+\ldots S_7$ 
	\item \hspce $2+8+32+\ldots S_{9}$
	\item \hspce $3+3\sqrt{3}+9+\ldots S_{9}$
	\item \hspce $1+(-2)+4+(-8)\ldots S_8$
	\item \hspce $(-2)+6+(-18)+\ldots S_6$ 
	
%end{multicols} 

\end{enumerate}

\vspce

\item Find the sum of the first $n$ terms of the related geometric series. 

\begin{enumerate}[label = \arabic*. ]

%begin{multicols}{2}

\item \hspce $\displaystyle a_1=\frac{1}{2}; r=4; n=6 $
\item \hspce $\displaystyle a_1=13; r=4; n=7 $ 
\item \hspce $\displaystyle a_1=318; \hspce r=\frac{1}{2};\hspce n=7 $
\item \hspce $\displaystyle a_1=168;\hspce r=\frac{3}{4};\hspce n=8 $
\item \hspce $\displaystyle a_1=4;\hspce r=-5; \hspce n=8 $


%end{multicols} 
\end{enumerate}
\end{enumerate}}

\def \ApplicationDayC { Let the students answer the following questions: 
\begin{enumerate}[label = \arabic*. ]
%1
\item In what real life situations or problems can we observe some examples of finite geometric series? 
%2
\item How can you apply your knowledge of finite geometric series in solving these real life problems? 
%3
%\item 
\end{enumerate}   
}

\def \GeneralizationDayC {Let the students answer the following questions: 
\begin{enumerate}[label = \arabic*. ]
%1
\item In your own words, what are finite geometric series? 
%2
\item How do we solve problems involving finite geometric series? 
%3
%\item 
\end{enumerate}   
}

\def \EvaluationDayC {
\input{/host-rootfs/storage/emulated/0/Documents/documents/latex/1920/Grade-10/1st/finite-geometric-series/qz-finite-geometric-series-input} 
}

\def \RemediationDayC {}

\def \LessonDayD {Infinite Geometric Series}

\def \LearningCompetenciesDayD {}

\def \ObjectivesDayD {
\item Tell the steps in finding the sum of the terms of an infinite geometric sequence; 
\item Generate the sum of the terms of an infinite geometric sequence; and, 
\item Display willingness and interest in solving problems.
}

\def \TeachersGuideDayD {pp. 63--71}

\def \LMPagesDayD {pp. 38--42}

\def \TextbookPagesDayD {pp. 53--59}

\def \AdditionalMaterialsDayD {}

\def \OtherResourcesDayD {Flashcards}

\def \ReviewDayD {\begin{center}
\textbf{Infinite Geometric Series}
\end{center}

\vspace*{1ex}

\textbf{Infinite Geometric Series:} a series of the form $a+ar+ar^{2}+ar^{3}+... +ar^{n-1}+... $

\vspce


To find the sum of an infinite geometric sequence, use

%\begin{center}
$$\displaystyle S = \frac{a}{1-r}, \hspce -1<r<1$$
%\end{center}


}

\def \PurposeDayD {The purpose of this lesson is to enable the students to solve real life problems involving infinite geometric series.}  

\def \ExamplesDayD {Some examples of infinite geometric series are the following: 
\begin{enumerate}[label = \arabic*. ]
%1
\item The sum of even integers from 5 to 11 is 24.
%2
\item $1 + 3 + 5 + 7 = 16$
%3
%\item 
\end{enumerate}   
}

\def \PracticeOneDayD {\textbf{Practice Exercises}

\vspce

A. Determine if each geometric series has a sum. If the sum exists, find the sum.   

\begin{enumerate}[label = \arabic*. ]


	\item \hspce $\displaystyle 4+1\frac{1}{4}+... $
	\item \hspce $4+2+1+...$
		\item \hspce$16+8+4+2+...$  
	\item \hspce $\displaystyle 1+\frac{1}{3}+\frac{1}{9}+\frac{1}{27}+...$

	\item \hspce $16+1.6+0.16+... $
	

\end{enumerate}
}

\def \PracticeTwoDayD {B. In each infinite geometric series, find the specified unknown.   

\begin{enumerate}[label = \arabic*. ]


\item \hspce $\displaystyle S=45; \hspce a_1=15;\hspce r=?$
\item \hspce $\displaystyle S=28; \hspce r=\frac{1}{7};\hspce a_1=? $

\item \hspce $\displaystyle S=80; \hspce r=\frac{1}{5};\hspce a_1=? $
\item \hspce $\displaystyle S=\frac{1}{3}; \hspce a_1=\frac{3}{10};\hspce r=? $
\item \hspce $\displaystyle S=\frac{4\sqrt{3}}{3};\hspce  r=\frac{1}{4};\hspce a_1=? $


\end{enumerate}
}

\def \MasteryDayD {\textbf{Problem Set }

\vspce

\begin{enumerate}[label = \Alph*. ]
\item Determine if each geometric series has a sum. If the sum exists, find the sum.     
  
\begin{enumerate}[label = \arabic*. ]
%

	\item \hspce $\displaystyle 1+(-\frac{1}{2})+\frac{1}{4}+(-\frac{1}{8})\ldots$

	\item \hspce $4+2.4+1.44+\ldots$
	\item \hspce $\displaystyle 6+2+\frac{2}{3}+\frac{2}{9}+\ldots$
		\item \hspce $(-5)+(-0.5)+(-0.05)+\ldots$
	\item \hspce $\displaystyle 1+(-\frac{1}{3}) +\frac{1}{9}+(-\frac{1}{27}) + \ldots$
%	
	
\end{enumerate}

\vspce

\item In each infinite geometric series, find the specified unknown.    

\begin{enumerate}[label = \arabic*. ]
%

\item \hspce $\displaystyle S=-10; \hspce a_1=-5;\hspce r=?$
\item \hspce $\displaystyle S=-52; \hspce a_1=-65;\hspce r=?$
\item \hspce $\displaystyle S=-\frac{2}{5}; \hspce a_1=-\frac{1}{4};\hspce r=? $
\item \hspce $\displaystyle S=-36; \hspce a_1=-60;\hspce r=?$  
\item \hspce $\displaystyle S=384;\hspce  r=\frac{1}{3};\hspce a_1=? $

%	
\end{enumerate}
\end{enumerate}




}

\def \ApplicationDayD { Let the students answer the following questions: 
\begin{enumerate}[label = \arabic*. ]
%1
\item In what real life situations or problems can we observe some examples of infinite geometric series? 
%2
\item How can you apply your knowledge of infinite geometric series in solving these real life problems? 
%3
%\item 
\end{enumerate}   
}

\def \GeneralizationDayD {Let the students answer the following questions: 
\begin{enumerate}[label = \arabic*. ]
%1
\item In your own words, what are infinite geometric series? 
%2
\item How do we solve problems involving infinite geometric series? 
%3
%\item 
\end{enumerate}   
}

\def \EvaluationDayD {
\input{/host-rootfs/storage/emulated/0/Documents/documents/latex/1920/Grade-10/1st/infinite-geometric-series/qz-infinite-geometric-series-input} 
}

\def \RemediationDayD {}

\def \LessonDayE {Geometric Means}

\def \LearningCompetenciesDayE {}

\def \ObjectivesDayE {
\item Describe the steps in finding the geometric mean; 
\item Compute the geometric mean  of a given geometric sequence; and, 
\item Show enjoyment and perseverance in solving problems.
}

\def \PurposeDayE {The purpose of this lesson is to enable the students to solve real life problems involving geometric means.}  

\def \ApplicationDayE { Let the students answer the following questions: 
\begin{enumerate}[label = \arabic*. ]
%1
\item In what real life situations or problems can we observe some examples of geometric means? 
%2
\item How can you apply your knowledge of geometric means in solving these real life problems? 

\end{enumerate}   
}

\def \GeneralizationDayE {Let the students answer the following questions: 
\begin{enumerate}[label = \arabic*. ]
%1
\item In your own words, what are geometric means? 
%2
\item How do we solve problems involving geometric means? 

\end{enumerate}   
}

\def \TeachersGuideDayE {pp. 72--82}

\def \LMPagesDayE {pp. 43--49}

\def \TextbookPagesDayE {pp. 60--69}

\def \AdditionalMaterialsDayE {}

\def \OtherResourcesDayE {Flashcards}

\def \ReviewDayE {\begin{center}
\textbf{Geometric Means}
\end{center}

\vspace*{1ex}


\textbf{Geometric Extremes:} the first and last terms of a geometric sequence 
\vspce

\textbf{Geometric Means:} the terms between the geometric extremes 
\vspce

\textbf{Mean Proportionality:} the geometric mean between two terms 
\vspce

To solve for the mean proportionality of two terms $a$ and $b$, use

%\begin{center}
$$\displaystyle GM = \pm\sqrt{ab}$$%\end{center}}



\def \ExamplesDayE {}

\def \PracticeOneDayE {\textbf{Practice Exercises}

\vspce

A. Insert the specified number of geometric means.   

\begin{enumerate}[label = \arabic*. ]


	\item \hspce Two: 3 and 81
	\item \hspce Two: 16 and $-2$ 
	\item \hspce Two: 2 and $-250$ 
	\item \hspce Two: $-3$ and 24
	\item \hspce One negative: 2 and 50 
	

\end{enumerate}}

\def \PracticeTwoDayE {B. Find the missing terms of each geometric sequence.  

\begin{enumerate}[label = \arabic*. ]
%

\item \hspce 3, \blank, 27
\item \hspce \blank, 24, \blank, \blank, 3, \blank
\item \hspce $x$, \blank, $x^{2}$
\item \hspce 81, \blank, \blank, \blank, \blank, $\displaystyle \frac{1}{3}$
\item \hspce \blank, \blank, $x^{4}$, $2x^{7}$, \blank, \blank

%
\end{enumerate}
}

\def \MasteryDayE {\textbf{Problem Set}

\vspce

A. Insert the specified number of geometric means.  
\begin{enumerate}[label = \arabic*. ]



	\item \hspce Two: 128 and 16
	\item \hspce Three: $-2$ and $-512$ 
	\item \hspce Two: 4 and 32
	\item \hspce Three: 4 and 324
	\item \hspce One positive: $-4$ and $-36 $
	
\end{enumerate}

\vspce

B. Find the missing terms of each geometric sequence.  

\begin{enumerate}[label = \arabic*. ]
%

\item \hspce 2, \blank, \blank, 54
\item \hspce \blank, \blank, \blank, 8, 16
\item \hspce $x$, \blank, \blank, $y$
\item \hspce \blank, \blank, 9, \blank, 1
\item \hspce \blank, $\displaystyle \frac{1}{3}$, 1, \blank, \blank
	
\end{enumerate}
  }



\def \EvaluationDayE {
\input{/host-rootfs/storage/emulated/0/Documents/documents/latex/1920/Grade-10/1st/geometric-means/qz-geometric-means-input} 
}

\def \RemediationDayE {}
