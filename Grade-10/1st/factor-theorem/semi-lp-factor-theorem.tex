% cd /storage/emulated/0/Documents/documents/latex/1920/Grade-10/1st/factor-theorem && pdflatex semi-lp-factor-theorem.tex && termux-open semi-lp-factor-theorem.pdf


\documentclass[12pt]{article}
\usepackage[legalpaper, portrait, right=0.5in,  left=0.5in, top=0.5in, bottom=1.5in]{geometry}
\usepackage{longtable}
\usepackage{tabularx}
\usepackage{enumitem}
\usepackage{etoolbox}
\usepackage{multicol, multirow} 
\usepackage{tikz}
\usetikzlibrary{angles,quotes}
\usepackage{pgfplots} 
\usetikzlibrary{calc}
\pgfplotsset{compat=newest}
\usetikzlibrary{arrows.meta}
\usetikzlibrary{intersections}
\usetikzlibrary{decorations.pathreplacing}


\newcolumntype{R}{>{\raggedleft\arraybackslash}X}

\newcolumntype{L}{>{\raggedright\arraybackslash}X}
 
\pagenumbering{gobble}

\newcommand{\vspce}{\vspace{0pt}}

\newcommand{\hspce}{\hspace{0.5em}}

\newcommand{\blank}{\rule{1em}{0.75pt}}

\def \Grade {10
}

\def \Teacher {
Mr. Jonathan R. Bacolod, LPT 
}

\def \Module {
Polynomial  Equations
}

\def \Date {
August 7, 2019
}

\def \ContentStandard {
The learner demonstrates understanding of key concepts of sequences, polynomials and polynomial equations.
}

\def \PerformanceStandard {
The learner is able to formulate and solve problems involving sequences, polynomials and polynomial equations in different 	 
disciplines through appropriate and accurate representations.
}

\def \LearningCompetency {
The learner proves the Remainder Theorem and the Factor Theorem. (M10AL-Ig-2)
}

\def \Objectives {
\item Describe the factor theorem; 
\item Determine whether a binomial is a factor of a polynomial using the factor theorem; and, 
\item Show interest and perseverance in solving problems. 
}

\def \Generalization {
\item[a. ] In your own words, what is the factor theorem?    
\item[b. ] How do we solve problems involving factor theorem? 
}


\def \Topic {
Factor Theorem 
}

\def \ReviewTopic {
Remainder Theorem \\
Use the remainder theorem  to find the remainder of the  polynomial function: \\
$f(x) = 4x^3+2x+10$ at $x = -3$  
}

\def \Motivation { A basket contains 5 apples. How do you distribute the 5 apples to 5 children while leaving 1 apple in the basket? 
}

\def \Reference {
Mathematics \Grade Learner’s Module pp. 93-94
}

\def \Materials {
Tarpapel showing the steps and the formula for\Topic
}

\def \Checker {
\underline{\textbf{DR. LORETO R. DOMINGO}}\\
OIC/MT II Mathematics Department
}

\def \DirectInstruction {
\begin{center}
\textbf{Factor Theorem 
}
\end{center}

\vspce 

\textbf{Factor Theorem:} If  $P(x) $ is a polynomial and $P(c)=0$, then $x-c$ is a factor of $P(x) $. Conversely, if $x-c$ is a factor of $P(x) $, then $P(c)=0$. 


}

\def \PE {
\textbf{Practice Exercises}
%\textbf{Problem Set}

\vspce
%\begin{enumerate}[label = \Alph*. ]

%\item 
Use the factor theorem to determine whether the binomial is a factor of the given polynomial. 

%\vspce

\begin{enumerate}[label = \arabic*. ]
%%begin{multicols}{2}

%#1
\item \hspce $(x+3); P(x) = 2x^3+11x^2+16x+6$
\vspce
%#2
\item \hspce $(x+1); P(x) = 2x^3+5x^2+4x+1$
\vspce
%#3
\item \hspce $(x-2); P(x) = 4x^3-11x^2+8x-4$
\vspce
%#4
\item \hspce $(x+3); P(x) = x^4+3x^3-2x^2-5x+3$
\vspce
%#5
\item \hspce $(2x-1); P(x) = 2x^3-7x^2+x+1$


%%end{multicols}
\end{enumerate}



%\end{enumerate} 
 



%\end{enumerate} 
}

\def \PS {
%\textbf{Practice Exercises}
\textbf{Problem Set}

\vspce
%\begin{enumerate}[label = \Alph*. ]

%\item 
Use the factor theorem to determine whether the binomial is a factor of the given polynomial. 

%\vspce

\begin{enumerate}[label = \arabic*. ]
%%begin{multicols}{2}

%#1
\item \hspce $(x-2); P(x) = x^{20}-4x^{18}+3x-6$
\vspce
%#2
\item \hspce $(x-4); P(x) = 3x^3-15x^2+10x+8$
\vspce
%#3
\item \hspce $(x+2); P(x) = x^4- 3x^3+5x-2$
\vspce
%#4
\item \hspce $(x-2); P(x) = 3x^4-6x^3+5x+10$
\vspce
%#5
\item \hspce $(x+5); P(x) = x^3+x^2-25x+25$


%%end{multicols}
\end{enumerate}



%\end{enumerate} 
 



%\end{enumerate}
}


\begin{document}

\noindent \begin{tabularx}{\textwidth}{LcR}

\multirow[c]{3}{15em}{
\begin{tikzpicture}[remember picture, overlay]
\node (shs) at (180:-2em) {\includegraphics[width=0.75in]{/storage/emulated/0/Documents/documents/latex/1920/shs.png}};
\end{tikzpicture}
}
&
\textbf{SAUYO HIGH SCHOOL} 
&
\multirow[c]{3}{15em}{
\begin{tikzpicture}[remember picture, overlay]
\node (deped) at (0:10.5em) {\includegraphics[width=0.75in]{/storage/emulated/0/Documents/documents/latex/1920/deped2.png}};
\end{tikzpicture}
}
\\

& MATHEMATICS DEPARTMENT & \\

& S.Y. 2019 -- 2020 & \\

& \textbf{Lesson Plan for Mathematics \Grade}&\\
\end{tabularx}

\vspace*{2em}

\noindent \begin{tabularx}{\textwidth}{LR}
Module title: \Module
&
Grade Level: Grade \Grade 
\\
Date: \Date
& 
Designed by: \Teacher 
\\
\end{tabularx}

\vspace*{2.5ex}

\begin{enumerate}[label = \textbf{\Roman*. }]
\item \textbf{Learning Competencies/Objectives }

\begin{enumerate}[label = \Alph*. ]
%1
\item Content Standard: \ContentStandard
%2
\item Performance Standard: \PerformanceStandard
%3
\item Learning Competency: \LearningCompetency
\end{enumerate}   

At the end of a 50-minute period, 80\% of the Grade \Grade students should be able to do the following with at least 75\% accuracy:
	\begin{enumerate}[label = \alph*. ]
	\Objectives 
	\end{enumerate}
	
\item \textbf{Subject Matter}
	\begin{enumerate}[label = \Alph*. ]
	\item Topic: \Topic
	\item Reference: \Reference
	\item Materials: \Materials
	\end{enumerate}

\item	 \textbf{Procedure}
	\begin{enumerate}[label = \Alph*. ]
	\item Daily routine
		\begin{enumerate}[label = \arabic*. ]
		\item Cleaning and arranging of chairs
		\item Greeting 
		\item Checking of assignment 
		\item Drill: Flashcards showing the operations on signed numbers 
		\item Review: \ReviewTopic
		\item Motivation: \Motivation
		\end{enumerate}
		
	\item Lesson Proper 
		\begin{enumerate}[label = \arabic*. ]
		\item Direct instruction: The teacher describes the main concepts of the lesson. \DirectInstruction
		\item Demonstration: The teacher shows how to solve the first item in the Practice Exercises. 
		\item Practice Exercises and Boardwork: (See at the end.) 	
		\item Generalization: 	Let the students answer the following questions. 
\begin{enumerate}[label = \arabic*. ]
\Generalization
\end{enumerate}

		\end{enumerate}

%\newpage

	\item Application: Problem Set (See at the end.) \\

	\end{enumerate}

\end{enumerate}



\PE 

%\vspace{1ex}
%\newpage

\PS

\vfill

\begin{flushright}
Prepared by:\\*[2.5ex]
\Teacher \\
Teacher I
\end{flushright} 

\vspace*{3.5ex}

\begin{flushright}
Checked by:\\*[2.5ex]
\Checker
\end{flushright} 

\end{document}