 

\def \LessonDayA {\makecell{Arithmetic Sequences}}

\def \LearningCompetenciesDayA {}

\def \ObjectivesDayA {
\item Execute the steps in finding the rule of a given arithmetic sequence; 
\item Find the next terms of a given arithmetic sequence; and, 
\item Project interest and enjoyment in solving problems.
}

\def \TeachersGuideDayA {pp. 11--21}

\def \LMPagesDayA {pp. 7--13}

\def \TextbookPagesDayA {pp. 9--18}

\def \AdditionalMaterialsDayA {}

\def \OtherResourcesDayA {Flashcards}

\def \ReviewDayA {\begin{center}
\textbf{Arithmetic Sequences}
\end{center}

\vspace*{1ex}

\textbf{Arithmetic Sequence:} a sequence where every term after the first is obtained by adding a constant called the common difference

\vspce

\textbf{Common difference:} the constant difference $d$ between any two consecutive terms

\vspce

To  find  any  term  in  an  arithmetic sequence, use

$$a_{n} = a_{1} + (n - 1)d$$

To  find  $d$, use

$$d = \displaystyle \frac{a_{n} - a_{k}}{n - k} $$

}

\def \PurposeDayA {The purpose of this lesson is to enable the students to solve real life problems involving arithmetic sequences.}  

\def \ExamplesDayA {Some examples of arithmetic sequences are the following: 
\begin{enumerate}[label = \arabic*. ]
%1
\item \hspce $14, 6, -2,... a_{28}$
%2
\item \hspce  $3, 5, 7,... a_{21}$
\item \hspce $1.4, 4.5, 7.6,... a_{51}$
\end{enumerate}   
}

\def \PracticeOneDayA {\textbf{Practice Exercises}
%\textbf{Problem Set}

\vspce

A. Find the specified term of each arithmetic sequence.   
\begin{enumerate}
\begin{multicols}{2}

\item \hspce $2, 5, 8,\ldots \hspce  a_{8}$
\item \hspce $-11, -7, -3,\ldots \hspce   a_{23}$
\item \hspce $10, -2, -14,\ldots \hspce   a_{17}$
\item \hspce $y, x, 2x-y,\ldots \hspce   a_{10}$
\item \hspce $3, 3.25, 3.5,\ldots \hspce   a_{16}$
\end{multicols} 
\end{enumerate}}

\def \PracticeTwoDayA {B. Find the specified term.   
\begin{enumerate}
\item \hspce $18^{th}$ term of the arithmetic sequence if $a_{1}$ = 25 and $d$~=~-2.
\item \hspce $11^{th}$ term if $a_{1}$ = -15 and $d$~=~6.
\item \hspce In the sequence 2, 6, 10,..., what term has a value of 106?
\item \hspce In the sequence 7, 4, 1,...,  what term has a value of -296?
\item \hspce If $a_{11}$ = 110 and $a_{28}$ = 280, find $a_{21}$. 
\item \hspce If $a_{10}$ = 14 and $a_{37}$ = 122, find $a_{25}$. 
\end{enumerate}
}

\def \MasteryDayA {%\textbf{Practice Exercises}
\textbf{Problem Set}

\vspce

A. Find the specified term of each arithmetic sequence.    
\begin{enumerate}
\item \hspce  $3, 5, 7,\ldots \hspce   a_{21}$
\item \hspce $1.4, 4.5, 7.6,\ldots \hspce   a_{51}$
\item \hspce $x-2, 4x, 7x+2,\ldots \hspce   a_{12}$
\item \hspce $14, 6, -2,\ldots \hspce   a_{28}$
\item \hspce $5, -1, -7,\ldots \hspce   a_{18}$
\end{enumerate}

B. Find the specified term.   
\begin{enumerate}

\item  \hspce $17^{th}$ term of the sequence if  $a_{8}$ = 5 and $a_{21}$ = -60.
\item \hspce $5^{th}$ term of the sequence if  $a_{15}$ = 29 and $a_{27}$ = 47.
\item \hspce If $a_{24}$ = 85 and $a_{28}$ = 100, $a_{1}$=?
\item \hspce If $a_{1}$ = -4 and $a_{25}$ = -100, $a_{101}$=?
\end{enumerate}}

\def \ApplicationDayA { Let the students answer the following questions: 
\begin{enumerate}[label = \arabic*. ]
%1
\item In what real life situations or problems can we observe some examples of arithmetic sequences? 
%2
\item How can you apply your knowledge of arithmetic sequences in solving these real life problems? 
%3
%\item 
\end{enumerate}   
}

\def \GeneralizationDayA {Let the students answer the following questions: 
\begin{enumerate}[label = \arabic*. ]
%1
\item In your own words, what are arithmetic sequences? 
%2
\item How do we solve problems involving arithmetic sequences? 
%3
%\item 
\end{enumerate}   
}

\def \EvaluationDayA {
\begin{center}
\textbf{Quiz \#1}
\end{center} 

\vspce 

Find the specified term.   
\begin{enumerate}
\item The $101^{th}$ term of the arithmetic sequence if $a_{1}$=-5 and $d$=-4.
\item The $39^{th}$ term of the arithmetic sequence if $a_{1}$ = 40 and $d$ = $\frac{1}{2}$.
\item  In the sequence 6, 10, 14,..., what term has a value of 286?
%\item  In the sequence 3, $\frac{7}{3}$, $\frac{5}{3}$,..., what term has a value of -27?
\item The $1^{st}$ term of the sequence if  $a_{5}$ = 26 and $a_{12}$ = 47.
\item The $61^{th}$ term of the sequence if  $a_{12}$ = 8 and $a_{21}$ = 26.
%\item  If $a_{3}$=8 and $a_{16}$=47, $a_{71}$=?
%\item If $a_{21}$=64 and $a_{100}$=301, $a_{11}$=?
\end{enumerate} 
}

\def \RemediationDayA {}

\def \LessonDayB {\makecell{Arithmetic Means}}

\def \LearningCompetenciesDayB {}

\def \ObjectivesDayB {
\item State the steps in finding the arithmetic mean; 
\item Generate the arithmetic mean given  two terms of a sequence; and, 
\item Display interest and willingness in solving problems.
}

\def \TeachersGuideDayB {pp. 22--30}

\def \LMPagesDayB {pp. 14--18}

\def \TextbookPagesDayB {pp. 19--25}

\def \AdditionalMaterialsDayB {}

\def \OtherResourcesDayB {Flashcards}

\def \ReviewDayB {\begin{center}
\textbf{Arithmetic Means}
\end{center}

\vspce


\textbf{Arithmetic Extremes:} the first and last terms of a finite arithmetic sequence 

\vspce

\textbf{Arithmetic Means:} the terms between the arithmetic extremes 

\vspce

\textbf{Average:} the arithmetic mean between two numbers  

\vspce 

To insert more than one arithmetic mean, use the difference  formula $d$. 

\begin{center}
%\scalebox{1.5}{\color{black}
$$\normalsize d=\frac{a_{n}-a_{k}}{n-k}$$
%}
\end{center}}

\def \PurposeDayB {The purpose of this lesson is to enable the students to solve real life problems involving arithmetic means.}  

\def \ExamplesDayB {Some examples of arithmetic means are the following: 
\begin{enumerate}[label = \arabic*. ]
%1
\item The arithmetic mean of 7 and 11 is 9.
%2
\item The arithmetic mean of 21 and 35 is 28.
%3
\item The two arithmetic means between 7 and 13 are 9 and 11.
\end{enumerate}   
}

\def \PracticeOneDayB {\textbf{Practice Exercises}
%\textbf{Problem Set}

\vspce

Insert the indicated number of arithmetic means between the given arithmetic extremes.  
\begin{enumerate}
%\begin{multicols}{2}
\item \hspce 2 and  32  \hspce [1]
\item \hspce -12 and  6  \hspce [3]
\item \hspce 68 and  3 \hspce  [4]
\item \hspce 15x and  23x  \hspce [1] 
\item \hspce $9\sqrt{3}$ and $11\sqrt{3}$ \hspce [1] 
\item \hspce $2\sqrt{5}$ and $14\sqrt{5}$ \hspce [2] 
\item \hspce $\displaystyle \frac{3}{7}$ and $ \displaystyle \frac{11}{7}$ \hspce [1]
%\end{multicols}
\end{enumerate}}

\def \PracticeTwoDayB {\input{/host-rootfs/storage/emulated/0/Documents/documents/latex/1920/Grade-10/1st/arithmetic-means/ps-arithmetic-means-input2}
}

\def \MasteryDayB {\textbf{Problem Set}

Insert the indicated number of arithmetic means between the given arithmetic extremes.   
\begin{enumerate}
\item \hspce  -5 and  1 \hspce  [2]
\item \hspce 24 and  -12 \hspce  [3]
\item \hspce 8 and  23  \hspce [4]
\item \hspce 4x and  -16x  \hspce [4] 
\item \hspce $6\sqrt{5}$ and $12\sqrt{5}$ \hspce [1] 
\item \hspce $-3\sqrt{3}$ and $15\sqrt{3}$ \hspce [5] 
\item \hspce $\displaystyle \frac{1}{2}$ and 2 \hspce [2]
\end{enumerate}}

\def \ApplicationDayB { Let the students answer the following questions: 
\begin{enumerate}[label = \arabic*. ]
%1
\item In what real life situations or problems can we observe some examples of arithmetic means? 
%2
\item How can you apply your knowledge of arithmetic means in solving these real life problems? 
%3
%\item 
\end{enumerate}   
}

\def \GeneralizationDayB {Let the students answer the following questions: 
\begin{enumerate}[label = \arabic*. ]
%1
\item In your own words, what are arithmetic means? 
%2
\item How do we solve problems involving arithmetic means? 
%3
%\item 
\end{enumerate}   
}

\def \EvaluationDayB {
\begin{center}
\textbf{Quiz \#2}
\end{center} 

Math Time pp. 9
\begin{enumerate}[label = \Alph*. ]
\begin{multicols}{2}

%A
\item \phantom{a}
\begin{enumerate}[label = \arabic*. ]
%1
\item \#4
%2
\item \#8
%3
\item \#9
\end{enumerate}   


%B
\item \phantom{a} 
\begin{enumerate}[label = \arabic*. ]
%1
\item \#1
%2
\item \#8
%3
\item \#10
\end{enumerate}   

%C
\item \phantom{a} 
\begin{enumerate}[label = \arabic*. ]
%1

%2
\item \#4
%3
\item \#10
\item \#13
\item \#14
\end{enumerate}   

\end{multicols} 
\end{enumerate}   
}

\def \RemediationDayB {}

\def \LessonDayC {\makecell{Arithmetic Series}}

\def \LearningCompetenciesDayC {}

\def \ObjectivesDayC {
\item Derive the formula in computing arithmetic series; 
\item Compute the number of terms of a given arithmetic series; and, 
\item Display enjoyment and independence in solving problems.
}

\def \TeachersGuideDayC {pp. 31--41}

\def \LMPagesDayC {pp. 19--25}

\def \TextbookPagesDayC {pp. 26--35}

\def \AdditionalMaterialsDayC {}

\def \OtherResourcesDayC {Flashcards}

\def \ReviewDayC {\begin{center}
\textbf{Arithmetic Series}
\end{center}

\vspce

\textbf{Arithmetic Series:} the indicated sum of the terms of an  arithmetic sequence 

\vspce

If the first term and the $n^{th}$ term are given, then 

$$\displaystyle S_{n}=\frac{n}{2}(a_{1}+a_{n})$$

\vspce

If the $n^{th}$ term is not  given, then 

$$\displaystyle S_{n}=\frac{n}{2}[2a_{1}+(n-1)d]$$

}

\def \PurposeDayC {The purpose of this lesson is to enable the students to solve real life problems involving arithmetic series.}  

\def \ExamplesDayC {Some examples of arithmetic series are the following: 
\begin{enumerate}[label = \arabic*. ]
%1
\item The sum of even integers from 5 to 11 is 24.
%2
\item $1 + 3 + 5 + 7 = 16$
%3
%\item 
\end{enumerate}   
}

\def \PracticeOneDayC {\textbf{Practice Exercises}

\vspce

A. Find the sum of each arithmetic sequence. 
\begin{enumerate}
\item \hspce 2, 5, 8,... to 8 terms
\item \hspce -11, -7, -3,... to 23 terms
\item \hspce Sum of odd integers from 1 to 100
\item \hspce Sum of the  integers between  50 and 200 which are divisible by 5
\end{enumerate}
}

\def \PracticeTwoDayC {B. In each arithmetic series, find the specified unknown. 
\begin{enumerate}
\item \hspce $S_{n}$ = 90, $a_{1}$=10, $a_{n}$=26, $n$=? 
\item \hspce $S_{n}$ = 1,800, $a_{n}$=185, $n$=18, $a_{1}$=? 
\item \hspce $S_{n}$ = 119, $a_{1}$=5, $d$=4, $n$=? 
\item \hspce $a_{10}$ = 27.5, $d$=3, $a_{1}$=?, $S_{n}$=?

\end{enumerate}

}

\def \MasteryDayC {\textbf{Problem Set}

\vspce

A. Find the sum of each arithmetic sequence. 
\begin{enumerate}
\item \hspce 3, 5, 7,... to 31 terms
\item \hspce 10, -2, -14,... to 17 terms
\item \hspce Sum of even integers from 10 to 90
\item \hspce Sum of the  integers between  2 and 100 which are divisible by 3
\end{enumerate}

\vspce

B. In each arithmetic series, find the specified unknown. 
\begin{enumerate}
\item \hspce $S_{n}$ = 50, $a_{1}$=4, $a_{n}$=16, $n$=? 
%\item \hspce $S_{n}$ = 195, $a_{n}$=33, $d$=3, $a_{1}$=? no solution! 
\item \hspce $S_{n}$ = -15, $a_{1}$=12, $d$=-3, $n$=? 
\item \hspce Sum of even integers between  20 and 80
\end{enumerate}
}

\def \ApplicationDayC { Let the students answer the following questions: 
\begin{enumerate}[label = \arabic*. ]
%1
\item In what real life situations or problems can we observe some examples of arithmetic series? 
%2
\item How can you apply your knowledge of arithmetic series in solving these real life problems? 
%3
%\item 
\end{enumerate}   
}

\def \GeneralizationDayC {Let the students answer the following questions: 
\begin{enumerate}[label = \arabic*. ]
%1
\item In your own words, what are arithmetic series? 
%2
\item How do we solve problems involving arithmetic series? 
%3
%\item 
\end{enumerate}   
}

\def \EvaluationDayC {
\begin{center}
\textbf{Quiz \#3}
\end{center} 

Math Time pp. 12
\begin{enumerate}[label = \Alph*. ]
\begin{multicols}{2}

%A
\item \phantom{a}
\begin{enumerate}[label = \arabic*. ]
%1
\item \#8
%2
\item \#11
%3
\item \#14
\item \#15
\end{enumerate}   


%B
\item \phantom{a}
\begin{enumerate}[label = \arabic*. ]
%1
\item \#1
%2
\item \#6
%3
\item \#7
\item \#10

\end{enumerate}   


\end{multicols} 
\end{enumerate}   
}

\def \RemediationDayC {}

\def \LessonDayD {\makecell{Geometric Sequences}}

\def \LearningCompetenciesDayD {}

\def \ObjectivesDayD {
\item Apply the steps in finding the rule of a given geometric sequence; 
\item Generate the next terms of a geometric sequence; and, 
\item Display determination and interest in solving problems.
}

\def \TeachersGuideDayD {pp. 42--50}

\def \LMPagesDayD {pp. 26--30}

\def \TextbookPagesDayD {pp. 36--42}

\def \AdditionalMaterialsDayD {}

\def \OtherResourcesDayD {Flashcards}

\def \ReviewDayD {\begin{center}
\textbf{Geometric Sequences}
\end{center}

\vspace*{1ex}

\textbf{Geometric Sequence:} a sequence in which each term after the first is obtained by multiplying the preceding term by a fixed nonzero constant 

\vspce

\textbf{Common Ratio ($r$):} the fixed constant 

\vspce

To  find  any  term  in  a  geometric sequence, use

%\centering \scalebox{1.5}{\color{black} 
$$a_{n} = a_{1}r^{n - 1}$$ 
%}


}

\def \PurposeDayD {The purpose of this lesson is to enable the students to solve real life problems involving geometric sequences.}  

\def \ExamplesDayD {Some examples of geometric sequences are the following: 
\begin{enumerate}[label = \arabic*. ]
%1
\item The sum of even integers from 5 to 11 is 24.
%2
\item $1 + 3 + 5 + 7 = 16$
%3
%\item 
\end{enumerate}   
}

\def \PracticeOneDayD {\textbf{Practice Exercises}

\vspce

\begin{enumerate}[label = \Alph*. ]
\item \hspce Find the common ratio and the next three terms of each geometric sequence.   
	\begin{enumerate}[label = \arabic*. ]
	

	\item \hspce $2, 6, 18, 54,\ldots$
	\item \hspce $\displaystyle \frac{1}{8}, \frac{1}{4}, \frac{1}{2},\ldots $
	\item \hspce $4, 12, 36,\ldots$
	\item \hspce $0.02, 0.2, 2,\ldots$
	\item \hspce $3x^{3}, 6x^{5}, 12x^{7},\ldots $
	 
	\end{enumerate}


\item \hspce Find the specified term of each geometric sequence. 
	\begin{enumerate}[label = \arabic*. ]
	\item \hspce $3, 6, 12,\ldots \hspce  a_{7}$
	\item \hspce $4, 20, 100,\ldots \hspce  a_{8}$
	\item \hspce $7, -7, 7,\ldots \hspce  a_{17}$
	\item \hspce $3, 1.2, 0.48,\ldots \hspce  a_{10}$
	\item \hspce $\displaystyle 1, \frac{3}{2}, \frac{9}{4},\ldots \hspce  a_{11}$
	\end{enumerate}

\end{enumerate}}

\def \PracticeTwoDayD {Solve each problem completely. 
	\begin{enumerate}[label = \arabic*. ]
	\item \hspce The first term of a geometric sequence is 8, and the second term is 4. Find the fifth term. 

	\item \hspce The first term of a geometric sequence is 3, and the third term is $\frac{4}{3}$. Find the fifth term. 

	\item \hspce The common ratio in a geometric sequence is $\frac{2}{5}$ and the fourth term is $\frac{5}{2}$. Find the third term. 

	\item \hspce Which term of the geometric sequence 2, 6, 18,... is 118098?
	\item \hspce The second and fifth terms of a geometric sequence are 10 and 1250, respectively.  Is 31,250 a term of this sequence? If so, which term is it?
	\end{enumerate}
}

\def \MasteryDayD {\textbf{Problem Set}

\vspce

\begin{enumerate}[label = \Alph*. ]
\item \hspce Find the common ratio and the next three terms of each geometric sequence.   
	\begin{enumerate}[label = \arabic*. ]
	
	\item \hspce $4, 8, 16, 32,\ldots$
	\item \hspce $\displaystyle \frac{4}{9}, \frac{4}{3}, 4,\ldots $
	\item \hspce $1, -5, 25,\ldots$
	\item \hspce $-5, -0.5, -0.05,\ldots$
	\item \hspce $x, 5x^{2}y, 25x^{3}y^{2},\ldots $
	 
	\end{enumerate}

\item \hspce Find the specified term of each geometric sequence. 
	\begin{enumerate}[label = \arabic*. ]
	\item \hspce $64, -32, 16,\ldots \hspce  a_{7}$
	\item \hspce $2, -10, 50,\ldots \hspce  a_{8}$
	\item \hspce $2, -6, 18,\ldots \hspce  a_{13}$
	\item \hspce $3, 1.2, 0.48,\ldots \hspce  a_{10}$
	\item \hspce $\displaystyle  \frac{1}{16}, \frac{1}{8}, \frac{1}{4},\ldots \hspce  a_{9}$
	\end{enumerate}


\item \hspce Solve each problem completely. 
	\begin{enumerate}[label = \arabic*. ]
	\item \hspce The first term of a geometric sequence is -2, and the third term is $-\frac{1}{2}$. Find the fifth term. 

	\item \hspce The common ratio in a geometric sequence is $\frac{2}{3}$ and the fourth term is 1. Find the third term. 

	\item \hspce The common ratio in a geometric sequence is $\frac{3}{4}$ and the fifth term is $\frac{81}{16}$. Find the first three terms.

	\item \hspce Which term of the geometric sequence 3, 6, 12,... is 768?
	\item \hspce The common ratio in a geometric sequence is $\frac{3}{2}$ and the fifth term is 1. Find the first three terms.

	\end{enumerate}
\end{enumerate}}

\def \ApplicationDayD { Let the students answer the following questions: 
\begin{enumerate}[label = \arabic*. ]
%1
\item In what real life situations or problems can we observe some examples of geometric sequences? 
%2
\item How can you apply your knowledge of geometric sequences in solving these real life problems? 
%3
%\item 
\end{enumerate}   
}

\def \GeneralizationDayD {Let the students answer the following questions: 
\begin{enumerate}[label = \arabic*. ]
%1
\item In your own words, what are geometric sequences? 
%2
\item How do we solve problems involving geometric sequences? 
%3
%\item 
\end{enumerate}   
}

\def \EvaluationDayD {
\input{/host-rootfs/storage/emulated/0/Documents/documents/latex/1920/Grade-10/1st/geometric-sequences/qz-geometric-sequences-input} 
}

\def \RemediationDayD {}

\def \LessonDayE {\makecell{Finite Geometric Series}}

\def \LearningCompetenciesDayE {}

\def \ObjectivesDayE {
\item Apply the steps in finding the rule of a given geometric sequence; 
\item Generate the next terms of a geometric sequence; and, 
\item Display enjoyment and interest in solving problems.
}

\def \TeachersGuideDayE {pp. 51--62}

\def \LMPagesDayE {pp. 31--37}

\def \TextbookPagesDayE {pp. 43--52}

\def \AdditionalMaterialsDayE {}

\def \OtherResourcesDayE {Flashcards}

\def \ReviewDayE {\begin{center}
\textbf{Finite Geometric Series}
\end{center}

\vspace*{1ex}

\textbf{Finite Geometric Series:} the indicated sum of the terms of a geometric sequence 

\vspce

If the last term $a_n$ is not given, use

%\begin{center}
$$\displaystyle S_n = \frac{a_1(1-r^n)}{1-r},\hspce r\neq1$$
%\end{center}

\vspce

If the last term $a_n$ is given, use

%\begin{center}
$$\displaystyle S_n = \frac{a_1-a_{n}r}{1-r}, \hspce r\neq1$$
%\end{center}

}

\def \PurposeDayE {The purpose of this lesson is to enable the students to solve real life problems involving finite geometric series.}  

\def \ExamplesDayE {Some examples of finite geometric series are the following: 
\begin{enumerate}[label = \arabic*. ]
%1
\item The sum of even integers from 5 to 11 is 24.
%2
\item $1 + 3 + 5 + 7 = 16$
%3
%\item 
\end{enumerate}   
}

\def \PracticeOneDayE {\textbf{Practice Exercises}

\vspce

A. Find the indicated sum of the following geometric series.   

\begin{enumerate}[label = \arabic*. ]

%begin{multicols}{2}


	\item \hspce $1+4+16+\ldots S_6$
	\item \hspce $2+4+8+16+\ldots S_{10}$
	\item \hspce $2+6+18+\ldots S_7$
	\item \hspce $(-9)+6+(-4)+\ldots S_8$
	\item \hspce $2+2\sqrt{2}+4+\ldots S_{10}$
	
%end{multicols} 
\end{enumerate}}

\def \PracticeTwoDayE {B. Find each specified term.  

\begin{enumerate}[label = \arabic*. ]
%

\item \hspce $S_5 = \displaystyle \frac{31}{4}; \hspce r=\frac{1}{2}; \hspce a_1=? $
\item \hspce $\displaystyle S_8 = 2,550; \hspce r=2;\hspce a_1=? $
\item \hspce $\displaystyle S_7 = 7,651;\hspce  r=3;\hspce a_1=? $
\item \hspce $\displaystyle S_{10} = 51,150; \hspce r=2;\hspce a_1=? $
\item \hspce $S_6 = 126;\hspce \displaystyle r=-\frac{1}{2};\hspce a_6=? $

%
\end{enumerate}
}

\def \MasteryDayE {\textbf{Problem Set }

\vspce

\begin{enumerate}[label = \Alph*. ]
\item Find the indicated sum of the following geometric series.   
  
\begin{enumerate}[label = \arabic*. ]
%begin{multicols}{2}


	\item \hspce $9+6+4+\ldots S_7$ 
	\item \hspce $2+8+32+\ldots S_{9}$
	\item \hspce $3+3\sqrt{3}+9+\ldots S_{9}$
	\item \hspce $1+(-2)+4+(-8)\ldots S_8$
	\item \hspce $(-2)+6+(-18)+\ldots S_6$ 
	
%end{multicols} 

\end{enumerate}

\vspce

\item Find the sum of the first $n$ terms of the related geometric series. 

\begin{enumerate}[label = \arabic*. ]

%begin{multicols}{2}

\item \hspce $\displaystyle a_1=\frac{1}{2}; r=4; n=6 $
\item \hspce $\displaystyle a_1=13; r=4; n=7 $ 
\item \hspce $\displaystyle a_1=318; \hspce r=\frac{1}{2};\hspce n=7 $
\item \hspce $\displaystyle a_1=168;\hspce r=\frac{3}{4};\hspce n=8 $
\item \hspce $\displaystyle a_1=4;\hspce r=-5; \hspce n=8 $


%end{multicols} 
\end{enumerate}
\end{enumerate}}

\def \ApplicationDayE { Let the students answer the following questions: 
\begin{enumerate}[label = \arabic*. ]
%1
\item In what real life situations or problems can we observe some examples of finite geometric series? 
%2
\item How can you apply your knowledge of finite geometric series in solving these real life problems? 
%3
%\item 
\end{enumerate}   
}

\def \GeneralizationDayE {Let the students answer the following questions: 
\begin{enumerate}[label = \arabic*. ]
%1
\item In your own words, what are finite geometric series? 
%2
\item How do we solve problems involving finite geometric series? 
%3
%\item 
\end{enumerate}   
}

\def \EvaluationDayE {
\input{/host-rootfs/storage/emulated/0/Documents/documents/latex/1920/Grade-10/1st/finite-geometric-series/qz-finite-geometric-series-input} 
}

\def \RemediationDayE {}
