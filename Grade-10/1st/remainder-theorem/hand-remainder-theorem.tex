% cd /storage/emulated/0/Documents/documents/latex/1920/Grade-10/1st/remainder-theorem && pdflatex hand-remainder-theorem.tex && termux-open hand-remainder-theorem.pdf


\documentclass[handout]{beamer} 

\usepackage{pgfpages} 
\mode<handout>{%
\pgfpagesuselayout{4 on 1}[%letterpaper, %
legalpaper, %landscape, 
border shrink=1mm] 
%\setbeameroption{show notes} 
}

\usepackage{xcolor}
\usepackage{anyfontsize}
\usepackage{enumitem}
\usepackage{multicol}
\usepackage{amsmath}
%\usepackage{amsfonts,dsfont}% for \mathds 
\usepackage{tabularx} 
\usepackage{gensymb}
\usepackage{multirow}
\usepackage{graphicx, tipa}
\usepackage{tikz}
\usetikzlibrary{angles,quotes}
\usepackage{pgfplots} 
\usetikzlibrary{calc}
\pgfplotsset{compat=newest}
\usetikzlibrary{arrows.meta}
\usetikzlibrary{intersections}
\usetikzlibrary{decorations.pathreplacing}
\usepackage{flafter}
\usepackage{amsmath,amssymb,cancel,units}
\usepackage{microtype} % nicer output 
\usepackage{hfoldsty} % nicer output 
\usepackage{fixltx2e} 
\usepackage{mathptmx}
%\usepackage{booktabs}
\usepackage{numprint}
\usepackage[utf8]{inputenc} 
\usepackage[T1]{fontenc}
%\usepackage{siunitx} 
%\sisetup{detect-all}


%\def\vadjust{-1.5in}

%\def\radB{3.6cm}

%\def\thirdrad{8cm}

\pagenumbering{gobble}
%\linespread{0.9}
\newcommand{\vspce}{\vspace{0.75ex}}
\newcommand{\hspce}{\hspace{0.5em}}
\newcommand{\vertadjust}{\vspace*{-2.5in}}
\newcommand{\blank}{\underline{\hspace{2em}}}%{\rule{1em}{0.15ex}}
\newcommand{\arc}[1]{{% 
\setbox9=\hbox{#1}% 
\ooalign{\resizebox{\wd9}{\height}{\texttoptiebar{\phantom{A}}}\cr#1}}}

\newcolumntype{C}{ >{\centering\arraybackslash} X}

%\frame[shrink=5] 
\begin{document} 
\vertadjust
\begin{frame} 
\begin{center}
\textbf{Remainder Theorem 
}
\end{center}

\vspce 

Remainder Theorem: If a polynomial $P(x) $ is divided by $x-c$, then the remainder is $P(c)$. 
\[ R = P(c)\] 

\vspce 

Ways to Find the Remainder: 
\begin{enumerate}[label = \arabic*. ]
\item \hspce Use synthetic division. 
\item \hspce Calculate $P(c)$. 

\end{enumerate}  

 
\textbf{Practice Exercises}
%\textbf{Problem Set}

\vspce
%\begin{enumerate}[label = \Alph*. ]

%\item 
A. Use synthetic division to find the remainder of the following polynomial functions. 

%\vspce

\begin{enumerate}[label = \arabic*. ]
%%begin{multicols}{2}

%#1
\item \hspce \hspce $f(x) = -x^3+6x-7$ at $x = 2$ 
\vspce
%#2
\item \hspce \hspce $f(x) = x^3+3x^2+2x+8$  at $x = -3$ 
\vspce
%#3
\item \hspce \hspce $f(x) = x^4+3x^3-17x^2+2x-7$ at $x=3$ 
\vspce
%#4
\item \hspce \hspce $f(x) = 3x^3+7x^2-18x+8$ at $x = -4$ 
\vspce
%#5
\item \hspce \hspce $f(x) = 2x^4-3x^3-3x-2$ at $x = 2$ 


%%end{multicols}
\end{enumerate}




%\end{enumerate}
B. Use the remainder theorem  to find the remainder of the following polynomial functions. 

\begin{enumerate}[label = \arabic*. ]
%#1
\item \hspce \hspce $f(x) = 4x^3+2x+10$ at $x = -3$ 
\vspce
%#2
\item \hspce \hspce $f(x) = 2x^3+4x^2-5x+9$ at $x = -3$ 
\vspce
%#3
\item \hspce \hspce $f(x) = 3x^3-7x^2+5x-2$ at $x = -2$ 
\vspce
%#4
\item \hspce \hspce $f(x) = 5x^3+7x^2+8$ at $x = -2$ 
\vspce
%#5
\item \hspce \hspce $f(x) = 6x^2+3x-9$ at $x = 1$ 


\end{enumerate} 
 

%\textbf{Practice Exercises}
\textbf{Problem Set}

\vspce
\begin{enumerate}[label = \Alph*. ]

\item 
Use synthetic division to find the remainder of the following polynomial functions. 

%\vspce

\begin{enumerate}[label = \arabic*. ]
%%begin{multicols}{2}

%#1
\item \hspce \hspce $f(x) = x^3+x^2-5x-6$ at $x = 2$ 
\vspce
%#2
\item \hspce \hspce $f(x) = x^3+5x^2+10x+12$  at $x = -2$ 
\vspce
%#3
\item \hspce \hspce $f(x) = x^5-47x^3-16x^2+8x+52$ at $x = 7$ 
\vspce
%#4
\item \hspce \hspce $f(x) = x^4-2x^3+x^2-4$ at $x = -1$ 
\vspce
%#5
\item \hspce \hspce $f(x) = x^2-5x-2$ at $x = -2$ 


%%end{multicols}
\end{enumerate}

\item Use the remainder theorem  to find the remainder of the following polynomial functions. 

\begin{enumerate}[label = \arabic*. ]
%#1
\item \hspce \hspce $f(x) = 2x^3-5x^2+3x-7$ at $x = 3$ 
\vspce
%#2
\item \hspce \hspce $f(x) = 2x^3-9x^2+14x-8$ at $x = -2$ 
\vspce
%#3
\item \hspce \hspce $f(x) = 4x^4+5x^3+8x^2$ at $x = 4$ 
\vspce
%#4
\item \hspce \hspce $f(x) = 5x^4+6x^3+10x^2$ at $x = 5$ 
\vspce
%#5
\item \hspce \hspce $f(x) = 2x^4-9x^3+14x^2-8$ at $x = 2$ 


\end{enumerate} 
 



\end{enumerate}
\end{frame}

\vertadjust
\begin{frame} 
\begin{center}
\textbf{Remainder Theorem 
}
\end{center}

\vspce 

Remainder Theorem: If a polynomial $P(x) $ is divided by $x-c$, then the remainder is $P(c)$. 
\[ R = P(c)\] 

\vspce 

Ways to Find the Remainder: 
\begin{enumerate}[label = \arabic*. ]
\item \hspce Use synthetic division. 
\item \hspce Calculate $P(c)$. 

\end{enumerate}  

 
\textbf{Practice Exercises}
%\textbf{Problem Set}

\vspce
%\begin{enumerate}[label = \Alph*. ]

%\item 
A. Use synthetic division to find the remainder of the following polynomial functions. 

%\vspce

\begin{enumerate}[label = \arabic*. ]
%%begin{multicols}{2}

%#1
\item \hspce \hspce $f(x) = -x^3+6x-7$ at $x = 2$ 
\vspce
%#2
\item \hspce \hspce $f(x) = x^3+3x^2+2x+8$  at $x = -3$ 
\vspce
%#3
\item \hspce \hspce $f(x) = x^4+3x^3-17x^2+2x-7$ at $x=3$ 
\vspce
%#4
\item \hspce \hspce $f(x) = 3x^3+7x^2-18x+8$ at $x = -4$ 
\vspce
%#5
\item \hspce \hspce $f(x) = 2x^4-3x^3-3x-2$ at $x = 2$ 


%%end{multicols}
\end{enumerate}




%\end{enumerate}
B. Use the remainder theorem  to find the remainder of the following polynomial functions. 

\begin{enumerate}[label = \arabic*. ]
%#1
\item \hspce \hspce $f(x) = 4x^3+2x+10$ at $x = -3$ 
\vspce
%#2
\item \hspce \hspce $f(x) = 2x^3+4x^2-5x+9$ at $x = -3$ 
\vspce
%#3
\item \hspce \hspce $f(x) = 3x^3-7x^2+5x-2$ at $x = -2$ 
\vspce
%#4
\item \hspce \hspce $f(x) = 5x^3+7x^2+8$ at $x = -2$ 
\vspce
%#5
\item \hspce \hspce $f(x) = 6x^2+3x-9$ at $x = 1$ 


\end{enumerate} 
 

%\textbf{Practice Exercises}
\textbf{Problem Set}

\vspce
\begin{enumerate}[label = \Alph*. ]

\item 
Use synthetic division to find the remainder of the following polynomial functions. 

%\vspce

\begin{enumerate}[label = \arabic*. ]
%%begin{multicols}{2}

%#1
\item \hspce \hspce $f(x) = x^3+x^2-5x-6$ at $x = 2$ 
\vspce
%#2
\item \hspce \hspce $f(x) = x^3+5x^2+10x+12$  at $x = -2$ 
\vspce
%#3
\item \hspce \hspce $f(x) = x^5-47x^3-16x^2+8x+52$ at $x = 7$ 
\vspce
%#4
\item \hspce \hspce $f(x) = x^4-2x^3+x^2-4$ at $x = -1$ 
\vspce
%#5
\item \hspce \hspce $f(x) = x^2-5x-2$ at $x = -2$ 


%%end{multicols}
\end{enumerate}

\item Use the remainder theorem  to find the remainder of the following polynomial functions. 

\begin{enumerate}[label = \arabic*. ]
%#1
\item \hspce \hspce $f(x) = 2x^3-5x^2+3x-7$ at $x = 3$ 
\vspce
%#2
\item \hspce \hspce $f(x) = 2x^3-9x^2+14x-8$ at $x = -2$ 
\vspce
%#3
\item \hspce \hspce $f(x) = 4x^4+5x^3+8x^2$ at $x = 4$ 
\vspce
%#4
\item \hspce \hspce $f(x) = 5x^4+6x^3+10x^2$ at $x = 5$ 
\vspce
%#5
\item \hspce \hspce $f(x) = 2x^4-9x^3+14x^2-8$ at $x = 2$ 


\end{enumerate} 
 



\end{enumerate}
\end{frame}

\vspace*{-2.7in}
\begin{frame} 
\begin{center}
\textbf{Remainder Theorem 
}
\end{center}

\vspce 

Remainder Theorem: If a polynomial $P(x) $ is divided by $x-c$, then the remainder is $P(c)$. 
\[ R = P(c)\] 

\vspce 

Ways to Find the Remainder: 
\begin{enumerate}[label = \arabic*. ]
\item \hspce Use synthetic division. 
\item \hspce Calculate $P(c)$. 

\end{enumerate}  

 
\textbf{Practice Exercises}
%\textbf{Problem Set}

\vspce
%\begin{enumerate}[label = \Alph*. ]

%\item 
A. Use synthetic division to find the remainder of the following polynomial functions. 

%\vspce

\begin{enumerate}[label = \arabic*. ]
%%begin{multicols}{2}

%#1
\item \hspce \hspce $f(x) = -x^3+6x-7$ at $x = 2$ 
\vspce
%#2
\item \hspce \hspce $f(x) = x^3+3x^2+2x+8$  at $x = -3$ 
\vspce
%#3
\item \hspce \hspce $f(x) = x^4+3x^3-17x^2+2x-7$ at $x=3$ 
\vspce
%#4
\item \hspce \hspce $f(x) = 3x^3+7x^2-18x+8$ at $x = -4$ 
\vspce
%#5
\item \hspce \hspce $f(x) = 2x^4-3x^3-3x-2$ at $x = 2$ 


%%end{multicols}
\end{enumerate}




%\end{enumerate}
B. Use the remainder theorem  to find the remainder of the following polynomial functions. 

\begin{enumerate}[label = \arabic*. ]
%#1
\item \hspce \hspce $f(x) = 4x^3+2x+10$ at $x = -3$ 
\vspce
%#2
\item \hspce \hspce $f(x) = 2x^3+4x^2-5x+9$ at $x = -3$ 
\vspce
%#3
\item \hspce \hspce $f(x) = 3x^3-7x^2+5x-2$ at $x = -2$ 
\vspce
%#4
\item \hspce \hspce $f(x) = 5x^3+7x^2+8$ at $x = -2$ 
\vspce
%#5
\item \hspce \hspce $f(x) = 6x^2+3x-9$ at $x = 1$ 


\end{enumerate} 
 

%\textbf{Practice Exercises}
\textbf{Problem Set}

\vspce
\begin{enumerate}[label = \Alph*. ]

\item 
Use synthetic division to find the remainder of the following polynomial functions. 

%\vspce

\begin{enumerate}[label = \arabic*. ]
%%begin{multicols}{2}

%#1
\item \hspce \hspce $f(x) = x^3+x^2-5x-6$ at $x = 2$ 
\vspce
%#2
\item \hspce \hspce $f(x) = x^3+5x^2+10x+12$  at $x = -2$ 
\vspce
%#3
\item \hspce \hspce $f(x) = x^5-47x^3-16x^2+8x+52$ at $x = 7$ 
\vspce
%#4
\item \hspce \hspce $f(x) = x^4-2x^3+x^2-4$ at $x = -1$ 
\vspce
%#5
\item \hspce \hspce $f(x) = x^2-5x-2$ at $x = -2$ 


%%end{multicols}
\end{enumerate}

\item Use the remainder theorem  to find the remainder of the following polynomial functions. 

\begin{enumerate}[label = \arabic*. ]
%#1
\item \hspce \hspce $f(x) = 2x^3-5x^2+3x-7$ at $x = 3$ 
\vspce
%#2
\item \hspce \hspce $f(x) = 2x^3-9x^2+14x-8$ at $x = -2$ 
\vspce
%#3
\item \hspce \hspce $f(x) = 4x^4+5x^3+8x^2$ at $x = 4$ 
\vspce
%#4
\item \hspce \hspce $f(x) = 5x^4+6x^3+10x^2$ at $x = 5$ 
\vspce
%#5
\item \hspce \hspce $f(x) = 2x^4-9x^3+14x^2-8$ at $x = 2$ 


\end{enumerate} 
 



\end{enumerate}
\end{frame}

\vspace*{-2.7in}
\begin{frame} 
\begin{center}
\textbf{Remainder Theorem 
}
\end{center}

\vspce 

Remainder Theorem: If a polynomial $P(x) $ is divided by $x-c$, then the remainder is $P(c)$. 
\[ R = P(c)\] 

\vspce 

Ways to Find the Remainder: 
\begin{enumerate}[label = \arabic*. ]
\item \hspce Use synthetic division. 
\item \hspce Calculate $P(c)$. 

\end{enumerate}  

 
\textbf{Practice Exercises}
%\textbf{Problem Set}

\vspce
%\begin{enumerate}[label = \Alph*. ]

%\item 
A. Use synthetic division to find the remainder of the following polynomial functions. 

%\vspce

\begin{enumerate}[label = \arabic*. ]
%%begin{multicols}{2}

%#1
\item \hspce \hspce $f(x) = -x^3+6x-7$ at $x = 2$ 
\vspce
%#2
\item \hspce \hspce $f(x) = x^3+3x^2+2x+8$  at $x = -3$ 
\vspce
%#3
\item \hspce \hspce $f(x) = x^4+3x^3-17x^2+2x-7$ at $x=3$ 
\vspce
%#4
\item \hspce \hspce $f(x) = 3x^3+7x^2-18x+8$ at $x = -4$ 
\vspce
%#5
\item \hspce \hspce $f(x) = 2x^4-3x^3-3x-2$ at $x = 2$ 


%%end{multicols}
\end{enumerate}




%\end{enumerate}
B. Use the remainder theorem  to find the remainder of the following polynomial functions. 

\begin{enumerate}[label = \arabic*. ]
%#1
\item \hspce \hspce $f(x) = 4x^3+2x+10$ at $x = -3$ 
\vspce
%#2
\item \hspce \hspce $f(x) = 2x^3+4x^2-5x+9$ at $x = -3$ 
\vspce
%#3
\item \hspce \hspce $f(x) = 3x^3-7x^2+5x-2$ at $x = -2$ 
\vspce
%#4
\item \hspce \hspce $f(x) = 5x^3+7x^2+8$ at $x = -2$ 
\vspce
%#5
\item \hspce \hspce $f(x) = 6x^2+3x-9$ at $x = 1$ 


\end{enumerate} 
 

%\textbf{Practice Exercises}
\textbf{Problem Set}

\vspce
\begin{enumerate}[label = \Alph*. ]

\item 
Use synthetic division to find the remainder of the following polynomial functions. 

%\vspce

\begin{enumerate}[label = \arabic*. ]
%%begin{multicols}{2}

%#1
\item \hspce \hspce $f(x) = x^3+x^2-5x-6$ at $x = 2$ 
\vspce
%#2
\item \hspce \hspce $f(x) = x^3+5x^2+10x+12$  at $x = -2$ 
\vspce
%#3
\item \hspce \hspce $f(x) = x^5-47x^3-16x^2+8x+52$ at $x = 7$ 
\vspce
%#4
\item \hspce \hspce $f(x) = x^4-2x^3+x^2-4$ at $x = -1$ 
\vspce
%#5
\item \hspce \hspce $f(x) = x^2-5x-2$ at $x = -2$ 


%%end{multicols}
\end{enumerate}

\item Use the remainder theorem  to find the remainder of the following polynomial functions. 

\begin{enumerate}[label = \arabic*. ]
%#1
\item \hspce \hspce $f(x) = 2x^3-5x^2+3x-7$ at $x = 3$ 
\vspce
%#2
\item \hspce \hspce $f(x) = 2x^3-9x^2+14x-8$ at $x = -2$ 
\vspce
%#3
\item \hspce \hspce $f(x) = 4x^4+5x^3+8x^2$ at $x = 4$ 
\vspce
%#4
\item \hspce \hspce $f(x) = 5x^4+6x^3+10x^2$ at $x = 5$ 
\vspce
%#5
\item \hspce \hspce $f(x) = 2x^4-9x^3+14x^2-8$ at $x = 2$ 


\end{enumerate} 
 



\end{enumerate}
\end{frame}

\end{document}

