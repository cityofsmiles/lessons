\begin{center}
\textbf{Zeros of a Polynomial Function}
\end{center}

\vspace*{1ex}

If $(x-c)$ is a factor of a polynomial $P(x)$,  then $c$ is called a \textbf{zero of the polynomial function}.

\vspce 

\textbf{Multiple Zeros of a Polynomial:} If a polynomial $P(x)$ has $x-c$ occurring as a factor exactly $k$ times, then $c$ is a \textbf{zero of multiplicity $k$} of the polynomial function $y=P(x) $. 

\vspce 

\textbf{Fundamental Theorem of Algebra:} A polynomial function $P(x) $ of degree $n$ has exactly $n$ complex zeros. 

\vspce 

\textbf{Integral Zero Theorem: }If an integer is a zero of a given integral polynomial function, then it is a divisor of the constant term of the polynomial.

