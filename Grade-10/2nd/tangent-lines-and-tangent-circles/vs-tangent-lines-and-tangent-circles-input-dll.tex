\begin{center}
\textbf{Tangent Lines and Tangent Circles}
\end{center}

\vspace*{1ex}

\begin{center}
\scalebox{0.5}{
\noindent\begin{minipage}{0.3\textwidth}
{
\textbf{Tangent Line:} a line in the plane of the circle that intersects the circle at exactly one point

\vspce 

\textbf{Point of Tangency:} the point of intersection 

\vspce 

\textbf{Tangent Circles:} two circles whose intersection is exactly one point

\vspce 

\textbf{Common Tangent:} a line which is tangent to two circles

\vspce 

\textbf{Common Internal Tangent:} a common tangent which intersects the segment joining the centers of two circles

\vspce 

\textbf{Common External Tangent:} a common tangent which does not intersect the segment joining the centers of two circles

\vspce 

\textbf{Internally Tangent Circles:} circles that are coplanar, share a common point of tangency, and with centers that lie on the same side of their common tangent 

\vspce 

\textbf{Externally Tangent Circles:} circles that are coplanar, share a common point of tangency, and with centers that lie on the opposite sides of their common tangent 

\vspce 

\textbf{Tangent Line Theorem:} If a line is tangent to a circle, then it is perpendicular to the radius drawn to the point of tangency. 
 
\vspce 

\textbf{Converse of the Tangent Line Theorem:} In a plane, if a line is perpendicular to a radius of a circle at the endpoint, then it is  drawn to the point of tangency. 
 
\vspce 

\textbf{Tangent Segments Theorem:} If two tangent segments are drawn to a circle from an external point, then
\begin{enumerate}[label = \alph*. ]
\item the two tangent segments are congruent, and
\item the angles between the tangent segments and the line joining the external point to the center of the circle are congruent 
\end{enumerate}  

\vspce 

\textbf{Tangent Circles Theorem:} If two circles are tangent internally or externally, then their line of centers pass through the  point of contact. 
}
\end{minipage}}
\end{center} 


