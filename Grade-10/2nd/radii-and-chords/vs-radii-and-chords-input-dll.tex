\begin{center}
\textbf{Radii and Chords}
\end{center}

\vspace*{1ex}

\begin{center}
\scalebox{0.5}{
\noindent\begin{minipage}{0.3\textwidth}

{
\textbf{Perpendicular to a Chord Theorem:} The perpendicular from the center of the circle to any chord bisects the chord. 

\vspce 

\textbf{Center to Chord Midpoint Theorem:} The line joining the center of the circle to the midpoint of any chord which is not a diameter is perpendicular to the chord. 

\vspce 

\textbf{Perpendicular Bisector Chord to Center Theorem:}  The perpendicular bisector of a chord of a circle passes through the center of the circle. 

\vspce 

\textbf{Perpendicular Bisector Chord to Central Angle Theorem:} The perpendicular bisector of a chord of a circle bisects the central angle subtended by the chord. 

\vspce 

\textbf{Central Angle Bisector Theorem:} The bisector of a central angle subtended by the chord is the perpendicular bisector of the chord. 

\vspce 

\textbf{Distance--Chord Theorem:} In the same circle or in congruent circles, chords are congruent if and only if their distances from the center(s) of the circle(s) are equal. 


\vspce 

\textbf{Chord -- Arc Congruence Theorem:} In a circle or in congruent circles, two minor arcs are congruent if and only if their corresponding chords are congruent. 
}
\end{minipage}}
\end{center} 

 


 




