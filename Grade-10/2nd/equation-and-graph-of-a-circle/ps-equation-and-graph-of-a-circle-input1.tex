\def\curdir{/storage/emulated/0/Documents/documents/latex/1920/Grade-10/2nd/equation-and-graph-of-a-circle/fa}

\textbf{Practice Exercises}
%\textbf{Problem Set}

\vspce

%\begin{enumerate}[label = \Alph*. ]
%A
%\item \hspce 
A. Give the radius and coordinates of the center of the circle. Then write the equation in standard form. 
\begin{enumerate}[label = \arabic*. ]

\begin{multicols}{3}
%1 
\item \begin{tikzpicture}[scale=0.3]

\fill [fill=\drawfill] (4,2) circle (2);

\draw [line width=0.1mm] (-1,-3) grid (7,5);
 
\draw[line width=0.3mm, <->, >={Latex[round]}] (-1.5, 0) -- (7.5, 0);

\draw[line width=0.3mm, <->, >={Latex[round]}] (0, -3.5) -- (0, 5.5);

\draw [black, line width=0.3mm] (4,2) circle (2);

\fill [fill=black] (4,2) circle (4pt);

\node[anchor=south, inner sep=2pt, rotate=0] (x-label) at (7.5,0) {$ x$};  

\node[anchor=west, inner sep=2pt, rotate=0] (y-label) at (0,5.5) {$ y$};  

\end{tikzpicture} 
%2
\item \begin{tikzpicture}[scale=0.3]

\fill [fill=\drawfill] (2,1) circle (2);

\draw [line width=0.1mm] (-1,-3) grid (7,5);
 
\draw[line width=0.3mm, <->, >={Latex[round]}] (-1.5, 0) -- (7.5, 0);

\draw[line width=0.3mm, <->, >={Latex[round]}] (0, -3.5) -- (0, 5.5);

\draw [black, line width=0.3mm] (2,1) circle (2);

\fill [fill=black] (2,1) circle (4pt);

\node[anchor=south, inner sep=2pt, rotate=0] (x-label) at (7.5,0) {$ x$};  

\node[anchor=west, inner sep=2pt, rotate=0] (y-label) at (0,5.5) {$ y$};  

\end{tikzpicture} 
%3
\item \begin{tikzpicture}[scale=0.3]

\fill [fill=\drawfill] (0,0) circle (2);

\draw [line width=0.1mm] (-4,-4) grid (4,4);
 
\draw[line width=0.3mm, <->, >={Latex[round]}] (-4.5, 0) -- (4.5, 0);

\draw[line width=0.3mm, <->, >={Latex[round]}] (0, -4.5) -- (0, 4.5);

\draw [black, line width=0.3mm] (0,0) circle (2);

\fill [fill=black] (0,0) circle (4pt);

\node[anchor=south, inner sep=2pt, rotate=0] (x-label) at (4.5,0) {$ x$};  

\node[anchor=west, inner sep=2pt, rotate=0] (y-label) at (0,4.5) {$ y$};  

\end{tikzpicture} 
%4
\item \begin{tikzpicture}[scale=0.3]

\fill [fill=\drawfill] (3,3) circle (1);

\draw [line width=0.1mm] (-1,-1) grid (7,7);
 
\draw[line width=0.3mm, <->, >={Latex[round]}] (-1.5, 0) -- (7.5, 0);

\draw[line width=0.3mm, <->, >={Latex[round]}] (0, -1.5) -- (0, 7.5);

\draw [black, line width=0.3mm] (3,3) circle (1);

\fill [fill=black] (3,3) circle (4pt);

\node[anchor=south, inner sep=2pt, rotate=0] (x-label) at (7.5,0) {$ x$};  

\node[anchor=west, inner sep=2pt, rotate=0] (y-label) at (0,7.5) {$ y$};  

\end{tikzpicture} 
%5 
\item \begin{tikzpicture}[scale=0.3]

\fill [fill=\drawfill] (0,1) circle (3);

\draw [line width=0.1mm] (-4,-3) grid (4,5);
 
\draw[line width=0.3mm, <->, >={Latex[round]}] (-4.5, 0) -- (4.5, 0);

\draw[line width=0.3mm, <->, >={Latex[round]}] (0, -3.5) -- (0, 5.5);

\draw [black, line width=0.3mm] (0,1) circle (3);

\fill [fill=black] (0,1) circle (4pt);

\node[anchor=south, inner sep=2pt, rotate=0] (x-label) at (4.5,0) {$ x$};  

\node[anchor=west, inner sep=2pt, rotate=0] (y-label) at (0,5.5) {$ y$};  

\end{tikzpicture} 
%6
\item \begin{tikzpicture}[scale=0.3]

\fill [fill=\drawfill] (-2,1) circle (3);

\draw [line width=0.1mm] (-6,-3) grid (2,5);
 
\draw[line width=0.3mm, <->, >={Latex[round]}] (-6.5, 0) -- (2.5, 0);

\draw[line width=0.3mm, <->, >={Latex[round]}] (0, -3.5) -- (0, 5.5);

\draw [black, line width=0.3mm] (-2,1) circle (3);

\fill [fill=black] (-2,1) circle (4pt);

\node[anchor=south, inner sep=2pt, rotate=0] (x-label) at (2.5,0) {$ x$};  

\node[anchor=west, inner sep=2pt, rotate=0] (y-label) at (0,5.5) {$ y$};  

\end{tikzpicture} 
\end{multicols} 
\end{enumerate}  
%B
%\item \hspce 
%\begin{center}
\vspace*{1ex}
\scalebox{0.8}{
\noindent\begin{minipage}{\textwidth}
{

B. Find the center and radius of each circle with the given equation. 
\begin{enumerate}[label = \arabic*. ]
\begin{multicols}{2}

%1
\item \hspce $2x^2 + 2y^2 =32 $
%2
\item \hspce $x^2 + (y+5)^2 =100 $
%3
\item \hspce $(x-5)^2 + y^2 =169 $
%4
\item \hspce $(x+2)^2 + (y-4)^2 =36 $
%5 
%\item \hspce $(x-3)^2 + (y+2)^2 =16 $

\end{multicols} 
\end{enumerate}  
%C
%\item \hspce 
%end{enumerate}  
C. Write an equation of circle $C$ based on the given information. 
\begin{enumerate}[label = \arabic*. ]
%1
\item Center at (5, 4) and touching the x-axis 
%2
\item Center at (10, 4) and passing through (2, 2)
%3
\item Center at (3, 8) and passing through the origin 
%4
\item Center at (2, 5) and tangent to the y-axis
%5 
\item A circle with area 49 $\text{cm}^2 $ and center at (2, 5)
\end{enumerate}   

}
\end{minipage}}
%\end{center} 
