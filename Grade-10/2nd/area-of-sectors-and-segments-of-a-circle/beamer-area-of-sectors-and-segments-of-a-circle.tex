% cd /storage/emulated/0/Documents/documents/latex/1920/Grade-10/2nd/area-of-sectors-and-segments-of-a-circle && pdflatex beamer-area-of-sectors-and-segments-of-a-circle.tex && termux-open beamer-area-of-sectors-and-segments-of-a-circle.pdf


\documentclass{beamer} 
\usetheme{Darmstadt}

\usepackage{xcolor}
\usepackage{anyfontsize}
\usepackage{enumitem}
\usepackage{multicol}
\usepackage{amsmath, makecell}
\usepackage{tabularx} 
\usepackage{gensymb}
\usepackage{wasysym} %for checked symbol 
\usepackage{multirow}
\usepackage{graphicx, tipa}
\usepackage{tikz}
\usetikzlibrary{angles,quotes}
\usepackage{pgfplots} 
\usetikzlibrary{calc}
\pgfplotsset{compat=newest}
\usetikzlibrary{arrows.meta}
\usetikzlibrary{intersections}
\usetikzlibrary{decorations.pathreplacing}
\usepackage{flafter}
%\usepackage{fourier} 
\usepackage{amsmath,amssymb,cancel,units}
\usepackage{microtype} % nicer output 
\usepackage{hfoldsty} % nicer output 
\usepackage{fixltx2e} 
\usepackage{mathptmx}
\usepackage{numprint}
\usepackage[T1]{fontenc}
\usepackage[utf8]{inputenc} 
\usepackage{stackengine} %to define \pesos 
\usepackage{lmodern} %scalable font
\usepackage{booktabs}
\usepackage{array}


\pagenumbering{gobble}
%\linespread{0.9}
\newcommand{\vspce}{\vspace{0.75ex}}

\newcommand{\hspce}{\hspace{0.5em}}

\newcommand{\blank}{\underline{\hspace{2em}}}%{\rule{1em}{0.15ex}}

\newcommand{\arc}[1]{{% 
\setbox9=\hbox{#1}% 
\ooalign{\resizebox{\wd9}{\height}{\texttoptiebar{\phantom{A}}}\cr#1}}}


\newcommand\pesos{\stackengine{-1.4ex}{P}{\stackengine{-1.25ex}{$-$}{$-$}{O}{c}{F}{F}{S}}{O}{c}{F}{T}{S}} 


\renewcommand\theadalign{bc} 

\renewcommand\theadfont{\bfseries} 

\renewcommand\theadgape{\Gape[4pt]} 

\renewcommand\cellgape{\Gape[4pt]} 

\pagenumbering{gobble}

\newcolumntype{Y}{>{\centering\arraybackslash}X} %for tabularx

\newcolumntype{R}{>{\raggedleft\arraybackslash}X} %for tabularx

\newcolumntype{Z}{>{\raggedleft\arraybackslash}X} %for tabularx

\newcolumntype{L}{>{\raggedright\arraybackslash}X} %for tabularx

\newcolumntype{A}[1]{>{\raggedright\arraybackslash}p{#1}} %for longtable LEFT

\newcolumntype{C}[1]{>{\centering\arraybackslash}p{#1}} %for longtable CENTER

\newcolumntype{B}[1]{>{\raggedleft\arraybackslash}p{#1}} %for longtable RIGHT 
 
\renewcommand{\tabularxcolumn}[1]{>{\small}m{#1}}

\newcolumntype{N}[1]{>{\raggedleft}p{#1}} %for tabular left 

\newcolumntype{M}[1]{>{\raggedright\arraybackslash}p{#1}} %for tabular right 

\newcommand{\myaxis}{xticklabels={}, 
yticklabels={}, 
ymin=-10, ymax=10,
xmin=-10, xmax=10,
axis lines = center, 
inner axis line style={Latex-Latex,very thick}, 
grid=both,
minor tick num=4, 
tick align=inside} % grid without labels, origin at the center, 10 units from origin

\newcommand{\axisfive}{xticklabels={}, 
yticklabels={}, 
ymin=-5, ymax=5,
xmin=-5, xmax=5,
axis lines = center, 
inner axis line style={Latex-Latex,very thick}, 
grid=both,
minor tick num=1, 
tick align=inside} % grid with labels, origin at the center, 5 units from origin 

\newcommand \redcheck {{\color{red}\checkmark}}



\title{Area of Sectors and Segments of a Circle} 
\author{Dr. Loreto G. Domingo\\
Noemi Sangria\\
Ferdinand Tala\\
Marie Carvy Hazel Galinato\\
Joanne A. Abia\\
Joshua Catungal\\
Jonathan R. Bacolod} 
\institute{Sauyo High School} 
\date{\today}

\def\figurefill{blue!20}

\begin{document} 

\begin{frame} 
\titlepage 
\end{frame}

%\begin{frame} 
%\frametitle{Outline} 
%\tableofcontents 
%\end{frame}  

\begin{frame} 
\section{Sector of a Circle} 
\frametitle{Sector of a Circle} 
\textbf{Sector:} a region in the circle bounded by two radii and the minor arc they determine

\vspce 

The \textbf{area of a sector} is represented by $A = \dfrac{n}{360}\pi r^2$, where $n$ is the number of degrees in the central angle of a sector.
\begin{center}
\input{/storage/emulated/0/Documents/documents/latex/1920/Grade-10/2nd/area-of-sectors-and-segments-of-a-circle/fig-a/fig-area-of-sectors-and-segments-of-a-circle-c1} 
\end{center} 
\end{frame}

\begin{frame} 
\subsection{Example}
\frametitle{Example} 
\vspace*{-2cm}
To solve for the area of the shaded region: \\
$n=90 \degree $\\
$\text{r = 5 in.}$\\
$A = \dfrac{n}{360}\pi r^2$\\
$A = \dfrac{90}{360}(\pi)(5^2)$\\
$A = \dfrac{1}{4}(\pi)(25)$\\
$A=\dfrac{25}{4}\pi$ in$^2$

\vspace*{-2.5cm}\hspace*{7.5cm}\def\radA{1cm}
\begin{tikzpicture}[remember picture, overlay,baseline = (current bounding box.west)]

\node[draw, circle, minimum size=2*\radA, inner sep=0pt, line width=0.5mm, outer sep=0] (circ) at (0,0) {};

\node[anchor=north, inner sep=2pt, rotate=0] (8cm.label) at ($(circ.center)!0.5!(180:\radA)$) {$ \text{5 in}$};

\node[anchor=south east, inner sep=2pt, rotate=0] (90.label) at ($(circ.center)+(135:\radA)$) {$ 90 \degree $};

\filldraw[fill=\figurefill, line width=0.3mm]  (90:\radA) arc (90:180:\radA) -- (circ.center) -- cycle;

\end{tikzpicture} 
 
\end{frame}

\section{Segment of a Circle} 
\begin{frame} 
\frametitle{Segment of a Circle} 
\textbf{Segment:} a region bounded by an arc and the chord of the arc

\vspce 

The \textbf{area of a segment} of a circle is found by subtracting the area of a triangle from the area of a sector. 
\[
\text A_{segment} = A_{sector}-A_{triangle}
\] 
\begin{center}
\def\radA{1cm}
\begin{tikzpicture}[baseline = (current bounding box.west)]


\node[draw, circle, minimum size=2*\radA, inner sep=0pt, line width=0.5mm, outer sep=0] (circ) at (0,0) {};

\filldraw[fill=\figurefill, line width=0.3mm]  (90:\radA) arc (90:180:\radA) -- cycle;

\draw[line width=0.3mm] (90:\radA) -- (180:\radA) -- (circ.center) -- cycle;


\end{tikzpicture} 
 
\end{center} 
\end{frame}

\begin{frame} 
\subsection{Example}
\frametitle{Example} 
\vspace*{-3cm}
To solve for the area of the shaded region: \\
%\begin{multicols}{2}
$n=90 \degree $, 
$\text{r = 5 in.}$\\
%\end{multicols} 
\begin{multicols}{3}
\small{$A_{sector} = \dfrac{n}{360}\pi r^2$\\
$A_{sector} = \dfrac{90}{360}(\pi)(5^2)$\\
$A_{sector} = \dfrac{1}{4}(\pi)(25)$\\
$A_{sector}=\dfrac{25}{4}\pi$ in$^2$\\
\columnbreak
$A_{triangle}=\dfrac{1}{2}bh$\\
$A_{triangle}=\dfrac{1}{2}(5)(5)$\\
$A_{triangle}=\dfrac{25}{2}$ in$^2$\\
\columnbreak
\tiny $A_{segment} = A_{sector}-A_{triangle}$\\
$A_{segment} = \dfrac{25}{4}\pi-\dfrac{25}{2}$\\
$A_{segment} = -\dfrac{25}{2}+\dfrac{25}{4}\pi$ in$^2$}
\end{multicols} 

\vspace*{-4.6cm}\hspace*{8.5cm}
\input{/storage/emulated/0/Documents/documents/latex/1920/Grade-10/2nd/area-of-sectors-and-segments-of-a-circle/fig-a/fig-area-of-sectors-and-segments-of-a-circle-d2} 

\end{frame}

\section{Practice Exercises} 
\begin{frame} 
\frametitle{Practice Exercises} 
\def\curdir{/storage/emulated/0/Documents/documents/latex/1920/Grade-10/2nd/area-of-sectors-and-segments-of-a-circle/fig-a}

\def\radA{1cm}

%\textbf{Practice Exercises}

\vspce

Find the area of each shaded region/s in each figure.  Express your answer in terms of $\pi$. 
\begin{center}
\vspace*{-2ex}
\scalebox{1}{
\noindent\begin{minipage}{\textwidth}
{\begin{tabularx}{\textwidth}{XX}
1. \begin{tikzpicture}[baseline = (current bounding box.west)]


\node[draw, circle, minimum size=2*\radA, inner sep=0pt, line width=0.5mm, outer sep=0] (circ) at (0,0) {};

\node[anchor=north, inner sep=2pt, rotate=0] (8cm.label) at ($(circ.center)!0.5!(0:\radA)$) {$ \text{8 cm}$};

\node[anchor=south west, inner sep=2pt, rotate=0] (90.label) at ($(circ.center)+(45:\radA)$) {$ 90 \degree $};

\filldraw[fill=\figurefill, line width=0.3mm]  (0:\radA) arc (0:90:\radA) -- (circ.center) -- cycle;

\end{tikzpicture} 
 & 3. \begin{tikzpicture}[baseline = (current bounding box.west)]



\node[draw, circle, minimum size=2*\radA, inner sep=0pt, line width=0.5mm, outer sep=0] (circ) at (0,0) {};

\node[anchor=south, inner sep=2pt, rotate=-90] (10cm.label) at ($(circ.center)!0.5!(90:\radA)$) {$ \text{10 cm}$};

\node[anchor=south east, inner sep=2pt, rotate=0] (120.label) at ($(circ.center)+(150:0.9*\radA)$) {$ 120 \degree $};

\filldraw[fill=\figurefill, line width=0.3mm]  (90:\radA) arc (90:210:\radA) -- (circ.center) -- cycle;

\end{tikzpicture} 

 \\
2. \begin{tikzpicture}[baseline = (current bounding box.west)]



\node[draw, circle, minimum size=2*\radA, inner sep=0pt, line width=0.5mm, outer sep=0] (circ) at (0,0) {};

\node[anchor=south, inner sep=2pt, rotate=-30] (12cm.label) at ($(circ.center)!0.5!(-30:\radA)$) {$ \text{12 cm}$};

\node[anchor=north west, inner sep=2pt, rotate=0] (60.label) at ($(circ.center)+(-60:\radA+0pt)$) {$ 60 \degree $};

\filldraw[fill=\figurefill, line width=0.3mm]  (-30:\radA) arc (-30:-90:\radA) -- (circ.center) -- cycle;

\end{tikzpicture} 

 & 4. \begin{tikzpicture}[baseline = (current bounding box.west)]



\node[draw, circle, minimum size=2*\radA, inner sep=0pt, line width=0.5mm, outer sep=0] (circ) at (0,0) {};

\node[anchor=south, inner sep=2pt, rotate=0] (16cm.label) at ($(circ.center)!0.5!(0:\radA)$) {$ \text{16 cm}$};

\filldraw[fill=\figurefill, line width=0.3mm]  (0:\radA) arc (0:-90:\radA) -- cycle;

\draw[line width=0.3mm]  (0:\radA) -- (circ.center) -- (-90:\radA);

\begin{scope} [rotate=-90]
\draw[line width=0.3mm] (circ.center) rectangle ++(0.15*\radA,0.15*\radA) node[transform shape]{};
\end{scope} 

\end{tikzpicture} 
 \\
\end{tabularx}}
\end{minipage}}
\end{center} 
   
\end{frame}

\begin{frame} 
\frametitle{Practice Exercises}
\def\curdir{/storage/emulated/0/Documents/documents/latex/1920/Grade-10/2nd/area-of-sectors-and-segments-of-a-circle/fig-a}

\def\radA{1cm}

\begin{center}
\vspace*{-2ex}
\scalebox{1}{
\noindent\begin{minipage}{\textwidth}
{\begin{tabularx}{\textwidth}{XX}
5. \begin{tikzpicture}[dot/.style={circle, fill=black, inner sep=0pt, outer sep=0pt, minimum size=3pt}, baseline = (current bounding box.west)]



\node[draw, circle, minimum size=2*\radA, inner sep=0pt, line width=0.5mm, outer sep=0] (circ) at (0,0) {};

\node[dot] (circ.dot)  at (circ) {}; 

\coordinate (a) at ($(circ) + (45:\radA)$); 

\coordinate (b) at ($(circ) - (45:\radA)$); 

\coordinate (c) at ($(circ) + (-45:\radA)$); 

\draw[line width=0.3mm] (a) -- (b) (b) -- (c) (c) -- (a) ;

\node[anchor=south, inner sep=2pt] (10cm1-label) at ($(b)!0.5!(c)$) {$ \text{10 cm}$};  

\filldraw[fill=\figurefill, line width=0.3mm]  (45:\radA) arc (45:-45:\radA) -- cycle;

\filldraw[fill=\figurefill, line width=0.3mm]  (-45:\radA) arc (-45:-135:\radA) -- cycle;

\node[anchor=south, inner sep=2pt, rotate=90] (10cm2-label) at ($(a)!0.5!(c)$) {$ \text{10 cm}$};  

\begin{scope} [rotate=90]
\draw[line width=0.3mm] (c) rectangle ++(0.15*\radA,0.15*\radA) node[transform shape]{};
\end{scope} 

\end{tikzpicture}  
 & 7. \begin{tikzpicture}[dot/.style={circle, fill=black, inner sep=0pt, outer sep=0pt, minimum size=0pt}, baseline = (current bounding box.west)]

\node[draw, circle, minimum size=2*\radA, inner sep=0pt, line width=0.5mm, outer sep=0] (circ) at (0,0) {};

\node[dot] (circ.dot)  at (circ) {}; 

\coordinate (a) at ($(circ) + (180:\radA)$); 

\coordinate (t) at ($(circ) + (90:\radA)$); 

\draw[line width=0.3mm] (a) -- (circ.center) -- (t) ;

\node[anchor=south, inner sep=2pt, rotate=0] (4.5cm-label) at ($(a)!0.5!(circ.center)$) {\tiny $ \text{4.5 cm} $};  

\filldraw[fill=\figurefill, line width=0.3mm]  (90:\radA) arc (90:-180:\radA) -- (circ.center) -- cycle;

\begin{scope} [rotate=90]
\draw[line width=0.3mm] (circ.center) rectangle ++(0.13*\radA,0.13*\radA) node[transform shape]{};
\end{scope} 

\end{tikzpicture}  
 \\
6. \begin{tikzpicture}[dot/.style={circle, fill=black, inner sep=0pt, outer sep=0pt, minimum size=3pt}, baseline = (current bounding box.west)]

\node[draw, circle, minimum size=2*\radA, inner sep=0pt, line width=0.5mm, outer sep=0] (circ) at (0,0) {};

\node[dot] (circ.dot)  at (circ) {}; 

\coordinate (a) at ($(circ) + (60:\radA)$); 

\coordinate (n) at ($(circ) + (0:\radA)$); 

\draw[line width=0.3mm] (a) -- (circ.center) -- (n) -- cycle;

\filldraw[fill=\figurefill, line width=0.3mm]  (0:\radA) arc (0:60:\radA) -- cycle;

\node[anchor=south, inner sep=2pt, rotate=60] (6cm1.label) at ($(circ.center)!0.5!(a)$) {$\text{6cm}$};

\node[anchor=south, inner sep=2pt, rotate=0] (60.label) at ($(circ.center)!0.45!(n)$) {$ 60 \degree $};

\end{tikzpicture}  
 & 8. \begin{tikzpicture}[baseline = (current bounding box.west)]

\node[draw, circle, minimum size=2*\radA, inner sep=0pt, line width=0.5mm, outer sep=0] (circ) at (0,0) {};

\node[anchor=south, inner sep=2pt, rotate=0] (8in.label) at ($(circ.center)!0.5!(0:\radA)$) {$ \text{8 in}$};

\filldraw[fill=\figurefill, line width=0.3mm]  (0:\radA) arc (0:-60:\radA) -- cycle;

\filldraw[fill=\figurefill, line width=0.3mm]  (180:\radA) arc (180:120:\radA) -- cycle;

\node[rotate=0] (120-label) at ($(circ.center)+(220:0.42*\radA)$) {$ 120\degree $}; 

\draw[line width=0.3mm] (0:\radA) -- (180:\radA) -- (120:\radA) -- (-60:\radA) -- cycle;

\end{tikzpicture} 

 \\
\end{tabularx}}
\end{minipage}}
\end{center} 
   
\end{frame}

\section{Problem Set} 
\begin{frame} 
\frametitle{Problem Set} 
\def\curdir{/storage/emulated/0/Documents/documents/latex/1920/Grade-10/2nd/area-of-sectors-and-segments-of-a-circle/fig-b}

\def\radB{1cm}

%\textbf{Problem Set}

\vspce

Find the area of each shaded region/s in each figure.  Express your answer in terms of $\pi$. 
\begin{center}
\vspace*{-2ex}
\scalebox{1}{
\noindent\begin{minipage}{\textwidth}
{\begin{tabularx}{\textwidth}{XX}
1. \begin{tikzpicture}[baseline = (current bounding box.west)]

\node[draw, circle, minimum size=2*\radB, inner sep=0pt, line width=0.5mm, outer sep=0] (circ) at (0,0) {};

\node[anchor=south, inner sep=2pt, rotate=-30] (12cm.label) at ($(circ.center)!0.5!(-30:\radB)$) {$ \text{12 cm}$};

\node[anchor=north west, inner sep=2pt, rotate=0] (60.label) at ($(circ.center)+(-60:\radB+0pt)$) {$ 60 \degree $};

\filldraw[fill=\figurefill, line width=0.3mm]  (-30:\radB) arc (-30:-90:\radB) -- cycle;

\draw[line width=0.3mm] (-30:\radB) -- (-90:\radB) -- (circ.center) -- cycle;

\end{tikzpicture} 

 & 3. \begin{tikzpicture}[dot/.style={circle, fill=black, inner sep=0pt, outer sep=0pt, minimum size=3pt}, baseline = (current bounding box.west)]

\node[draw, circle, minimum size=2*\radB, inner sep=0pt, line width=0.5mm, outer sep=0] (circ) at (0,0) {};

\node[dot] (circ.dot)  at (circ) {}; 

\coordinate (v) at ($(circ) + (180:\radB)$); 

\coordinate (c) at ($(circ) + (60:\radB)$); 

\draw[line width=0.3mm] (c) -- (circ.center) -- (v) ;

\node[anchor=north, inner sep=2pt, rotate=60] (10cm-label) at ($(circ.center)! 0.5!(c) $)  {$ \text{10 cm}$};  

\node[anchor=north, inner sep=2pt, rotate=0] (240-label) at (-40:1.22*\radB) {$ \text{240\degree } $}; 

\filldraw[fill=\figurefill, line width=0.3mm]  (60:\radB) arc (60:180:\radB) -- cycle;

\end{tikzpicture}  
 \\
2. \begin{tikzpicture}[dot/.style={circle, fill=black, inner sep=0pt, outer sep=0pt, minimum size=3pt}, baseline = (current bounding box.west)]

\node[draw, circle, minimum size=2*\radB, inner sep=0pt, line width=0.5mm, outer sep=0] (circ) at (0,0) {};

\node[dot] (circ.dot)  at (circ) {}; 

\coordinate (l) at ($(circ) + (45:\radB)$); 

\coordinate (e) at ($(circ) + (0:\radB)$); 


\draw[line width=0.3mm] (l) -- (circ.center) -- (e);

\node[anchor=north, inner sep=2pt, rotate=0] (9cm.label) at ($(circ.center)!0.5!(e)$) {$ \text{9 cm}$};

\node[anchor=west, inner sep=2pt, rotate=0] (45.label) at (20:1.1*\radB) {$ 45 \degree $};

\filldraw[fill=\figurefill, line width=0.3mm]  (0:\radB) arc (0:45:\radB) -- (circ.center) -- cycle;

\end{tikzpicture}  

 & 4. \begin{tikzpicture}[dot/.style={circle, fill=black, inner sep=0pt, outer sep=0pt, minimum size=3pt}, baseline = (current bounding box.west)]

\node[draw, circle, minimum size=2*\radB, inner sep=0pt, line width=0.5mm, outer sep=0] (circ) at (0,0) {};

\node[dot] (circ.dot)  at (circ) {}; 

\coordinate (u) at ($(circ) + (25:\radB)$); 

\coordinate (m) at ($(circ) + (155:\radB)$); 

\draw[line width=0.3mm] (u) -- (circ.center) -- (m) ;

\node[anchor=north, inner sep=2pt, rotate=25] (8cm-label) at ($(u)!0.5!(circ.center)$) {$ \text{8 cm} $};  

\node[rotate=0] (130-label) at ($(circ.center)+(80:1.22*\radB)$) {$ 130\degree $};

\filldraw[fill=\figurefill, line width=0.3mm]  (25:\radB) arc (25:155:\radB) --(circ.center) -- cycle;

\end{tikzpicture}  
 \\

\end{tabularx}}
\end{minipage}}
\end{center} 
   
\end{frame}

\begin{frame} 
\frametitle{Problem Set} 
\def\curdir{/storage/emulated/0/Documents/documents/latex/1920/Grade-10/2nd/area-of-sectors-and-segments-of-a-circle/fig-b}

\def\radB{1cm}

\begin{center}
\vspace*{-2ex}
\scalebox{1}{
\noindent\begin{minipage}{\textwidth}
{\begin{tabularx}{\textwidth}{XX}
5. \begin{tikzpicture}[baseline = (current bounding box.west)]


\node[draw, circle, minimum size=2*\radB, inner sep=0pt, line width=0.5mm, outer sep=0] (circ) at (0,0) {};

\node[anchor=south, inner sep=2pt, rotate=90] (18cm.label) at ($(circ.center)!0.5!(90:\radB)$) {\tiny $ \text{18 cm}$};
  
\filldraw[fill=\figurefill, line width=0.3mm]  (90:\radB) arc (90:-180:\radB) -- (circ.center) -- cycle;

\begin{scope} [rotate=90]
\draw[line width=0.3mm] (circ.center) rectangle ++(0.15*\radB,0.15*\radB) node[transform shape]{};
\end{scope} 

\end{tikzpicture} 
 & 7. \begin{tikzpicture}[dot/.style={circle, fill=black, inner sep=0pt, outer sep=0pt, minimum size=2pt}, dim-label/.style={fill=white, rectangle, inner sep=2pt, outer sep=0pt}, baseline = (current bounding box.west)]  

\node[draw, circle, minimum size=2*\radB, inner sep=0pt, line width=0.5mm, outer sep=0] (circ) at (0,0) {};

\filldraw[fill=\figurefill, line width=0.3mm]  (180:\radB) arc (180:0:\radB) -- cycle;

\coordinate(center2) at ($(180:\radB)! 0.5! (circ.center) $);

\filldraw[fill=white, line width=0.3mm] (center2) circle (0.5*\radB); 

\coordinate(center3) at ($(0:\radB)! 0.5! (circ.center) $);

\filldraw[fill=white, line width=0.3mm] (center3) circle (0.5*\radB); 

\node [dot] at (center2){}; 

\node [dot] at (center3){}; 


\draw[line width=0.3mm] (180:\radB) -- (0:\radB); 

\node[anchor=north, inner sep=2pt, rotate=0] (8cm1.label) at (center2) {$ \text{8 cm}$};

\node[anchor=north, inner sep=2pt, rotate=0] (8cm2.label) at (center3) {$ \text{8 cm}$};

\end{tikzpicture} 
 \\
6. \begin{tikzpicture}[baseline = (current bounding box.west)][dot/.style={circle, fill=black, inner sep=0pt, outer sep=0pt, minimum size=3pt}, dim-label/.style={fill=white, rectangle, inner sep=2pt, outer sep=0pt}%, remember picture, overlay
]  

\coordinate (center) at (0,0);

\filldraw[fill=\figurefill, name path=circ, line width=0.3mm] (center) circle (\radB); 

\filldraw[fill=white, line width=0.3mm] (center) circle (0.65*\radB); 

\node[anchor=west, inner sep=2pt, rotate=0] (5cm.label) at ($(center)!0.35!(250:\radB)$) {\tiny $ \text{5 cm}$};

\node[anchor=west, inner sep=2pt, rotate=0] (3cm.label) at ($(center)!0.87!(250:\radB)$) {\tiny $ \text{3 cm}$};

\draw[line width=0.3mm] (25:0.65*\radB) -- (center) --(250:\radB); 

\end{tikzpicture} 

 & 8. \begin{tikzpicture}[baseline = (current bounding box.west)]



\node[draw, circle, minimum size=2*\radB, inner sep=0pt, line width=0.5mm, outer sep=0] (circ) at (0,0) {};

\node[anchor=south, inner sep=2pt, rotate=0] (10in.label) at ($(circ.center)!0.5!(0:\radB)$) {$ \text{10 in}$};

\filldraw[fill=\figurefill, line width=0.3mm]  (0:\radB) arc (0:-60:\radB) -- (circ.center) -- cycle;

\filldraw[fill=\figurefill, line width=0.3mm]  (180:\radB) arc (180:120:\radB) -- (circ.center) -- cycle;

\node[rotate=0] (120-label) at ($(circ.center)+(220:0.42*\radB)$) {$ 120\degree $}; 


\end{tikzpicture} 

 \\
\end{tabularx}}
\end{minipage}}
\end{center} 
   
\end{frame}


\end{document}


