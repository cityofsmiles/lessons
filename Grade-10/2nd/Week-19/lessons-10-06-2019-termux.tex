 

\def \LessonDayA {Angles Formed by Secants and Tangents}

\def \LearningCompetenciesDayA {}

\def \ObjectivesDayA {
\item %cogverbstart
Describe
%cogverbend
the angles formed by secants and tangents; 
\item %psyverbstart
Find
%psyverbend
the angles formed by secants and tangents to determine whether a binomial is a factor of a given polynomial; and, 
\item %affverbstart
Show
%affverbend
%valsstart
independence and willingness
%valsend
in solving problems.
}

\def \PurposeDayA {The purpose of this lesson is to enable the students to solve real life problems involving the angles formed by secants and tangents.}  

\def \ApplicationDayA { Let the students answer the following questions: 
\begin{enumerate}[label = \arabic*. ]
%1
\item In what real life situations or problems can we observe some examples of angles formed by secants and tangents? 
%2
\item How can you apply your knowledge of angles formed by secants and tangents in solving these real life problems? 

\end{enumerate}   
}

\def \GeneralizationDayA {Let the students answer the following questions: 
\begin{enumerate}[label = \arabic*. ]
%1
\item In your own words, what is the angles formed by secants and tangents? 
%2
\item How do we solve problems involving angles formed by secants and tangents? 

\end{enumerate}   
}

\def \TeachersGuideDayA {pp. 141--144}

\def \LMPagesDayA {pp. 130--133}

\def \TextbookPagesDayA {pp. 140--144}

\def \AdditionalMaterialsDayA {}

\def \OtherResourcesDayA {Flashcards}

\def \ReviewDayA {\begin{center}
\textbf{Angles Formed by Secants and Tangents}
\end{center}

\vspace*{1ex}

\begin{center}
\vspace*{-2ex}
\scalebox{0.5}{
\noindent\begin{minipage}{0.3\textwidth}
{
\textbf{Intersecting Secants -- Exterior Theorem:} The measure of an angle formed by two secants that intersect in the exterior of a circle is one-half the difference of its intercepted arcs. 

\vspce 

\textbf{Tangent Point -- Secant Theorem:}  The measure of an angle formed by a tangent and a secant drawn at the point of contact is one-half the measure of its intercepted arc. 

\vspce 

\textbf{Intersecting Secants -- Interior  Theorem:} The measure of an angle formed by two secants intersecting in the interior of the circle is equal to one-half the sum of the measures of its intercepted arcs. 
}
\end{minipage}}
\end{center} 
 }



\def \ExamplesDayA {}

\def \PracticeOneDayA {\def\radA{1cm}
\def\curdir{/storage/emulated/0/Documents/documents/latex/1920/Grade-10/2nd/angles-formed-by-secants-and-tangents}

\textbf{Practice Exercises}

\vspce

Find the value of $x$. 

\begin{center}
\scalebox{0.5}{
\noindent\begin{minipage}{0.3\textwidth}
{\begin{tabularx}{\textwidth}{XX}
1. \begin{tikzpicture}[dot/.style={circle, fill=black, inner sep=0pt, outer sep=0pt, minimum size=3pt}, baseline = (current bounding box.west)]

\node[draw,circle,minimum size=2*\radA, inner sep=0pt, line width=0.5mm, outer sep=0] (circ) at (0,0) {};

\coordinate (a) at ($(circ) + (120:\radA)$);

\coordinate (b) at ($(circ) + (-60:\radA)$);

\coordinate (c) at ($(circ) + (70:\radA)$);

\coordinate (d) at ($(circ) + (200:\radA)$);

\draw[name path=line1, line width=0.3mm] (a)--(b);

\draw[name path=line2, line width=0.3mm] (c)--(d);

\fill[black, name intersections={of=line1 and line2, name=point}];

\node[dot] (x) at (point-1){};

\node(37-label) at ($(x)+(80: 0.85*\radA)$) {$  37\degree$};

\node(x.label) at ($(x)+(85:0.22*\radA)$) {$  x$};

\node(105.label) at ($(circ)+(-110:1.3*\radA)$) {$  105\degree $};

\end{tikzpicture} 
 & 5. \begin{tikzpicture}[dot/.style={circle, fill=black, inner sep=0pt, outer sep=0pt, minimum size=3pt}, baseline = (current bounding box.west)]

\node[draw,circle,minimum size=2*\radA, inner sep=0pt, line width=0.5mm, outer sep=0] (circ) at (0,0) {};

\coordinate (a) at ($(circ) + (50: 1.4*\radA)$);

\coordinate (b) at ($(circ) + (215: 1.8*\radA)$);

\coordinate (c) at ($(circ) + (-25: 1.4*\radA)$);

\node(x.label) at ($(b)+(23:0.44*\radA)$) {$  x$};

\node(80.label) at ($(circ)+(5:0.7*\radA)$) {$  80\degree $};

\node(140.label) at ($(circ)+(130:1.25*\radA)$) {$  140\degree $};

\node(130.label) at ($(circ)+(-80:1.2*\radA)$) {$  130\degree $};

\draw[line width=0.3mm, <->, >={Latex[round]}] (a) -- (b) -- (c) ;

\end{tikzpicture} 

 \\
 
2. \begin{tikzpicture}[dot/.style={circle, fill=black, inner sep=0pt, outer sep=0pt, minimum size=3pt}, baseline = (current bounding box.west)]

\node[draw,circle,minimum size=2*\radA, inner sep=0pt, line width=0.5mm, outer sep=0] (circ) at (0,0) {};

\coordinate (a) at ($(circ) + (\radA, 0)$);

\coordinate (b) at ($(circ) + (110:\radA)$);

\coordinate (c) at ($(a) + (0, -\radA)$);

\pic [draw, line width=0.3mm, angle radius=0.4*\radA, "$x$"] {angle=b--a--c};

\draw[line width=0.3mm] (b)--(a);

\draw[line width=0.3mm, <->, >={Latex[round]}] (c)--($(a)!-1!(c) $);

\node(250-label) at ($(circ)+(-120:1.35*\radA)$) {$  250\degree $};

\end{tikzpicture} 
 & 6. \begin{tikzpicture}[dot/.style={circle, fill=black, inner sep=0pt, outer sep=0pt, minimum size=3pt}, baseline = (current bounding box.west)]

\node[draw,circle,minimum size=2*\radA, inner sep=0pt, line width=0.5mm, outer sep=0] (circ) at (0,0) {};

\coordinate (a) at ($(circ) + (150: 1.4*\radA)$);

\coordinate (b) at ($(circ) + (-17: 1.4*\radA)$);

\coordinate (c) at ($(circ) + (27: 1.4*\radA)$);

\coordinate (d) at ($(circ) + (200: 1.4*\radA)$);

\draw[name path=line1, line width=0.3mm, <->, >={Latex[round]}] (a) -- (b);

\draw[name path=line2, line width=0.3mm, <->, >={Latex[round]}] (c) -- (d);

\fill[black, name intersections={of=line1 and line2, name=point}];

\node[dot] (int) at (point-1){};

\node(x.label) at ($(int)+(0:0.35*\radA)$) {$  x$};

\node(40-label) at ($(int)+(0:1.2*\radA)$) {$  40\degree $};

\node(60-label) at ($(int)+(180:1.4*\radA)$) {$  60\degree $};

\end{tikzpicture} 
 \\

3. \begin{tikzpicture}[dot/.style={circle, fill=black, inner sep=0pt, outer sep=0pt, minimum size=3pt}, baseline = (current bounding box.west)]

\node[draw,circle,minimum size=2*\radA, inner sep=0pt, line width=0.5mm, outer sep=0] (circ) at (0,0) {};

\coordinate (a) at ($(circ) + (35:1.4*\radA)$);

\coordinate (b) at ($(circ) + (180:2*\radA)$);

\coordinate (c) at ($(circ) + (-30:1.4*\radA)$);

\node(x.label) at ($(b)+(0:0.5*\radA)$) {$  x$};

\node(24.label) at ($(circ)+(180:0.65*\radA)$) {$ 24\degree $};

\node(118.label) at ($(circ)+(0:0.6*\radA)$) {$ 118\degree $};

\draw[line width=0.3mm, <->, >={Latex[round]}] (a) -- (b) -- (c) ;

\end{tikzpicture} 

 & 7. \begin{tikzpicture}[dot/.style={circle, fill=black, inner sep=0pt, outer sep=0pt, minimum size=3pt}, baseline = (current bounding box.west)
]

\node[draw,circle,minimum size=2*\radA, inner sep=0pt, line width=0.5mm, outer sep=0] (circ) at (0,0) {};

\coordinate (a) at ($(circ) + (120: \radA)$);

\coordinate (b) at ($(circ) + (290: \radA)$);

\coordinate (c) at ($(circ) + (40: \radA)$);

\coordinate (d) at ($(circ) + (240: \radA)$);

\node(78.label) at ($(circ)+(80:1.25*\radA)$) {$  78\degree $};

\node(28.label) at ($(circ)-(87:1.25*\radA)$) {$  28\degree $};

\draw[line width=0.3mm, name path=line1] (a) -- (b);

\draw[line width=0.3mm, name path=line2] (c) -- (d);

\fill[black, name intersections={of=line1 and line2, name=point}];

\node[dot] (int) at (point-1){};

\node(x.label) at ($(int)+(160:0.6*\radA)$) {\tiny $\text{x+53}\degree$};

\pic [draw, line width=0.2mm, angle radius=0.25*\radA] {angle=a--int--d};


\end{tikzpicture} 


 \\

4. \begin{tikzpicture}[dot/.style={circle, fill=black, inner sep=0pt, outer sep=0pt, minimum size=3pt}, baseline = (current bounding box.west)]

\node[draw,circle,minimum size=2*\radA, inner sep=0pt, line width=0.5mm, outer sep=0] (circ) at (0,0) {};

\coordinate (ext) at ($(circ) + (-90:1.7*\radA)$);

\node(x.label) at ($(ext)+(90:0.22*\radA)$) {$x$};

\node[dot] (tan1) at (tangent cs:node=circ, point={(ext)}, solution=1) {}; 

\node[dot] (tan2) at (tangent cs:node=circ, point={(ext)}, solution=2) {}; 

\node(135.label) at ($(circ)+(-90:0.75*\radA) $) {$135\degree  $};

\draw[line width=0.3mm, <->, >={Latex[round]}] ($(tan2)!-0.6!(ext) $) -- (ext) -- ($(tan1)!-0.6!(ext) $) ;

\end{tikzpicture} 

 & 8. \begin{tikzpicture}[dot/.style={circle, fill=black, inner sep=0pt, outer sep=0pt, minimum size=3pt}, baseline = (current bounding box.west)]

\node[draw,circle,minimum size=2*\radA, inner sep=0pt, line width=0.5mm, outer sep=0] (circ) at (0,0) {};

\coordinate (a) at ($(circ) + (120: \radA)$);

\coordinate (b) at ($(circ) + (-120: \radA)$);

\coordinate (c) at ($(circ) + (200: \radA)$);

\coordinate (d) at ($(circ) + (-20: \radA)$);

\node(1-label) at ($(circ)+(50:1.3*\radA)$) {$50x$};

\node(2-label) at ($(circ)+(210:1.55*\radA)$) {$  \text{2x+24}\degree $};

\draw[line width=0.3mm, name path=line1] (a) -- (b);

\draw[line width=0.3mm, name path=line2] (c) -- (d);

\fill[black, name intersections={of=line1 and line2, name=point}];

\node[dot] (int) at (point-1){};

\begin{scope} [rotate=0]
\draw[line width=0.3mm] (int) rectangle ++(0.22*\radA,0.22*\radA) node[transform shape]{};
\end{scope}

\end{tikzpicture} 

 \\

\end{tabularx}}
\end{minipage}}
\end{center} 
  }

\def \PracticeTwoDayA {\input{/storage/emulated/0/Documents/documents/latex/1920/Grade-10/2nd/angles-formed-by-secants-and-tangents/ps-angles-formed-by-secants-and-tangents-input2-dll}
}

\def \MasteryDayA {\def\curdir{/storage/emulated/0/Documents/documents/latex/1920/Grade-10/2nd/angles-formed-by-secants-and-tangents}
\def\radA{1cm}
\def\radB{1cm}

\textbf{Problem Set}

\vspce 

%\begin{enumerate}[label = \Alph*. ]
%\item \hspce 
A. Use the figure to solve the following. 
\begin{center}
\vspace*{2ex}
\scalebox{0.5}{
\noindent\begin{minipage}{0.3\textwidth}
{
\begin{enumerate}[label = \arabic*. ]
%1
\item If $m\arc{AC}=40\degree $ and $m\arc{BD}=80\degree $, find $m\angle{ATC} $. 
%2
\item If $m\angle{BTC} =142\degree $ and $m\arc{AD}=156\degree $, find $m\arc{BC}$. 
%3
\item If $m\arc{ADB} =208\degree $, $m\arc{AC} =52\degree $ and $m\arc{DBC}=192\degree $, find $m\angle{ATD}$.

\begin{center}

\hspace*{-40pt}\begin{tikzpicture}[dot/.style={circle, fill=black, inner sep=0pt, outer sep=0pt, minimum size=3pt}]

\node(circ) at (0,0) {};

\draw[line width=0.5mm] (circ) circle (\radA);

\coordinate (d) at ($(circ) + (50:\radA)$); 

\coordinate (c) at ($(circ) + (220:\radA)$);

\coordinate (a) at ($(circ) + (175:\radA)$);

\coordinate (b) at ($(circ) + (-40:\radA)$);

\draw[name path=line1, line width=0.3mm, <->, >={Latex[round]}] ($(a)!-35pt!(b) $) -- ($(b)!-0.66*\radA!(a) $);

\draw[name path=line2, line width=0.3mm, <->, >={Latex[round]}] ($(c)!-35pt!(d) $) -- ($(d)!-0.66*\radA!(c) $);

\fill[black, name intersections={of=line1 and line2, name=point}];

\node[dot] (int) at (point-1){};

\node(t.label) at ($(int)+(90:0.22*\radA)$) {$  T$};

\node(d.label) at ($(d)+(90:0.25*\radA)$) {$  D$};

\node(c.label) at ($(c)+(-100:0.25*\radA)$) {$  C$};

\node(a.label) at ($(a)+(120:0.33*\radA)$) {$  A$};

\node(b.label) at ($(b)+(20:0.33*\radA)$) {$  B$};

\end{tikzpicture} 
 

\end{center} 

\end{enumerate} 
}
\end{minipage}}
\end{center} 

%\item \hspce
B. Solve each problem completely.  
\begin{center}
\vspace*{2ex}
\scalebox{0.5}{
\noindent\begin{minipage}{0.3\textwidth}
{
\begin{enumerate}[label = \arabic*. ]
%1
\item The angle formed by two secants intersecting in the exterior of a circle measures $64\degree $. One of the intercepted arcs is $208\degree $. Find the other arc. 
%2
\item In $\odot{Q} $, the endpoints of chord $\overline{OK} $ are the points of tangency of lines $\overleftrightarrow{TO} $ and $\overleftrightarrow{TK}$. If $\triangle{TOK}$ is an equilateral triangle, find $m\arc{OIK}$. 
%3
\item In $\odot{O} $, $\overline{LM} $ is a diameter of $\odot{O} $ where $\overline{OM} $ extends its own length to $P$. If $\overline{PN} $ is a tangent segment to $\odot{O} $ at $N$, find $m\angle{P} $. 

\begin{center}
\begin{tikzpicture}[dot/.style={circle, fill=black, inner sep=0pt, outer sep=0pt, minimum size=3pt}%, remember picture, overlay
]

\node[draw, circle, minimum size=2*\radB,inner sep=0pt, line width=0.5mm, outer sep=0] (q) at (0,0) {};

\node(q.dot)[dot] at (q.center) {};

\node(q.label) at ($(q)+(190:0.22*\radB)$) {$  Q$};

\coordinate (t) at ($(q) + (0:2.2*\radB)$); 

\node(t.label) at ($(t)+(0:0.22*\radB)$) {$  T$};

\node[dot] (i) at ($(q)+(230:\radB)$) {};

\node(i.label) at ($(i)+(230:0.22*\radB)$) {$  I$};

\node[dot] (k) at (tangent cs:node=q, point={(t)}, solution=1) {};

\node(k.label) at ($(k)+(90:0.22*\radB)$) {$  K$};

\node[dot] (o) at (tangent cs:node=q, point={(t)}, solution=2) {};

\node(o.label) at ($(o)+(280:0.22*\radB)$) {$  O$};

\draw[line width=0.3mm, <->, >={Latex[round]}] ($(k)!-\radB!(t)$) -- (t) -- ($(o)!-\radB!(t)$);

\draw[line width=0.3mm] (k) -- (o) ;

\end{tikzpicture}  
 

%\hspace*{-30pt}
\begin{tikzpicture}[dot/.style={circle, fill=black, inner sep=0pt, outer sep=0pt, minimum size=3pt}]

\node[draw, circle, minimum size=2*\radB, inner sep=0pt, line width=0.5mm, outer sep=0] (o) at (0,0) {};

\node(o.dot)[dot] at (o.center) {};

\node(o.label) at ($(o)+(270:0.22*\radB)$) {$  O$};

\coordinate (l) at ($(o) + (-\radB, 0)$); 

\node(l.label) at ($(l)+(180:0.18*\radB)$) {$  L$};

\coordinate (p) at ($(o) + (0:2*\radB)$); 

\node(p.label) at ($(p)+(0:0.22*\radB)$) {$  P$};

\node(m.label) at ($(o)+(0:\radB)+(-40:0.3*\radB)$) {$  M$};

\node[dot] (n) at (tangent cs:node=o, point={(p)}, solution=1) {};

\node(n.label) at ($(n)+(60:0.22*\radB)$) {$  N$};

\draw[line width=0.3mm] (l) -- (p) -- (n) -- (o.center) ;

\begin{scope} [rotate=-120]
\draw[line width=0.3mm] (n) rectangle ++(0.35*\radB,0.35*\radB) node[transform shape]{};
\end{scope}

\end{tikzpicture} 

 

\end{center} 

\end{enumerate} 
}
\end{minipage}}
\end{center} 
%\end{enumerate} }



\def \EvaluationDayA {
\input{/storage/emulated/0/Documents/documents/latex/1920/Grade-10/2nd/angles-formed-by-secants-and-tangents/qz-angles-formed-by-secants-and-tangents-input} 
}

\def \RemediationDayA {}

\def \LessonDayB {Power Theorems}

\def \LearningCompetenciesDayB {}

\def \ObjectivesDayB {
\item %cogverbstart
Perform
%cogverbend
the power theorems; 
\item %psyverbstart
Find
%psyverbend
the power theorems to determine whether a binomial is a factor of a given polynomial; and, 
\item %affverbstart
Exhibit
%affverbend
%valsstart
determination and independence
%valsend
in solving problems.
}

\def \PurposeDayB {The purpose of this lesson is to enable the students to solve real life problems involving the power theorems.}  

\def \ApplicationDayB { Let the students answer the following questions: 
\begin{enumerate}[label = \arabic*. ]
%1
\item In what real life situations or problems can we observe some examples of power theorems? 
%2
\item How can you apply your knowledge of power theorems in solving these real life problems? 

\end{enumerate}   
}

\def \GeneralizationDayB {Let the students answer the following questions: 
\begin{enumerate}[label = \arabic*. ]
%1
\item In your own words, what is the power theorems? 
%2
\item How do we solve problems involving power theorems? 

\end{enumerate}   
}

\def \TeachersGuideDayB {pp. 145--150}

\def \LMPagesDayB {pp. 134--139}

\def \TextbookPagesDayB {pp. 145--151}

\def \AdditionalMaterialsDayB {}

\def \OtherResourcesDayB {Flashcards}

\def \ReviewDayB {\begin{center}
\textbf{Power Theorems}
\end{center}

\vspace*{1ex}

\begin{center}
\vspace*{-2ex}
\scalebox{0.5}{
\noindent\begin{minipage}{0.3\textwidth}
{
\textbf{Intersecting Segments of Chords Power Theorem:} If two chords intersect in the interior of a circle, then the product of the lengths of the segments of one chord is equal to the product of the lengths of the segments of the other chord. 

\vspce 

\textbf{Segments of Secants Power Theorem:} If two secants intersect in the exterior of a circle, the product of the length of one secant segment and the length of its external part is equal to the product of the length of the other secant segment and the length of its external part. 

\vspce 

\textbf{Tangent Secant Segments Power Theorem:}  If a tangent segment and a secant intersect in the exterior of a circle, then the square of the length of the tangent segment is equal to the product of the lengths of the secant segment and its external part. 
}
\end{minipage}}
\end{center}  }



\def \ExamplesDayB {}

\def \PracticeOneDayB {\def\curdir{/storage/emulated/0/Documents/documents/latex/1920/Grade-10/2nd/power-theorems}
\def\radA{1cm}

\textbf{Practice Exercises}

\vspce

Find the value of $x$. 
\begin{center}
%\vspace*{-2ex}
\scalebox{0.5}{
\noindent\begin{minipage}{0.3\textwidth}
{\begin{tabularx}{\textwidth}{XX}
1. \begin{tikzpicture}[dot/.style={circle, fill=black, inner sep=0pt, outer sep=0pt, minimum size=3pt}, baseline = (current bounding box.west)]

\coordinate (circ) at (0,0); 

\draw[name path=circum, line width=0.5mm] (circ) circle (\radA);

\coordinate (a) at ($(circ) + (50:\radA)$); 

\coordinate (b) at ($(circ) + (175:1.8*\radA)$); 

\coordinate (c) at ($(circ) + (-60:\radA)$); 

\draw[name path=line1, line width=0.3mm] (a) -- (b) -- (c) ;

\fill[black, name intersections={of=line1 and circum, name=int}];

\node(6.label) at ($(int-2)+(170:0.45*\radA)$) {$  6$};

\node(8.label) at ($(int-3)+(190:0.45*\radA)$) {$  8$};

\node(x-label) at ($(a)!0.5!(int-2) +(100:0.17*\radA)$) {$  x$}; 

\node(14-label) at ($(c)!0.5!(int-3) +(-100:0.22*\radA)$) {$  14$}; 

\end{tikzpicture} 

 & 5. \begin{tikzpicture}[dot/.style={circle, fill=black, inner sep=0pt, outer sep=0pt, minimum size=3pt}, baseline = (current bounding box.west)]

\coordinate (circ) at (0,0); 

\draw[name path=circum, line width=0.5mm] (circ) circle (\radA);

\coordinate (a) at ($(circ) + (170:\radA)$); 

\coordinate (b) at ($(circ) + (275:\radA)$); 

\coordinate (d) at ($(circ) + (43:\radA)$); 

\coordinate (e) at ($(a)! 0.5!(b)$); 

\draw[name path=line1, line width=0.3mm] (a) -- (b);

\path[name path=line.guess, line width=0.3mm] (e) -- ($(e)!1!180:(d)$); 

\fill[black, name intersections={of=circum and line.guess, name=int}];

\coordinate (tick1) at ($(a)! 0.5!(e)$);

\coordinate (tick2) at ($(b)! 0.5!(e)$);  

\tikzset{onetick/.pic={\draw[line width=0.3mm] ($(0,0)+(0,0.1*\radA)$) -- ($(0,0)-(0,0.1*\radA)$) ; }}

\pic[rotate=135] at (tick1) [pic type = onetick];  

\pic[rotate=135] at (tick2) [pic type = onetick];  

\node(6.label) at ($(tick2)+(-5:0.27*\radA)$) {$ 6$};

\node(3.label) at ($(e)! 0.5!(int-1)+(130:0.18*\radA)$) {$ 3$};

\node(x-label) at ($(d)!0.5!(int-1) +(100:0.25*\radA)$) {$ x$}; 

\node(e-label) at ($(e)+(90:0.3*\radA)$) {$ E$}; 

\node(d-label) at ($(d)+(55:0.25*\radA)$) {$ D$}; 

\node(a-label) at ($(a)+(130:0.25*\radA)$) {$ A$};

\node(b-label) at ($(b)-(130:0.25*\radA)$) {$ B$}; 

\node(c-label) at ($(int-1)-(55:0.25*\radA)$) {$ C$}; 

\draw[line width=0.3mm] (int-1) -- (d);

\end{tikzpicture} 
 \\
2. \begin{tikzpicture}[dot/.style={circle, fill=black, inner sep=0pt, outer sep=0pt, minimum size=3pt}, baseline = (current bounding box.west)]

\coordinate (circ) at (0,0); 

\draw[name path=circum, line width=0.5mm] (circ) circle (\radA);

\coordinate (a) at ($(circ) + (95:\radA)$); 

\coordinate (b) at ($(circ) + (-80:\radA)$); 

\coordinate (c) at ($(circ) + (60:\radA)$); 

\coordinate (d) at ($(circ) + (200:\radA)$); 

\draw[name path=line1, line width=0.3mm] (a) -- (b);

\draw[name path=line2, line width=0.3mm] (c) -- (d);

\fill[black, name intersections={of=line1 and line2, name=int}];

\node(10.label) at ($(a)!0.5!(int-1) + (180:0.25*\radA)$) {$  10$};

\node(x.label) at ($(c)!0.5!(int-1) +(-60:0.2*\radA)$) {$  x$};

\node(25-label) at ($(d)!0.5!(int-1) +(100:0.27*\radA)$) {$  25$}; 

\node(x6-label) at ($(b)!0.5!(int-1) +(10:0.44*\radA)$) {$  x+6$}; 

\end{tikzpicture} 
 & 6. \begin{tikzpicture}[dot/.style={circle, fill=black, inner sep=0pt, outer sep=0pt, minimum size=3pt}, baseline = (current bounding box.west)]

\node[draw,circle,minimum size=2*\radA,inner sep=0pt, line width=0.5mm, outer sep=0] (circ) at (0,0) {};

\path[name path=circum, line width=0.3mm] (circ) circle (\radA);

\coordinate (f) at ($(circ) + (100:\radA)$); 

\coordinate (h) at ($(circ) + (260:1.9*\radA)$); 

\node[dot] (i) at (tangent cs:node=circ, point={(h)}, solution=1) {};  

\draw[name path=line1, line width=0.3mm] (f) -- (h) -- (i) ;

\fill[black, name intersections={of=line1 and circum, name=int}];

\node(g.label) at ($(int-1)+(230:0.25*\radA)$) {$ G$};

\node(f.label) at ($(f)+(80:0.25*\radA)$) {$ F$};

\node(h.label) at ($(h)-(80:0.28*\radA)$) {$ H$};

\node(i.label) at ($(i)+(-70:0.25*\radA)$) {$ I$};

\node(11.label) at ($(h)!0.5!(i) +(-30:0.25*\radA)$) {$ 11$};

\node(9.label) at ($(h)!0.5!(int-1) +(190:0.13*\radA)$) {$ 9$};

\node(x-label) at ($(f)!0.5!(int-1) +(180:0.2*\radA)$) {$ x$}; 

\end{tikzpicture} 
 \\
3. \begin{tikzpicture}[dot/.style={circle, fill=black, inner sep=0pt, outer sep=0pt, minimum size=3pt}, baseline = (current bounding box.west)]

\coordinate (circ) at (0,0); 

\draw[name path=circum, line width=0.5mm] (circ) circle (\radA);

\coordinate (a) at ($(circ) + (180:\radA)$); 

\coordinate (b) at ($(circ) + (0:\radA)$); 

\coordinate (c) at ($(circ) + (70:\radA)$);

\coordinate (d) at ($(circ) + (-70:\radA)$); 

\draw[name path=line1, line width=0.3mm] (a) -- (b);

\draw[name path=line2, line width=0.3mm] (c) -- (d);

\fill[black, name intersections={of=line1 and line2, name=int}];

\begin{scope} [rotate=90]
\draw[line width=0.3mm] (int-1) rectangle ++(0.15*\radA,0.15*\radA) node[transform shape]{};
\end{scope} 

\coordinate (tick1) at ($(c)!0.5!(int-1)$);

\coordinate (tick2) at ($(d)!0.5!(int-1)$); 

\tikzset{onetick/.pic={\draw[line width=0.3mm] ($(0,0)+(0,0.1*\radA)$) -- ($(0,0)-(0,0.1*\radA)$) ; }}

\pic[rotate=90] at (tick1) [pic type = onetick]; 

\pic[rotate=90] at (tick2) [pic type = onetick]; 

\node(10.label) at ($(tick1)+(155:0.45*\radA)$) {$  10$};

\node(4.label) at ($(b)! 0.5!(int-1)+(90:0.22*\radA)$) {$  4$};

\node(x-label) at ($(a)!0.5!(int-1) +(90:0.18*\radA)$) {$  x$}; 

\end{tikzpicture} 
 & 7. \begin{tikzpicture}[dot/.style={circle, fill=black, inner sep=0pt, outer sep=0pt, minimum size=3pt}, baseline = (current bounding box.west)]

\coordinate (circ) at (0,0); 

\draw[name path=circum, line width=0.5mm] (circ) circle (\radA);

\coordinate (l) at ($(circ) + (90:\radA)$); 

\coordinate (k) at ($(circ) + (190:\radA)$);

\coordinate (n) at ($(circ) + (-15:\radA)$); 

\coordinate (j) at ($(l)!2!(k)$); 

\draw[name path=line1, line width=0.3mm] (l) -- (j) -- (n) ;

\fill[black, name intersections={of=line1 and circum, name=int}];

\node(m.label) at ($(int-3)+(-100:0.33*\radA)$) {$ M$};

\node(l.label) at ($(l)+(90:0.25*\radA)$) {$ L$};

\node(j.label) at ($(j)+(220:0.33*\radA)$) {$ J$};

\node(k.label) at ($(k)+(170:0.33*\radA)$) {$ K$};

\node(n.label) at ($(n)+(0:0.25*\radA)$) {$ N$};

\node(x-label) at ($(j)!0.5!(int-3) +(-90:0.2*\radA)$) {$ x$};

\node(10-label1) at ($(j)!0.5!(k) +(180:0.35*\radA)$) {$ 10$};

\node(10-label2) at ($(l)!0.5!(k) +(180:0.2*\radA)$) {$ 10$}; 

\node(8-label) at ($(n)!0.5!(int-3) +(-90:0.25*\radA)$) {$ 8$};

\end{tikzpicture} 
 \\
4. \begin{tikzpicture}[dot/.style={circle, fill=black, inner sep=0pt, outer sep=0pt, minimum size=3pt}, baseline = (current bounding box.west)]

\coordinate (circ) at (0,0); 

\draw[name path=circum, line width=0.5mm] (circ) circle (\radA);

\coordinate (a) at ($(circ) + (130:\radA)$); 

\coordinate (b) at ($(circ) + (260:1.6*\radA)$); 

\coordinate (c) at ($(circ) + (40:\radA)$); 

\draw[name path=line1, line width=0.3mm] (a) -- (b) -- (c) ;

\fill[black, name intersections={of=line1 and circum, name=int}];

\node(2.label) at ($(int-2)! 0.5!(b) + (180:0.2*\radA)$) {$  2$};

\node(1.label) at ($(int-4)! 0.5!(b) + (0:0.2*\radA)$) {$  1$};

\node(6x.label) at ($(int-2)! 0.5!(a) + (180:0.24*\radA)$) {$  6x$};

\node[anchor=north, inner sep=2pt, rotate=65] (2x4.label) at ($(int-4)! 0.5!(c)$) {$  2x+4$};


\end{tikzpicture} 
 & 8. \begin{tikzpicture}[dot/.style={circle, fill=black, inner sep=0pt, outer sep=0pt, minimum size=3pt}, baseline = (current bounding box.west)]

\coordinate (circ) at (0,0); 

\draw[name path=circum, line width=0.5mm] (circ) circle (\radA);

\coordinate (p) at ($(circ) + (175:\radA)$); 

\coordinate (r) at ($(circ) + (0:\radA)$); 

\coordinate (o) at ($(circ) + (130:\radA)$); 

\coordinate (q) at ($(circ) + (220:\radA)$); 

\draw[name path=line1, line width=0.3mm] (p) -- (r) ;

\draw[name path=line2, line width=0.3mm] (o) -- (q)  ;

\fill[black, name intersections={of=line1 and line2, name=int}];

\node(s.label) at ($(int-1)-(55:0.28*\radA)$) {$ S$};

\node(o.label) at ($(o)+(110:0.25*\radA)$) {$ O$};

\node(p.label) at ($(p)+(175:0.25*\radA)$) {$ P$};

\node(q.label) at ($(q)+(250:0.25*\radA)$) {$ Q$};

\node(r.label) at ($(r)+(-5:0.25*\radA)$) {$ R$};

\node(50-label) at ($(r)!0.5!(int-1) +(100:0.25*\radA)$) {$ 50$}; 

\node(3-label) at ($(p)!0.5!(int-1) +(90:0.17*\radA)$) {$ 3$}; 

\node(2x-label) at ($(q)!0.5!(int-1) +(0:0.25*\radA)$) {$ 2x$}; 

\node(3x.label) at ($(o)!0.5!(int-1) +(0:0.25*\radA)$) {$ 3x$};


\end{tikzpicture} 

 \\
\end{tabularx}}
\end{minipage}}
\end{center} 
  }

\def \PracticeTwoDayB {\input{/storage/emulated/0/Documents/documents/latex/1920/Grade-10/2nd/power-theorems/ps-power-theorems-input2-dll}
}

\def \MasteryDayB {\def\curdir{/storage/emulated/0/Documents/documents/latex/1920/Grade-10/2nd/power-theorems}
\def\radB{1cm}

\textbf{Problem Set}

\vspce

Find the value of $x$. 
\begin{center}
\vspace*{-2ex}
\scalebox{0.5}{
\noindent\begin{minipage}{0.3\textwidth}
{\begin{tabularx}{\textwidth}{XX}
1. \begin{tikzpicture}[dot/.style={circle, fill=black, inner sep=0pt, outer sep=0pt, minimum size=3pt}, baseline = (current bounding box.west)]

\coordinate (circ) at (0,0); 

\draw[name path=circum, line width=0.5mm] (circ) circle (\radB);

\coordinate (z) at ($(circ) + (170:\radB)$); 

\coordinate (v) at ($(circ) + (275:1.8*\radB)$); 

\coordinate (t) at ($(circ) + (70:\radB)$); 

\draw[line width=0.3mm] (z) -- (v) -- (t) ;

\node(z.label) at ($(z)+(130:0.25*\radB)$) {$ Z$};

\node(t.label) at ($(t)+(80:0.25*\radB)$) {$ T$};

\node(v.label) at ($(v)+(-80:0.25*\radB)$) {$ V$};


\path[name path=linezv, line width=0.3mm] (z) -- (v);

\fill[black, name intersections={of=linezv and circum, name=intzv}];

\node(w.label) at ($(intzv-1)+(205:0.25*\radB)$) {$W$};

\node(x-label) at ($(z)!0.5!(intzv-1)+ (0:0.17*\radB)$) {$x$}; 

\node(2x-label) at ($(v)!0.5!(intzv-1) + (190:0.22*\radB)$) {$ 2x$}; 

\path[name path=linetv, line width=0.3mm] (v) -- (t) ;

\fill[black, name intersections={of=linetv and circum, name=inttv}];

\node(h.label) at ($(inttv-1)+(-50:0.22*\radB)$) {$ H$};

\node(5-label) at ($(t)!0.5!(inttv-1) +(0:0.15*\radB)$) {$ 5$}; 

\node(3-label) at ($(v)!0.5!(inttv-1) +(-20:0.17*\radB)$) {$ 3$}; 



\end{tikzpicture} 
 & 5. \begin{tikzpicture}[dot/.style={circle, fill=black, inner sep=0pt, outer sep=0pt, minimum size=3pt}, baseline = (current bounding box.west)]

\coordinate (circ) at (0,0); 

\draw[name path=circum, line width=0.5mm] (circ) circle (\radB);

\coordinate (p) at ($(circ) + (90:\radB)$); 

\coordinate (r) at ($(circ) + (260:\radB)$); 

\coordinate (e) at ($(circ) + (70:\radB)$); 

\coordinate (s) at ($(r) + (175:1.2*\radB)$); 

\draw[name path=line1, line width=0.3mm] (p) -- (r) -- (s) -- (e) ;


\path[name path=line2, line width=0.3mm] (s) -- (e) ;

\fill[black, name intersections={of=line1 and line2, name=intt}];

\node(t.label) at ($(intt-1)+(-35:0.2*\radB)$) {$ T$};

\fill[black, name intersections={of=line2 and circum, name=intu}];

\node(u.label) at ($(intu-1)+(185:0.25*\radB)$) {$ U$};

\node(p-label) at ($(p)+(90:0.26*\radB)$) {$ P$}; 

\node(e-label) at ($(e)+(50:0.26*\radB)$) {$ E$}; 

\node(r-label) at ($(r)+(-90:0.26*\radB)$) {$ R$}; 

\node(s-label) at ($(s)+(190:0.26*\radB)$) {$ S$}; 

\node(3-label) at ($(p)!0.5!(intt-1) +(180:0.2*\radB)$) {$ 3$}; 

\node(4-label) at ($(e)!0.5!(intt-1) +(-20:0.17*\radB)$) {$ 4$}; 

\node(x-label) at ($(intu-1)!0.5!(intt-1) +(110:0.2*\radB)$) {$ x$}; 

\node(y-label) at ($(r)!0.5!(intt-1) +(0:0.2*\radB)$) {$ y$}; 

\node(6-label) at ($(s)!0.5!(r) +(-90:0.2*\radB)$) {$ 6$}; 

\node(4-label) at ($(s)!0.5!(intu-1)+(-30:0.12*\radB)$) {$ 4$}; 

\end{tikzpicture} 
 \\
2. \begin{tikzpicture}[dot/.style={circle, fill=black, inner sep=0pt, outer sep=0pt, minimum size=3pt}, baseline = (current bounding box.west)]

\coordinate (circ) at (0,0); 

\draw[name path=circum, line width=0.5mm] (circ) circle (\radB);

\coordinate (r) at ($(circ) + (120:\radB)$); 

\coordinate (n) at ($(circ) + (0:1.6*\radB)$); 

\coordinate (e) at ($(circ) + (210:\radB)$); 

\draw[name path=line1, line width=0.3mm] (r) -- (n) -- (e) ;

\fill[black, name intersections={of=line1 and circum, name=int}];

\node(i.label) at ($(int-1)+(60:0.25*\radB)$) {$ I$};

\node(r-label) at ($(r)+(160:0.25*\radB)$) {$ R$}; 

\node(n-label) at ($(n)+(0:0.25*\radB)$) {$ N$};

\node(e-label) at ($(e)+(-150:0.25*\radB)$) {$ E$};  

\node(g.label) at ($(int-4)+(-40:0.2*\radB)$) {$ G$};

\node(x-label) at ($(r)!0.5!(int-1) +(90:0.17*\radB)$) {$ x$}; 

\node(3-label) at ($(n)!0.5!(int-1) +(70:0.25*\radB)$) {$ 3$}; 

\node(5-label) at ($(e)!0.5!(int-4) +(-90:0.25*\radB)$) {$ 5$}; 

\node(4-label) at ($(n)!0.5!(int-4) +(-70:0.25*\radB)$) {$ 4$}; 

\end{tikzpicture} 

 & 6. \begin{tikzpicture}[dot/.style={circle, fill=black, inner sep=0pt, outer sep=0pt, minimum size=3pt}, baseline = (current bounding box.west)]

\node[draw,circle,minimum size=2*\radB,inner sep=0pt, line width=0.5mm, outer sep=0] (circ) at (0,0) {};

\path[name path=circum, line width=0.5mm] (circ) circle (\radB);

\coordinate (e) at ($(circ) + (180:1.7*\radB)$); 

\coordinate (o) at ($(circ) + (20:1.7*\radB)$); 

\coordinate (s) at (tangent cs:node=circ, point={(o)}, solution=2);  

\draw[name path=line1, line width=0.3mm] (e) -- (o) -- (s) -- cycle ;

\path[name path=line2, line width=0.3mm] (e) -- (o) ;

\fill[black, name intersections={of=line2 and circum, name=int}];

\node(l.label) at ($(int-2)+(130:0.28*\radB)$) {$ L$};

\node(e.label) at ($(e)+(180:0.17*\radB)$) {$ E$};

\node(o.label) at ($(o)+(30:0.28*\radB)$) {$ O$};

\node(s.label) at ($(s)+(-40:0.25*\radB)$) {$ S$};

\node(m.label) at ($(int-1)+(60:0.23*\radB)$) {$ M$};

\node(12-label) at ($(int-1)!0.5!(int-2) +(90:0.25*\radB)$) {$ 12$};

\node(x-label) at ($(e)!0.5!(int-2) +(120:0.22*\radB)$) {$ x$}; 

\node(x2-label) at ($(o)!0.5!(int-1) +(50:0.23*\radB)$) {$ x$}; 

\path[name path=line3, line width=0.3mm] (e) -- (s) ;

\fill[black, name intersections={of=line3 and circum, name=int}];

\node(y-label) at ($(e)!0.5!(int-1) +(-100:0.25*\radB)$) {$ y$};

\node(9-label) at ($(s)!0.5!(int-1) +(-100:0.25*\radB)$) {$ 9$}; 

\node(8-label) at ($(s)!0.5!(o) +(-40:0.25*\radB)$) {$ 8$};

\end{tikzpicture} 
 \\
3. \begin{tikzpicture}[dot/.style={circle, fill=black, inner sep=0pt, outer sep=0pt, minimum size=3pt}, baseline = (current bounding box.west)]

\node[draw,circle,minimum size=2*\radB,inner sep=0pt, line width=0.5mm, outer sep=0] (circ) at (0,0) {};

\coordinate (g) at ($(circ) + (-80:\radB)$); 

\coordinate (n) at ($(circ) + (0:\radB)$); 

\coordinate (o) at ($(n)!-1!(g)$); 

\coordinate (s) at (tangent cs:node=circ, point={(o)}, solution=1);  

\draw[name path=line1, line width=0.3mm] (g) --(n) node[midway] (tick1) {} -- (o) node[midway] (tick2) {} -- (s) ;

\node(g.label) at ($(g)+(-90:0.25*\radB)$) {$ G$};

\node(s.label) at ($(s)+(90:0.25*\radB)$) {$ S$};

\node(o.label) at ($(o)+(30:0.25*\radB)$) {$ O$};

\tikzset{onetick/.pic={\draw[line width=0.3mm] ($(0,0)+(0,0.1*\radB)$) -- ($(0,0)-(0,0.1*\radB)$) ; }}

\pic[rotate=50] at (tick1) [pic type = onetick];  

\pic[rotate=50] at (tick2) [pic type = onetick];  

\node(x.label) at ($(tick1)+(90:0.28*\radB)$) {$ x$};

\node(10-label) at ($(s)!0.5!(o) +(90:0.25*\radB)$) {$ 10$}; 

\end{tikzpicture} 

 & 7. \begin{tikzpicture}[dot/.style={circle, fill=black, inner sep=0pt, outer sep=0pt, minimum size=3pt}, baseline = (current bounding box.west)]

\coordinate (circ) at (0,0); 

\draw[name path=circum, line width=0.5mm] (circ) circle (\radB);

\coordinate (e) at ($(circ) + (100:\radB)$); 

\coordinate (o) at ($(circ) + (5:\radB)$); 

\coordinate (r) at ($(circ) + (70:\radB)$);

\coordinate (d) at ($(circ) + (-90:\radB)$); 

\coordinate (a) at ($(d) + (180:1.3*\radB)$);

\draw[name path=line1, line width=0.3mm] (e) -- (o);

\draw[line width=0.3mm](r) -- (a) -- (d) ;

\path[name path=line2, line width=0.3mm] (r) -- (a);

\fill[black, name intersections={of=line2 and circum, name=ints}];

\node(e.label) at ($(e)+(130:0.25*\radB)$) {$ E$};

\node(s.label) at ($(ints-1)+(180:0.25*\radB)$) {$ S$};

\node(r.label) at ($(r)+(40:0.25*\radB)$) {$ R$};

\node(o.label) at ($(o)+(-20:0.25*\radB)$) {$ O$};

\node(d.label) at ($(d)+(-90:0.25*\radB)$) {$ D$};

\node(11-label) at ($(a)!0.5!(d) +(-90:0.25*\radB)$) {$ 11$}; 

\node(a.label) at ($(a)+(170:0.25*\radB)$) {$ A$};

\fill[black, name intersections={of=line1 and line2, name=int}];

\node(y-label) at ($(e)!0.5!(int-1) +(235:0.25*\radB)$) {$ y$}; 

\node(5-label) at ($(r)!0.5!(int-1) +(-30:0.25*\radB)$) {$ 5$}; 

\node(9-label) at ($(o)!0.5!(int-1) +(260:0.25*\radB)$) {$ 9$}; 

\node(x-label) at ($(ints-1)!0.5!(int-1) +(-40:0.2*\radB)$) {$ x$}; 

\node(7-label) at ($(a)!0.5!(ints-1) +(180:0.25*\radB)$) {$ 7$}; 

\end{tikzpicture} 

 \\
4.  \begin{tikzpicture}[dot/.style={circle, fill=black, inner sep=0pt, outer sep=0pt, minimum size=3pt}, baseline = (current bounding box.west)]

\node[draw,circle,minimum size=2*\radB,inner sep=0pt, line width=0.5mm, outer sep=0] (circ) at (0,0) {};

\coordinate (b) at ($(circ) + (95:\radB)$); 

\coordinate (l) at ($(circ) + (-10:\radB)$); 

\coordinate (e) at ($(l)! -0.5! (b)$); 

\coordinate (s) at (tangent cs:node=circ, point={(e)}, solution=2);  

\draw[name path=line1, line width=0.3mm] (b) -- (l) node[midway] (tick1) {} -- (e) node[midway] (tick2) {}  -- (s) node[midway] (tick3) {} ;

\node(b.label) at ($(b)+(90:0.26*\radB)$) {$ B$};

\node(l.label) at ($(l)+(40:0.26*\radB)$) {$ L$};

\node(e.label) at ($(e)+(0:0.26*\radB)$) {$ E$};

\node(s.label) at ($(s)+(-90:0.26*\radB)$) {$ S$};

\node[rotate=-47](x2.label) at ($(tick1)+(30:0.17*\radB)$) {$ x+2$};

\node(x-label) at ($(tick2) +(20:0.25*\radB)$) {$ x$};

\node[rotate=5](x1-label) at ($(tick3) +(-90:0.25*\radB)$) {$ x+1$}; 

\end{tikzpicture} 

 & \\
\end{tabularx}}
\end{minipage}}
\end{center} 
  }



\def \EvaluationDayB {
\input{/storage/emulated/0/Documents/documents/latex/1920/Grade-10/2nd/power-theorems/qz-power-theorems-input} 
}

\def \RemediationDayB {}

\def \LessonDayC {Area of Sectors and Segments of a Circle}

\def \LearningCompetenciesDayC {}

\def \ObjectivesDayC {
\item %cogverbstart
Measure
%cogverbend
the area of sectors and segments of a circle; 
\item %psyverbstart
Calculate
%psyverbend
the area of sectors and segments of a circle to determine whether a binomial is a factor of a given polynomial; and, 
\item %affverbstart
Exhibit
%affverbend
%valsstart
independence and self-reliance
%valsend
in solving problems.
}

\def \PurposeDayC {The purpose of this lesson is to enable the students to solve real life problems involving the area of sectors and segments of a circle.}  

\def \ApplicationDayC { Let the students answer the following questions: 
\begin{enumerate}[label = \arabic*. ]
%1
\item In what real life situations or problems can we observe some examples of area of sectors and segments of a circle? 
%2
\item How can you apply your knowledge of area of sectors and segments of a circle in solving these real life problems? 

\end{enumerate}   
}

\def \GeneralizationDayC {Let the students answer the following questions: 
\begin{enumerate}[label = \arabic*. ]
%1
\item In your own words, what is the area of sectors and segments of a circle? 
%2
\item How do we solve problems involving area of sectors and segments of a circle? 

\end{enumerate}   
}

\def \TeachersGuideDayC {pp. 145--150}

\def \LMPagesDayC {pp. 134--139}

\def \TextbookPagesDayC {pp. 145--151}

\def \AdditionalMaterialsDayC {}

\def \OtherResourcesDayC {Flashcards}

\def \ReviewDayC {\begin{center}
\textbf{Area of Sectors and Segments of a Circle}
\end{center}

\vspace*{1ex}

\textbf{Sector of a Circle:} a region in the circle bounded by two radii and the minor arc they determine

\vspce 

The \textbf{area of a sector} is represented by $A = \dfrac{n}{360}\pi r^2$, where $n$ is the number of degrees in the central angle of a sector.

\vspce 

\textbf{Segment of a Circle:} a region bounded by an arc and the chord of the arc

\vspce 

The \textbf{area of a segment} of a circle is found by subtracting the area of a triangle from the area of a sector. 
\[
\text A_{segment} = A_{sector}-A_{triangle}
\] 



}



\def \ExamplesDayC {}

\def \PracticeOneDayC {\def\curdir{/storage/emulated/0/Documents/documents/latex/1920/Grade-10/2nd/area-of-sectors-and-segments-of-a-circle/fig-a}

\def\radA{1.2cm}

\textbf{Practice Exercises}

\vspce

Find the area of each shaded region/s in each figure.  Express your answer in terms of $\pi$. 
\begin{center}
\vspace*{-2ex}
\scalebox{0.5}{
\noindent\begin{minipage}{0.3\textwidth}
{\begin{tabularx}{\textwidth}{XX}
1. \begin{tikzpicture}[baseline = (current bounding box.west)]


\node[draw, circle, minimum size=2*\radA, inner sep=0pt, line width=0.5mm, outer sep=0] (circ) at (0,0) {};

\node[anchor=north, inner sep=2pt, rotate=0] (8cm.label) at ($(circ.center)!0.5!(0:\radA)$) {$ \text{8 cm}$};

\node[anchor=south west, inner sep=2pt, rotate=0] (90.label) at ($(circ.center)+(45:\radA)$) {$ 90 \degree $};

\filldraw[fill=\figurefill, line width=0.3mm]  (0:\radA) arc (0:90:\radA) -- (circ.center) -- cycle;

\end{tikzpicture} 
 & 5. \begin{tikzpicture}[dot/.style={circle, fill=black, inner sep=0pt, outer sep=0pt, minimum size=3pt}, baseline = (current bounding box.west)]



\node[draw, circle, minimum size=2*\radA, inner sep=0pt, line width=0.5mm, outer sep=0] (circ) at (0,0) {};

\node[dot] (circ.dot)  at (circ) {}; 

\coordinate (a) at ($(circ) + (45:\radA)$); 

\coordinate (b) at ($(circ) - (45:\radA)$); 

\coordinate (c) at ($(circ) + (-45:\radA)$); 

\draw[line width=0.3mm] (a) -- (b) (b) -- (c) (c) -- (a) ;

\node[anchor=south, inner sep=2pt] (10cm1-label) at ($(b)!0.5!(c)$) {$ \text{10 cm}$};  

\filldraw[fill=\figurefill, line width=0.3mm]  (45:\radA) arc (45:-45:\radA) -- cycle;

\filldraw[fill=\figurefill, line width=0.3mm]  (-45:\radA) arc (-45:-135:\radA) -- cycle;

\node[anchor=south, inner sep=2pt, rotate=90] (10cm2-label) at ($(a)!0.5!(c)$) {$ \text{10 cm}$};  

\begin{scope} [rotate=90]
\draw[line width=0.3mm] (c) rectangle ++(0.15*\radA,0.15*\radA) node[transform shape]{};
\end{scope} 

\end{tikzpicture}  
 \\
2. \begin{tikzpicture}[baseline = (current bounding box.west)]



\node[draw, circle, minimum size=2*\radA, inner sep=0pt, line width=0.5mm, outer sep=0] (circ) at (0,0) {};

\node[anchor=south, inner sep=2pt, rotate=-30] (12cm.label) at ($(circ.center)!0.5!(-30:\radA)$) {$ \text{12 cm}$};

\node[anchor=north west, inner sep=2pt, rotate=0] (60.label) at ($(circ.center)+(-60:\radA+0pt)$) {$ 60 \degree $};

\filldraw[fill=\figurefill, line width=0.3mm]  (-30:\radA) arc (-30:-90:\radA) -- (circ.center) -- cycle;

\end{tikzpicture} 

 & 6. \begin{tikzpicture}[dot/.style={circle, fill=black, inner sep=0pt, outer sep=0pt, minimum size=3pt}, baseline = (current bounding box.west)]

\node[draw, circle, minimum size=2*\radA, inner sep=0pt, line width=0.5mm, outer sep=0] (circ) at (0,0) {};

\node[dot] (circ.dot)  at (circ) {}; 

\coordinate (a) at ($(circ) + (60:\radA)$); 

\coordinate (n) at ($(circ) + (0:\radA)$); 

\draw[line width=0.3mm] (a) -- (circ.center) -- (n) -- cycle;

\filldraw[fill=\figurefill, line width=0.3mm]  (0:\radA) arc (0:60:\radA) -- cycle;

\node[anchor=south, inner sep=2pt, rotate=60] (6cm1.label) at ($(circ.center)!0.5!(a)$) {$\text{6cm}$};

\node[anchor=south, inner sep=2pt, rotate=0] (60.label) at ($(circ.center)!0.45!(n)$) {$ 60 \degree $};

\end{tikzpicture}  
 \\
3. \begin{tikzpicture}[baseline = (current bounding box.west)]



\node[draw, circle, minimum size=2*\radA, inner sep=0pt, line width=0.5mm, outer sep=0] (circ) at (0,0) {};

\node[anchor=south, inner sep=2pt, rotate=-90] (10cm.label) at ($(circ.center)!0.5!(90:\radA)$) {$ \text{10 cm}$};

\node[anchor=south east, inner sep=2pt, rotate=0] (120.label) at ($(circ.center)+(150:0.9*\radA)$) {$ 120 \degree $};

\filldraw[fill=\figurefill, line width=0.3mm]  (90:\radA) arc (90:210:\radA) -- (circ.center) -- cycle;

\end{tikzpicture} 

 & 7. \begin{tikzpicture}[dot/.style={circle, fill=black, inner sep=0pt, outer sep=0pt, minimum size=0pt}, baseline = (current bounding box.west)]

\node[draw, circle, minimum size=2*\radA, inner sep=0pt, line width=0.5mm, outer sep=0] (circ) at (0,0) {};

\node[dot] (circ.dot)  at (circ) {}; 

\coordinate (a) at ($(circ) + (180:\radA)$); 

\coordinate (t) at ($(circ) + (90:\radA)$); 

\draw[line width=0.3mm] (a) -- (circ.center) -- (t) ;

\node[anchor=south, inner sep=2pt, rotate=0] (4.5cm-label) at ($(a)!0.5!(circ.center)$) {\tiny $ \text{4.5 cm} $};  

\filldraw[fill=\figurefill, line width=0.3mm]  (90:\radA) arc (90:-180:\radA) -- (circ.center) -- cycle;

\begin{scope} [rotate=90]
\draw[line width=0.3mm] (circ.center) rectangle ++(0.13*\radA,0.13*\radA) node[transform shape]{};
\end{scope} 

\end{tikzpicture}  
 \\
4. \begin{tikzpicture}[baseline = (current bounding box.west)]



\node[draw, circle, minimum size=2*\radA, inner sep=0pt, line width=0.5mm, outer sep=0] (circ) at (0,0) {};

\node[anchor=south, inner sep=2pt, rotate=0] (16cm.label) at ($(circ.center)!0.5!(0:\radA)$) {$ \text{16 cm}$};

\filldraw[fill=\figurefill, line width=0.3mm]  (0:\radA) arc (0:-90:\radA) -- cycle;

\draw[line width=0.3mm]  (0:\radA) -- (circ.center) -- (-90:\radA);

\begin{scope} [rotate=-90]
\draw[line width=0.3mm] (circ.center) rectangle ++(0.15*\radA,0.15*\radA) node[transform shape]{};
\end{scope} 

\end{tikzpicture} 
 & 8. \begin{tikzpicture}[baseline = (current bounding box.west)]

\node[draw, circle, minimum size=2*\radA, inner sep=0pt, line width=0.5mm, outer sep=0] (circ) at (0,0) {};

\node[anchor=south, inner sep=2pt, rotate=0] (8in.label) at ($(circ.center)!0.5!(0:\radA)$) {$ \text{8 in}$};

\filldraw[fill=\figurefill, line width=0.3mm]  (0:\radA) arc (0:-60:\radA) -- cycle;

\filldraw[fill=\figurefill, line width=0.3mm]  (180:\radA) arc (180:120:\radA) -- cycle;

\node[rotate=0] (120-label) at ($(circ.center)+(220:0.42*\radA)$) {$ 120\degree $}; 

\draw[line width=0.3mm] (0:\radA) -- (180:\radA) -- (120:\radA) -- (-60:\radA) -- cycle;

\end{tikzpicture} 

 \\
\end{tabularx}}
\end{minipage}}
\end{center} 
  }

\def \PracticeTwoDayC {\input{/storage/emulated/0/Documents/documents/latex/1920/Grade-10/2nd/area-of-sectors-and-segments-of-a-circle/ps-area-of-sectors-and-segments-of-a-circle-input2-dll}
}

\def \MasteryDayC {\def\curdir{/storage/emulated/0/Documents/documents/latex/1920/Grade-10/2nd/area-of-sectors-and-segments-of-a-circle/fig-b}

\def\radB{1.7cm}

\textbf{Problem Set}

\vspce

Find the area of each shaded region/s in each figure.  Express your answer in terms of $\pi$. 
\begin{center}
\vspace*{-2ex}
\scalebox{0.5}{
\noindent\begin{minipage}{0.3\textwidth}
{\begin{tabularx}{\textwidth}{XX}
1. \begin{tikzpicture}[baseline = (current bounding box.west)]

\node[draw, circle, minimum size=2*\radB, inner sep=0pt, line width=0.5mm, outer sep=0] (circ) at (0,0) {};

\node[anchor=south, inner sep=2pt, rotate=-30] (12cm.label) at ($(circ.center)!0.5!(-30:\radB)$) {$ \text{12 cm}$};

\node[anchor=north west, inner sep=2pt, rotate=0] (60.label) at ($(circ.center)+(-60:\radB+0pt)$) {$ 60 \degree $};

\filldraw[fill=\figurefill, line width=0.3mm]  (-30:\radB) arc (-30:-90:\radB) -- cycle;

\draw[line width=0.3mm] (-30:\radB) -- (-90:\radB) -- (circ.center) -- cycle;

\end{tikzpicture} 

 & 5. \begin{tikzpicture}[baseline = (current bounding box.west)]


\node[draw, circle, minimum size=2*\radB, inner sep=0pt, line width=0.5mm, outer sep=0] (circ) at (0,0) {};

\node[anchor=south, inner sep=2pt, rotate=90] (18cm.label) at ($(circ.center)!0.5!(90:\radB)$) {\tiny $ \text{18 cm}$};
  
\filldraw[fill=\figurefill, line width=0.3mm]  (90:\radB) arc (90:-180:\radB) -- (circ.center) -- cycle;

\begin{scope} [rotate=90]
\draw[line width=0.3mm] (circ.center) rectangle ++(0.15*\radB,0.15*\radB) node[transform shape]{};
\end{scope} 

\end{tikzpicture} 
 \\
2. \begin{tikzpicture}[dot/.style={circle, fill=black, inner sep=0pt, outer sep=0pt, minimum size=3pt}, baseline = (current bounding box.west)]

\node[draw, circle, minimum size=2*\radB, inner sep=0pt, line width=0.5mm, outer sep=0] (circ) at (0,0) {};

\node[dot] (circ.dot)  at (circ) {}; 

\coordinate (l) at ($(circ) + (45:\radB)$); 

\coordinate (e) at ($(circ) + (0:\radB)$); 


\draw[line width=0.3mm] (l) -- (circ.center) -- (e);

\node[anchor=north, inner sep=2pt, rotate=0] (9cm.label) at ($(circ.center)!0.5!(e)$) {$ \text{9 cm}$};

\node[anchor=west, inner sep=2pt, rotate=0] (45.label) at (20:1.1*\radB) {$ 45 \degree $};

\filldraw[fill=\figurefill, line width=0.3mm]  (0:\radB) arc (0:45:\radB) -- (circ.center) -- cycle;

\end{tikzpicture}  

 & 6. \begin{tikzpicture}[baseline = (current bounding box.west)][dot/.style={circle, fill=black, inner sep=0pt, outer sep=0pt, minimum size=3pt}, dim-label/.style={fill=white, rectangle, inner sep=2pt, outer sep=0pt}%, remember picture, overlay
]  

\coordinate (center) at (0,0);

\filldraw[fill=\figurefill, name path=circ, line width=0.3mm] (center) circle (\radB); 

\filldraw[fill=white, line width=0.3mm] (center) circle (0.65*\radB); 

\node[anchor=west, inner sep=2pt, rotate=0] (5cm.label) at ($(center)!0.35!(250:\radB)$) {\tiny $ \text{5 cm}$};

\node[anchor=west, inner sep=2pt, rotate=0] (3cm.label) at ($(center)!0.87!(250:\radB)$) {\tiny $ \text{3 cm}$};

\draw[line width=0.3mm] (25:0.65*\radB) -- (center) --(250:\radB); 

\end{tikzpicture} 

 \\
3. \begin{tikzpicture}[dot/.style={circle, fill=black, inner sep=0pt, outer sep=0pt, minimum size=3pt}, baseline = (current bounding box.west)]

\node[draw, circle, minimum size=2*\radB, inner sep=0pt, line width=0.5mm, outer sep=0] (circ) at (0,0) {};

\node[dot] (circ.dot)  at (circ) {}; 

\coordinate (v) at ($(circ) + (180:\radB)$); 

\coordinate (c) at ($(circ) + (60:\radB)$); 

\draw[line width=0.3mm] (c) -- (circ.center) -- (v) ;

\node[anchor=north, inner sep=2pt, rotate=60] (10cm-label) at ($(circ.center)! 0.5!(c) $)  {$ \text{10 cm}$};  

\node[anchor=north, inner sep=2pt, rotate=0] (240-label) at (-40:1.22*\radB) {$ \text{240\degree } $}; 

\filldraw[fill=\figurefill, line width=0.3mm]  (60:\radB) arc (60:180:\radB) -- cycle;

\end{tikzpicture}  
 & 7. \begin{tikzpicture}[dot/.style={circle, fill=black, inner sep=0pt, outer sep=0pt, minimum size=2pt}, dim-label/.style={fill=white, rectangle, inner sep=2pt, outer sep=0pt}, baseline = (current bounding box.west)]  

\node[draw, circle, minimum size=2*\radB, inner sep=0pt, line width=0.5mm, outer sep=0] (circ) at (0,0) {};

\filldraw[fill=\figurefill, line width=0.3mm]  (180:\radB) arc (180:0:\radB) -- cycle;

\coordinate(center2) at ($(180:\radB)! 0.5! (circ.center) $);

\filldraw[fill=white, line width=0.3mm] (center2) circle (0.5*\radB); 

\coordinate(center3) at ($(0:\radB)! 0.5! (circ.center) $);

\filldraw[fill=white, line width=0.3mm] (center3) circle (0.5*\radB); 

\node [dot] at (center2){}; 

\node [dot] at (center3){}; 


\draw[line width=0.3mm] (180:\radB) -- (0:\radB); 

\node[anchor=north, inner sep=2pt, rotate=0] (8cm1.label) at (center2) {$ \text{8 cm}$};

\node[anchor=north, inner sep=2pt, rotate=0] (8cm2.label) at (center3) {$ \text{8 cm}$};

\end{tikzpicture} 
 \\
4. \begin{tikzpicture}[dot/.style={circle, fill=black, inner sep=0pt, outer sep=0pt, minimum size=3pt}, baseline = (current bounding box.west)]

\node[draw, circle, minimum size=2*\radB, inner sep=0pt, line width=0.5mm, outer sep=0] (circ) at (0,0) {};

\node[dot] (circ.dot)  at (circ) {}; 

\coordinate (u) at ($(circ) + (25:\radB)$); 

\coordinate (m) at ($(circ) + (155:\radB)$); 

\draw[line width=0.3mm] (u) -- (circ.center) -- (m) ;

\node[anchor=north, inner sep=2pt, rotate=25] (8cm-label) at ($(u)!0.5!(circ.center)$) {$ \text{8 cm} $};  

\node[rotate=0] (130-label) at ($(circ.center)+(80:1.22*\radB)$) {$ 130\degree $};

\filldraw[fill=\figurefill, line width=0.3mm]  (25:\radB) arc (25:155:\radB) --(circ.center) -- cycle;

\end{tikzpicture}  
 & 8. \begin{tikzpicture}[baseline = (current bounding box.west)]



\node[draw, circle, minimum size=2*\radB, inner sep=0pt, line width=0.5mm, outer sep=0] (circ) at (0,0) {};

\node[anchor=south, inner sep=2pt, rotate=0] (10in.label) at ($(circ.center)!0.5!(0:\radB)$) {$ \text{10 in}$};

\filldraw[fill=\figurefill, line width=0.3mm]  (0:\radB) arc (0:-60:\radB) -- (circ.center) -- cycle;

\filldraw[fill=\figurefill, line width=0.3mm]  (180:\radB) arc (180:120:\radB) -- (circ.center) -- cycle;

\node[rotate=0] (120-label) at ($(circ.center)+(220:0.42*\radB)$) {$ 120\degree $}; 


\end{tikzpicture} 

 \\
\end{tabularx}}
\end{minipage}}
\end{center} 
  }



\def \EvaluationDayC {
\input{/storage/emulated/0/Documents/documents/latex/1920/Grade-10/2nd/area-of-sectors-and-segments-of-a-circle/qz-area-of-sectors-and-segments-of-a-circle-input} 
}

\def \RemediationDayC {}

\def \LessonDayD {Distance Formula}

\def \LearningCompetenciesDayD {}

\def \ObjectivesDayD {
\item %cogverbstart
Summarize
%cogverbend
the distance formula; 
\item %psyverbstart
Compute
%psyverbend
the distance formula to determine whether a binomial is a factor of a given polynomial; and, 
\item %affverbstart
Show
%affverbend
%valsstart
willingness and self-reliance
%valsend
in solving problems.
}

\def \PurposeDayD {The purpose of this lesson is to enable the students to solve real life problems involving the distance formula.}  

\def \ApplicationDayD { Let the students answer the following questions: 
\begin{enumerate}[label = \arabic*. ]
%1
\item In what real life situations or problems can we observe some examples of distance formula? 
%2
\item How can you apply your knowledge of distance formula in solving these real life problems? 

\end{enumerate}   
}

\def \GeneralizationDayD {Let the students answer the following questions: 
\begin{enumerate}[label = \arabic*. ]
%1
\item In your own words, what is the distance formula? 
%2
\item How do we solve problems involving distance formula? 

\end{enumerate}   
}

\def \TeachersGuideDayD {pp. 151--154}

\def \LMPagesDayD {pp. 140--143}

\def \TextbookPagesDayD {pp. 152--156}

\def \AdditionalMaterialsDayD {}

\def \OtherResourcesDayD {Flashcards}

\def \ReviewDayD {\begin{center}
\textbf{The Distance Formula 
}
\end{center}

\vspce 

If $A(x_1, y_1)$ and $B(x_2, y_2)$ are any two points on the coordinate plane, then
\[
AB = \sqrt{(x_2-x_1)^2 + (y_2-y_1)^2}
\] 

}



\def \ExamplesDayD {}

\def \PracticeOneDayD {\textbf{Practice Exercises}
%\textbf{Problem Set}

\vspce
%\begin{enumerate}[label = \Alph*. ]
%A
%\item \hspce 
A. Find $PQ$. 
\begin{enumerate}[label = \arabic*. ]
%1
\item $P(4, 2), Q(8, 2)$
%2
\item $P(3, 5), Q(3, -2)$
%3
\item $P(0, 0), Q(-4, 3)$
%4
\item $P(-2, 6), Q(-7, 7)$
%5
\item $P(5, 2), Q(0,-6)$
\end{enumerate} 

%B
%\item \hspce 
B. Find the  length  of each side of $\triangle EXP$. Tell whether $\triangle EXP$ is isosceles, right, or neither. 
\begin{enumerate}[label = \arabic*. ]
%1
\item $E(4,3), X(-1,1), P(5,0)$ 
%2
\item $E(-3,-2), X(1,-1), P(0,2)$ 
%3
\item $E(0,8), X(9,6), P(8,10) $

\end{enumerate} 
%C
%\item \hspce
%\end{enumerate} 




}

\def \PracticeTwoDayD {j
}

\def \MasteryDayD {%\textbf{Practice Exercises}
\textbf{Problem Set}

\vspce
\begin{enumerate}[label = \Alph*. ]
%A
\item \hspce Find $PQ$. 
\begin{enumerate}[label = \arabic*. ]
%1
\item $P(-3, 1), Q(3, 9)$
%2
\item $P(4, 2), Q(9, 14)$
%3
\item $P(-8, 0), Q(16, -24)$
%4
\item $P(-3, 4), Q(5, 2)$
%5
\item $P(4, 5), Q(1,3)$
\end{enumerate} 

%B
\item \hspce Find the  length  of each side of $\triangle EXP$. Tell whether $\triangle EXP$ is isosceles, right, or neither. 
\begin{enumerate}[label = \arabic*. ]
%1
\item $E(1,8)$, $X(6,-4)$, $P(11,8)$
%2
\item $E(3,3), X(5,8), P(7,3)$ 
%3
\item $E(-1,1), X(-4,-3), P(3,-2) $

\end{enumerate} 
%C
%\item \hspce Find the perimeter and area of the triangles in Part B. 
\end{enumerate} 




}



\def \EvaluationDayD {
\input{/storage/emulated/0/Documents/documents/latex/1920/Grade-10/2nd/distance-formula/qz-distance-formula-input} 
}

\def \RemediationDayD {}

\def \LessonDayE {Equation and Graph of a Circle}

\def \LearningCompetenciesDayE {}

\def \ObjectivesDayE {
\item %cogverbstart
Compare
%cogverbend
the equation and graph of a circle; 
\item %psyverbstart
Find
%psyverbend
the equation and graph of a circle to determine whether a binomial is a factor of a given polynomial; and, 
\item %affverbstart
Display
%affverbend
%valsstart
interest and perseverance
%valsend
in solving problems.
}

\def \PurposeDayE {The purpose of this lesson is to enable the students to solve real life problems involving the equation and graph of a circle.}  

\def \ApplicationDayE { Let the students answer the following questions: 
\begin{enumerate}[label = \arabic*. ]
%1
\item In what real life situations or problems can we observe some examples of equation and graph of a circle? 
%2
\item How can you apply your knowledge of equation and graph of a circle in solving these real life problems? 

\end{enumerate}   
}

\def \GeneralizationDayE {Let the students answer the following questions: 
\begin{enumerate}[label = \arabic*. ]
%1
\item In your own words, what is the equation and graph of a circle? 
%2
\item How do we solve problems involving equation and graph of a circle? 

\end{enumerate}   
}

\def \TeachersGuideDayE {pp. 155--160}

\def \LMPagesDayE {pp. 144--149}

\def \TextbookPagesDayE {pp. 157--163}

\def \AdditionalMaterialsDayE {}

\def \OtherResourcesDayE {Flashcards}

\def \ReviewDayE {\input{/storage/emulated/0/Documents/documents/latex/1920/Grade-10/2nd/equation-and-graph-of-a-circle/vs-equation-and-graph-of-a-circle-input-dll}}



\def \ExamplesDayE {}

\def \PracticeOneDayE {\input{/storage/emulated/0/Documents/documents/latex/1920/Grade-10/2nd/equation-and-graph-of-a-circle/ps-equation-and-graph-of-a-circle-input1-dll}}

\def \PracticeTwoDayE {\input{/storage/emulated/0/Documents/documents/latex/1920/Grade-10/2nd/equation-and-graph-of-a-circle/ps-equation-and-graph-of-a-circle-input2-dll}
}

\def \MasteryDayE {\input{/storage/emulated/0/Documents/documents/latex/1920/Grade-10/2nd/equation-and-graph-of-a-circle/ps-equation-and-graph-of-a-circle-input3-dll}}



\def \EvaluationDayE {
\input{/storage/emulated/0/Documents/documents/latex/1920/Grade-10/2nd/equation-and-graph-of-a-circle/qz-equation-and-graph-of-a-circle-input} 
}

\def \RemediationDayE {}
