\def\curdir{/storage/emulated/0/Documents/documents/latex/1920/Grade-10/2nd/angles-formed-by-secants-and-tangents}
\def\radA{1cm}
\def\radB{1cm}

\textbf{Problem Set}

\vspce 

%\begin{enumerate}[label = \Alph*. ]
%\item \hspce 
A. Use the figure to solve the following. 
\begin{center}
\vspace*{2ex}
\scalebox{0.5}{
\noindent\begin{minipage}{0.3\textwidth}
{
\begin{enumerate}[label = \arabic*. ]
%1
\item If $m\arc{AC}=40\degree $ and $m\arc{BD}=80\degree $, find $m\angle{ATC} $. 
%2
\item If $m\angle{BTC} =142\degree $ and $m\arc{AD}=156\degree $, find $m\arc{BC}$. 
%3
\item If $m\arc{ADB} =208\degree $, $m\arc{AC} =52\degree $ and $m\arc{DBC}=192\degree $, find $m\angle{ATD}$.

\begin{center}

\hspace*{-40pt}\begin{tikzpicture}[dot/.style={circle, fill=black, inner sep=0pt, outer sep=0pt, minimum size=3pt}]

\node(circ) at (0,0) {};

\draw[line width=0.5mm] (circ) circle (\radA);

\coordinate (d) at ($(circ) + (50:\radA)$); 

\coordinate (c) at ($(circ) + (220:\radA)$);

\coordinate (a) at ($(circ) + (175:\radA)$);

\coordinate (b) at ($(circ) + (-40:\radA)$);

\draw[name path=line1, line width=0.3mm, <->, >={Latex[round]}] ($(a)!-35pt!(b) $) -- ($(b)!-0.66*\radA!(a) $);

\draw[name path=line2, line width=0.3mm, <->, >={Latex[round]}] ($(c)!-35pt!(d) $) -- ($(d)!-0.66*\radA!(c) $);

\fill[black, name intersections={of=line1 and line2, name=point}];

\node[dot] (int) at (point-1){};

\node(t.label) at ($(int)+(90:0.22*\radA)$) {$  T$};

\node(d.label) at ($(d)+(90:0.25*\radA)$) {$  D$};

\node(c.label) at ($(c)+(-100:0.25*\radA)$) {$  C$};

\node(a.label) at ($(a)+(120:0.33*\radA)$) {$  A$};

\node(b.label) at ($(b)+(20:0.33*\radA)$) {$  B$};

\end{tikzpicture} 
 

\end{center} 

\end{enumerate} 
}
\end{minipage}}
\end{center} 

%\item \hspce
B. Solve each problem completely.  
\begin{center}
\vspace*{2ex}
\scalebox{0.5}{
\noindent\begin{minipage}{0.3\textwidth}
{
\begin{enumerate}[label = \arabic*. ]
%1
\item The angle formed by two secants intersecting in the exterior of a circle measures $64\degree $. One of the intercepted arcs is $208\degree $. Find the other arc. 
%2
\item In $\odot{Q} $, the endpoints of chord $\overline{OK} $ are the points of tangency of lines $\overleftrightarrow{TO} $ and $\overleftrightarrow{TK}$. If $\triangle{TOK}$ is an equilateral triangle, find $m\arc{OIK}$. 
%3
\item In $\odot{O} $, $\overline{LM} $ is a diameter of $\odot{O} $ where $\overline{OM} $ extends its own length to $P$. If $\overline{PN} $ is a tangent segment to $\odot{O} $ at $N$, find $m\angle{P} $. 

\begin{center}
\begin{tikzpicture}[dot/.style={circle, fill=black, inner sep=0pt, outer sep=0pt, minimum size=3pt}%, remember picture, overlay
]

\node[draw, circle, minimum size=2*\radB,inner sep=0pt, line width=0.5mm, outer sep=0] (q) at (0,0) {};

\node(q.dot)[dot] at (q.center) {};

\node(q.label) at ($(q)+(190:0.22*\radB)$) {$  Q$};

\coordinate (t) at ($(q) + (0:2.2*\radB)$); 

\node(t.label) at ($(t)+(0:0.22*\radB)$) {$  T$};

\node[dot] (i) at ($(q)+(230:\radB)$) {};

\node(i.label) at ($(i)+(230:0.22*\radB)$) {$  I$};

\node[dot] (k) at (tangent cs:node=q, point={(t)}, solution=1) {};

\node(k.label) at ($(k)+(90:0.22*\radB)$) {$  K$};

\node[dot] (o) at (tangent cs:node=q, point={(t)}, solution=2) {};

\node(o.label) at ($(o)+(280:0.22*\radB)$) {$  O$};

\draw[line width=0.3mm, <->, >={Latex[round]}] ($(k)!-\radB!(t)$) -- (t) -- ($(o)!-\radB!(t)$);

\draw[line width=0.3mm] (k) -- (o) ;

\end{tikzpicture}  
 

%\hspace*{-30pt}
\begin{tikzpicture}[dot/.style={circle, fill=black, inner sep=0pt, outer sep=0pt, minimum size=3pt}]

\node[draw, circle, minimum size=2*\radB, inner sep=0pt, line width=0.5mm, outer sep=0] (o) at (0,0) {};

\node(o.dot)[dot] at (o.center) {};

\node(o.label) at ($(o)+(270:0.22*\radB)$) {$  O$};

\coordinate (l) at ($(o) + (-\radB, 0)$); 

\node(l.label) at ($(l)+(180:0.18*\radB)$) {$  L$};

\coordinate (p) at ($(o) + (0:2*\radB)$); 

\node(p.label) at ($(p)+(0:0.22*\radB)$) {$  P$};

\node(m.label) at ($(o)+(0:\radB)+(-40:0.3*\radB)$) {$  M$};

\node[dot] (n) at (tangent cs:node=o, point={(p)}, solution=1) {};

\node(n.label) at ($(n)+(60:0.22*\radB)$) {$  N$};

\draw[line width=0.3mm] (l) -- (p) -- (n) -- (o.center) ;

\begin{scope} [rotate=-120]
\draw[line width=0.3mm] (n) rectangle ++(0.35*\radB,0.35*\radB) node[transform shape]{};
\end{scope}

\end{tikzpicture} 

 

\end{center} 

\end{enumerate} 
}
\end{minipage}}
\end{center} 
%\end{enumerate} 