% cd ~/storage/emulated/0/Documents/documents/latex/1920/Grade-10/2nd/distance-formula/ && pdflatex semi-lp-distance-formula.tex && ula-open semi-lp-distance-formula.pdf


\documentclass[12pt]{article}
\usepackage[legalpaper, portrait, right=0.5in,  left=0.5in, top=0.5in, bottom=1.5in]{geometry}

\usepackage{xcolor}
\usepackage{anyfontsize}
\usepackage{enumitem}
\usepackage{multicol}
\usepackage{amsmath, makecell}
\usepackage{tabularx} 
\usepackage{gensymb}
\usepackage{wasysym} %for checked symbol 
\usepackage{multirow}
\usepackage{graphicx, tipa}
\usepackage{tikz}
\usetikzlibrary{angles,quotes}
\usepackage{pgfplots} 
\usetikzlibrary{calc}
\pgfplotsset{compat=newest}
\usetikzlibrary{arrows.meta}
\usetikzlibrary{intersections}
\usetikzlibrary{decorations.pathreplacing}
\usepackage{flafter}
%\usepackage{fourier} 
\usepackage{amsmath,amssymb,cancel,units}
\usepackage{microtype} % nicer output 
\usepackage{hfoldsty} % nicer output 
\usepackage{fixltx2e} 
\usepackage{mathptmx}
\usepackage{numprint}
\usepackage[T1]{fontenc}
\usepackage[utf8]{inputenc} 
\usepackage{stackengine} %to define \pesos 
\usepackage{lmodern} %scalable font
\usepackage{booktabs}
\usepackage{array}


\pagenumbering{gobble}
%\linespread{0.9}
\newcommand{\vspce}{\vspace{0.75ex}}

\newcommand{\hspce}{\hspace{0.5em}}

\newcommand{\blank}{\underline{\hspace{2em}}}%{\rule{1em}{0.15ex}}

\newcommand{\arc}[1]{{% 
\setbox9=\hbox{#1}% 
\ooalign{\resizebox{\wd9}{\height}{\texttoptiebar{\phantom{A}}}\cr#1}}}


\newcommand\pesos{\stackengine{-1.4ex}{P}{\stackengine{-1.25ex}{$-$}{$-$}{O}{c}{F}{F}{S}}{O}{c}{F}{T}{S}} 


\renewcommand\theadalign{bc} 

\renewcommand\theadfont{\bfseries} 

\renewcommand\theadgape{\Gape[4pt]} 

\renewcommand\cellgape{\Gape[4pt]} 

\pagenumbering{gobble}

\newcolumntype{Y}{>{\centering\arraybackslash}X} %for tabularx

\newcolumntype{R}{>{\raggedleft\arraybackslash}X} %for tabularx

\newcolumntype{Z}{>{\raggedleft\arraybackslash}X} %for tabularx

\newcolumntype{L}{>{\raggedright\arraybackslash}X} %for tabularx

\newcolumntype{A}[1]{>{\raggedright\arraybackslash}p{#1}} %for longtable LEFT

\newcolumntype{C}[1]{>{\centering\arraybackslash}p{#1}} %for longtable CENTER

\newcolumntype{B}[1]{>{\raggedleft\arraybackslash}p{#1}} %for longtable RIGHT 
 
\renewcommand{\tabularxcolumn}[1]{>{\small}m{#1}}

\newcolumntype{N}[1]{>{\raggedleft}p{#1}} %for tabular left 

\newcolumntype{M}[1]{>{\raggedright\arraybackslash}p{#1}} %for tabular right 

\newcommand{\myaxis}{xticklabels={}, 
yticklabels={}, 
ymin=-10, ymax=10,
xmin=-10, xmax=10,
axis lines = center, 
inner axis line style={Latex-Latex,very thick}, 
grid=both,
minor tick num=4, 
tick align=inside} % grid without labels, origin at the center, 10 units from origin

\newcommand{\axisfive}{xticklabels={}, 
yticklabels={}, 
ymin=-5, ymax=5,
xmin=-5, xmax=5,
axis lines = center, 
inner axis line style={Latex-Latex,very thick}, 
grid=both,
minor tick num=1, 
tick align=inside} % grid with labels, origin at the center, 5 units from origin 

\newcommand \redcheck {{\color{red}\checkmark}}

 

\def \Grade {10
}

\def \Teacher {
Mr. Jonathan R. Bacolod, LPT 
}

\def \Module {
Geometry 
}

\def \Date {
Oct 8, 2019 
}

\def \ContentStandard {
The learner demonstrates understanding of key concepts of circles and coordinate geometry.   
}

\def \PerformanceStandard {
The learner is able to formulate and find solutions to challenging situations involving circles and other related terms in different disciplines through appropriate and accurate representations.
}

\def \LearningCompetency {
The learner applies the distance formula to prove some geometric properties. (M10GE-IIg-2)
}

\def \Objectives {
\item Describe the distance formula; 
\item Prove some geometric properties using the distance formula; and, 
\item Show interest and perseverance in solving problems. 
}

\def \Generalization {
\item[a. ] In your own words, what is the distance formula?    
\item[b. ] How do we solve problems involving the  distance formula? 
}


\def \Topic {
Distance Formula 
}

\def \ReviewTopic {
Area of Sectors and Segments of a Circle   
}

\def \Motivation {
}

\def \Reference {
Mathematics \Grade Learner’s Module pp. 229-237
}

\def \Materials {
Handouts showing the steps and the formula for the \Topic
}

\def \Checker {
\underline{\textbf{DR. LORETO R. DOMINGO}}\\
OIC/MT II Mathematics Department
}

\def \DirectInstruction {
\begin{center}
\textbf{The Distance Formula 
}
\end{center}

\vspce 

If $A(x_1, y_1)$ and $B(x_2, y_2)$ are any two points on the coordinate plane, then
\[
AB = \sqrt{(x_2-x_1)^2 + (y_2-y_1)^2}
\] 


}

\def \PE {
\textbf{Practice Exercises}
%\textbf{Problem Set}

\vspce
%\begin{enumerate}[label = \Alph*. ]
%A
%\item \hspce 
A. Find $PQ$. 
\begin{enumerate}[label = \arabic*. ]
%1
\item $P(4, 2), Q(8, 2)$
%2
\item $P(3, 5), Q(3, -2)$
%3
\item $P(0, 0), Q(-4, 3)$
%4
\item $P(-2, 6), Q(-7, 7)$
%5
\item $P(5, 2), Q(0,-6)$
\end{enumerate} 

%B
%\item \hspce 
B. Find the  length  of each side of $\triangle EXP$. Tell whether $\triangle EXP$ is isosceles, right, or neither. 
\begin{enumerate}[label = \arabic*. ]
%1
\item $E(4,3), X(-1,1), P(5,0)$ 
%2
\item $E(-3,-2), X(1,-1), P(0,2)$ 
%3
\item $E(0,8), X(9,6), P(8,10) $

\end{enumerate} 
%C
%\item \hspce
%\end{enumerate} 




 
}

\def \PS {
%\textbf{Practice Exercises}
\textbf{Problem Set}

\vspce
\begin{enumerate}[label = \Alph*. ]
%A
\item \hspce Find $PQ$. 
\begin{enumerate}[label = \arabic*. ]
%1
\item $P(-3, 1), Q(3, 9)$
%2
\item $P(4, 2), Q(9, 14)$
%3
\item $P(-8, 0), Q(16, -24)$
%4
\item $P(-3, 4), Q(5, 2)$
%5
\item $P(4, 5), Q(1,3)$
\end{enumerate} 

%B
\item \hspce Find the  length  of each side of $\triangle EXP$. Tell whether $\triangle EXP$ is isosceles, right, or neither. 
\begin{enumerate}[label = \arabic*. ]
%1
\item $E(1,8)$, $X(6,-4)$, $P(11,8)$
%2
\item $E(3,3), X(5,8), P(7,3)$ 
%3
\item $E(-1,1), X(-4,-3), P(3,-2) $

\end{enumerate} 
%C
%\item \hspce Find the perimeter and area of the triangles in Part B. 
\end{enumerate} 





}


\begin{document}

\noindent \begin{tabularx}{\textwidth}{LYR}

\multirow[c]{3}{15em}{
\begin{tikzpicture}[remember picture, overlay]
\node (shs) at (180:-2em) {\includegraphics[width=0.75in]{/host-rootfs/storage/emulated/0/Documents/documents/latex/1920/shs.png}};
\end{tikzpicture}
}
&
\textbf{SAUYO HIGH SCHOOL} 
&
\multirow[c]{3}{15em}{
\begin{tikzpicture}[remember picture, overlay]
\node (deped) at (0:10.5em) {\includegraphics[width=0.75in]{/host-rootfs/storage/emulated/0/Documents/documents/latex/1920/deped2.png}};
\end{tikzpicture}
}
\\

& MATHEMATICS DEPARTMENT & \\

& S.Y. 2019 -- 2020 & \\

& \textbf{Lesson Plan for Mathematics \Grade}&\\
\end{tabularx}

\vspace*{2em}

\noindent \begin{tabularx}{\textwidth}{LR}
Module title: \Module
&
Grade Level: Grade \Grade 
\\
Date: \Date
& 
Designed by: \Teacher 
\\
\end{tabularx}

\vspace*{2.5ex}

\begin{enumerate}[label = \textbf{\Roman*. }]
\item \textbf{Learning Competencies/Objectives }

\begin{enumerate}[label = \Alph*. ]
%1
\item Content Standard: \ContentStandard
%2
\item Performance Standard: \PerformanceStandard
%3
\item Learning Competency: \LearningCompetency
\end{enumerate}   

At the end of a 50-minute period, 80\% of the Grade \Grade students should be able to do the following with at least 75\% accuracy:
	\begin{enumerate}[label = \alph*. ]
	\Objectives 
	\end{enumerate}
	
\item \textbf{Subject Matter}
	\begin{enumerate}[label = \Alph*. ]
	\item Topic: \Topic
	\item Reference: \Reference
	\item Materials: \Materials
	\end{enumerate}

\item	 \textbf{Procedure}
	\begin{enumerate}[label = \Alph*. ]
	\item Daily routine
		\begin{enumerate}[label = \arabic*. ]
		\item Cleaning and arranging of chairs
		\item Greeting 
		\item Checking of assignment 
		\item Drill: Flashcards showing the operations on signed numbers 
		\item Review: \ReviewTopic
	%	\item Motivation: \Motivation
		\end{enumerate}
		
	\item Lesson Proper 
		\begin{enumerate}[label = \arabic*. ]
		\item Direct instruction: The teacher describes the main concepts of the lesson. \DirectInstruction
		\item Demonstration: The teacher shows how to solve the first item in the Practice Exercises. 
		\item Practice Exercises and Boardwork: \\%(See at the end.) 
		Answer the following problems in your notebook. \\
		{\PE}
			
		\item Generalization: 	Let the students answer the following questions. 
\begin{enumerate}[label = \arabic*. ]
\Generalization
\end{enumerate}

		\end{enumerate}

%\newpage

	\item Application: Problem Set\\ %(See at the end.) 
	In a sheet of paper, answer the following problems. \\
	{\PS}%\\

	\end{enumerate}

\end{enumerate}

%\PE 

%\vspace{1ex}
%\newpage

%\PS

\vfill

\begin{flushright}
Prepared by:\\*[2.5ex]
\Teacher \\
Teacher I
\end{flushright} 

\vspace*{3.5ex}

\begin{flushright}
Checked by:\\*[2.5ex]
\Checker
\end{flushright} 
	 

\end{document}