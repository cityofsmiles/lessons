\begin{center}
\textbf{Probability of Simple Event
}
\end{center}

\vspace*{1ex}

Probability: a measure that is associated with how certain we are of outcomes of a particular experiment or activity

\vspce

Experiment: a planned operation carried out under controlled conditions

\vspce

Chance Experiment: an experiment whose result is not predetermined

\vspce

Outcome: a result of an experiment 

\vspce

Sample Space: the list of all the possible outcomes of an experiment

\vspce

Ways to Represent a Sample Space: 
\begin{enumerate}[label = \arabic*. ]
%1
\item List the possible outcomes
%2
\item Create a tree diagram
%3
\item Create a Venn
diagram

\end{enumerate}  

\vspce

Event: any combination of outcomes

\vspce

Equally likely: each outcome of an experiment occurs with equal probability

\vspce

If an event $E$ has $n(E)$ equally likely outcomes and its sample space $S$ has $n(S)$ likely outcomes, then the \emph{probability} of the event $E$ is: 
\[
P(E) = \displaystyle \frac{n(E) }{n(S)} = \displaystyle \frac{\text{number of elements in } E}{\text{number of elements in }S} 
\] 

\vspace*{1.5ex}

\emph{Properties of Probability} 
\begin{enumerate}[label = \arabic*. ]
%1
\item A probability is a number between 0 and 1, inclusive. \\
The closer the probability of an event to 1, the more likely the event is to happen and the closer the probability of an event to zero, the less likely it is to happen.
%2
\item The probability of an event that cannot happen is 0.
%3
\item The probability of an event that must happen is 1.
%4
\item If the probability of an event $E$ is $P$, then the probability of the complement of $E$ is $1-P$.

\end{enumerate} 


