 

\def \LessonDayA {Factorial Notation}

\def \LearningCompetenciesDayA {}

\def \ObjectivesDayA {
\item %cogverbstart
Define
%cogverbend
the factorial notation; 
\item %psyverbstart
Find
%psyverbend
the factorial notation to determine whether a binomial is a factor of a given polynomial; and, 
\item %affverbstart
Demonstrate
%affverbend
%valsstart
perseverance and willingness
%valsend
in solving problems.
}

\def \PurposeDayA {The purpose of this lesson is to enable the students to solve real life problems involving the factorial notation.}  

\def \ApplicationDayA { Let the students answer the following questions: 
\begin{enumerate}[label = \arabic*. ]
%1
\item In what real life situations or problems can we observe some examples of factorial notation? 
%2
\item How can you apply your knowledge of factorial notation in solving these real life problems? 

\end{enumerate}   
}

\def \GeneralizationDayA {Let the students answer the following questions: 
\begin{enumerate}[label = \arabic*. ]
%1
\item In your own words, what is the factorial notation? 
%2
\item How do we solve problems involving factorial notation? 

\end{enumerate}   
}

\def \TeachersGuideDayA {pp. 290--296}

\def \LMPagesDayA {pp. 275--281}

\def \TextbookPagesDayA {pp. 303--309}

\def \AdditionalMaterialsDayA {}

\def \OtherResourcesDayA {Flashcards}

\def \ReviewDayA {\begin{center}
\textbf{Factorial Notation 
}
\end{center}

\vspce 

%\begin{center}
%\vspace*{-2ex}
\scalebox{0.5}{
\noindent\begin{minipage}{0.3\textwidth}
{
 
n-Factorial: the product of the positive integer $n$ and all the positive integers less than $n$. 
 
\vspce 
For any natural number $n$,   
\[
n! = n(n-1)(n-2)\ldots (3)\cdot(2)\cdot(1). 
\]

\vspce 

For the number 0,
\[
0!=1.
\] 

}
\end{minipage}}
%\end{center} }



\def \ExamplesDayA {}

\def \PracticeOneDayA {\textbf{Practice Exercises}
%\textbf{Problem Set}

\vspce


%A
A. Evaluate. 

\begin{enumerate}[label = \arabic*. ]

\begin{multicols}{3}

%1
\item \hspce $6!$
%2
\item \hspce $4!+5!$
%3
\item \hspce $9!-4!$
%4
\item \hspce $7!-5!$
%5 
\item \hspce  $\displaystyle \frac{6!}{4!3!} $

\end{multicols} 

\end{enumerate}  
%B
B. Simplify  by factorization. 
\begin{enumerate}[label = \arabic*. ]
\begin{multicols}{3}
%1
\item \hspce $\displaystyle \frac{7!-6!}{6} $
%2
\item \hspce $\displaystyle \frac{7!-6!}{6!} $
%3
\item \hspce $\displaystyle \frac{6!+4!}{31} $
%4
\item \hspce $\displaystyle \frac{8!-6!}{55} $
%5 
\item \hspce  $\displaystyle \frac{7!+6!-5!}{5!} $

\end{multicols} 
\end{enumerate}  

}

\def \PracticeTwoDayA {%C

C. Simplify the following. 
\begin{enumerate}[label = \arabic*. ]
\begin{multicols}{3}
%1
\item \hspce $\displaystyle \frac{n!}{n} $
%2
\item \hspce $\displaystyle \frac{(n+1)!}{n!} $
%3
\item \hspce $\displaystyle \frac{(n+2)!}{n!} $
%4
\item \hspce $\displaystyle \frac{(n+1)!}{(n-1)!} $
%5 
\item \hspce  $\displaystyle \frac{(n+1)!+n!-(n-1)!}{(n-1)!} $

\end{multicols} 
\end{enumerate}  
}

\def \MasteryDayA {\input{/storage/emulated/0/Documents/documents/latex/1920/Grade-10/3rd/factorial-notation/ps-factorial-notation-input3-dll}}



\def \EvaluationDayA {
%\input{/storage/emulated/0/Documents/documents/latex/1920/Grade-10/3rd/factorial-notation/qz-factorial-notation-input} 
}

\def \RemediationDayA {}

\def \LessonDayB {Fundamental Principle of Counting}

\def \LearningCompetenciesDayB {}

\def \ObjectivesDayB {
\item %cogverbstart
Distinguish
%cogverbend
the fundamental principle of counting; 
\item %psyverbstart
Calculate
%psyverbend
the fundamental principle of counting to determine whether a binomial is a factor of a given polynomial; and, 
\item %affverbstart
Project
%affverbend
%valsstart
enjoyment and willingness
%valsend
in solving problems.
}

\def \PurposeDayB {The purpose of this lesson is to enable the students to solve real life problems involving the fundamental principle of counting.}  

\def \ApplicationDayB { Let the students answer the following questions: 
\begin{enumerate}[label = \arabic*. ]
%1
\item In what real life situations or problems can we observe some examples of fundamental principle of counting? 
%2
\item How can you apply your knowledge of fundamental principle of counting in solving these real life problems? 

\end{enumerate}   
}

\def \GeneralizationDayB {Let the students answer the following questions: 
\begin{enumerate}[label = \arabic*. ]
%1
\item In your own words, what is the fundamental principle of counting? 
%2
\item How do we solve problems involving fundamental principle of counting? 

\end{enumerate}   
}

\def \TeachersGuideDayB {pp. 297--304}

\def \LMPagesDayB {pp. 282--289}

\def \TextbookPagesDayB {pp. 310--317}

\def \AdditionalMaterialsDayB {}

\def \OtherResourcesDayB {Flashcards}

\def \ReviewDayB {\begin{center}
\textbf{The Fundamental Principle of Counting 
}
\end{center}

\vspce 


Fundamental Principle of Counting: If one thing can occur in $m$ ways and a second thing can occur in $n$ ways, and a third thing can occur in $p$ ways, and so on, then the sequence of things can occur in $m \times n \times p \times \ldots$ ways.

\vspce

}



\def \ExamplesDayB {}

\def \PracticeOneDayB {\textbf{Practice Exercises}
%\textbf{Problem Set}

\vspce

Find the number of possible outcomes for each scenario using the fundamental counting principle. 
\begin{enumerate}[label = \arabic*. ]
%1
\item Boys and girls in a family with two children.
%2
\item Choosing a cellphone that comes in black, white, or transparent that is 3G or 4G.
%3
\item A choice of muffin or toast bread with coffee, milk, or juice.
%4
\item Basketball uniform in white, red, blue, yellow, or green which comes in sizes small, medium, or large.
%5
\item A die is rolled thrice.

\end{enumerate}   
   }

\def \PracticeTwoDayB {\input{/storage/emulated/0/Documents/documents/latex/1920/Grade-10/3rd/fundamental-principle-of-counting/ps-fundamental-principle-of-counting-input2-dll}
}

\def \MasteryDayB {\input{/storage/emulated/0/Documents/documents/latex/1920/Grade-10/3rd/fundamental-principle-of-counting/ps-fundamental-principle-of-counting-input3-dll}}



\def \EvaluationDayB {
%\input{/storage/emulated/0/Documents/documents/latex/1920/Grade-10/3rd/fundamental-principle-of-counting/qz-fundamental-principle-of-counting-input} 
}

\def \RemediationDayB {}

\def \LessonDayC {Permutation}

\def \LearningCompetenciesDayC {}

\def \ObjectivesDayC {
\item %cogverbstart
Illustrate
%cogverbend
the permutation; 
\item %psyverbstart
Compute
%psyverbend
the permutation to determine whether a binomial is a factor of a given polynomial; and, 
\item %affverbstart
Exhibit
%affverbend
%valsstart
enjoyment and self-reliance
%valsend
in solving problems.
}

\def \PurposeDayC {The purpose of this lesson is to enable the students to solve real life problems involving the permutation.}  

\def \ApplicationDayC { Let the students answer the following questions: 
\begin{enumerate}[label = \arabic*. ]
%1
\item In what real life situations or problems can we observe some examples of permutation? 
%2
\item How can you apply your knowledge of permutation in solving these real life problems? 

\end{enumerate}   
}

\def \GeneralizationDayC {Let the students answer the following questions: 
\begin{enumerate}[label = \arabic*. ]
%1
\item In your own words, what is the permutation? 
%2
\item How do we solve problems involving permutation? 

\end{enumerate}   
}

\def \TeachersGuideDayC {pp. 305--310}

\def \LMPagesDayC {pp. 290--295}

\def \TextbookPagesDayC {pp. 318--323}

\def \AdditionalMaterialsDayC {}

\def \OtherResourcesDayC {Flashcards}

\def \ReviewDayC {\input{/storage/emulated/0/Documents/documents/latex/1920/Grade-10/3rd/permutation/vs-permutation-input-dll}}



\def \ExamplesDayC {}

\def \PracticeOneDayC {\input{/storage/emulated/0/Documents/documents/latex/1920/Grade-10/3rd/permutation/ps-permutation-input1-dll}}

\def \PracticeTwoDayC {\input{/storage/emulated/0/Documents/documents/latex/1920/Grade-10/3rd/permutation/ps-permutation-input2-dll}
}

\def \MasteryDayC {%\textbf{Practice Exercises}
\textbf{Problem Set}

\vspce

Solve each  permutation  problem completely. 

\begin{enumerate}[label = \arabic*. ]
\item How many 4-digit  numbers  can  be  formed  from  the  digits  1, 3, 5, 6,  8, and  9 if  no repetition is  allowed?
%2
\item If there  are  10  people  and  only  6 chairs  are  available,  in  how  many ways  can they  be seated?
\item In how many different ways can a president, vice president, a secretary, and a treasurer  be chosen from a class of 15 students? 

\item In how many different ways can a first,  second,  and third prizes be awarded in a game with  eight contestants? 

\item If four persons enter a bus on which there are ten vacant seats, how many ways  can the four be seated?

\end{enumerate} 





}



\def \EvaluationDayC {
%\input{/storage/emulated/0/Documents/documents/latex/1920/Grade-10/3rd/permutation/qz-permutation-input} 
}

\def \RemediationDayC {}

\def \LessonDayD {Distinguishable Permutation}

\def \LearningCompetenciesDayD {}

\def \ObjectivesDayD {
\item %cogverbstart
Describe
%cogverbend
the distinguishable permutation; 
\item %psyverbstart
Compute
%psyverbend
the distinguishable permutation to determine whether a binomial is a factor of a given polynomial; and, 
\item %affverbstart
Display
%affverbend
%valsstart
enjoyment and interest
%valsend
in solving problems.
}

\def \PurposeDayD {The purpose of this lesson is to enable the students to solve real life problems involving the distinguishable permutation.}  

\def \ApplicationDayD { Let the students answer the following questions: 
\begin{enumerate}[label = \arabic*. ]
%1
\item In what real life situations or problems can we observe some examples of distinguishable permutation? 
%2
\item How can you apply your knowledge of distinguishable permutation in solving these real life problems? 

\end{enumerate}   
}

\def \GeneralizationDayD {Let the students answer the following questions: 
\begin{enumerate}[label = \arabic*. ]
%1
\item In your own words, what is the distinguishable permutation? 
%2
\item How do we solve problems involving distinguishable permutation? 

\end{enumerate}   
}

\def \TeachersGuideDayD {pp. 311--318}

\def \LMPagesDayD {pp. 296--303}

\def \TextbookPagesDayD {pp. 324--331}

\def \AdditionalMaterialsDayD {}

\def \OtherResourcesDayD {Flashcards}

\def \ReviewDayD {\input{/storage/emulated/0/Documents/documents/latex/1920/Grade-10/3rd/distinguishable-permutation/vs-distinguishable-permutation-input-dll}}



\def \ExamplesDayD {}

\def \PracticeOneDayD {\textbf{Practice Exercises}
%\textbf{Problem Set}

\vspce

%\begin{enumerate}[label = \Alph*. ]
%A
%\item \hspce 
A. Find the number of distinguishable permutations for the following. 

\begin{enumerate}[label = \arabic*. ]
\begin{multicols}{3}

%1
\item ALAPAAP
%2
\item MAGSASAKA
%3
\item HIMPAPAWID
%4
\item PALAYAN
%5 
\item BINIBINI

\end{multicols} 
\end{enumerate}  

%\vspce 

%B
%\item \hspce 


%\end{enumerate}  

 



}

\def \PracticeTwoDayD {\input{/storage/emulated/0/Documents/documents/latex/1920/Grade-10/3rd/distinguishable-permutation/ps-distinguishable-permutation-input2-dll}
}

\def \MasteryDayD {\input{/storage/emulated/0/Documents/documents/latex/1920/Grade-10/3rd/distinguishable-permutation/ps-distinguishable-permutation-input3-dll}}



\def \EvaluationDayD {
%\input{/storage/emulated/0/Documents/documents/latex/1920/Grade-10/3rd/distinguishable-permutation/qz-distinguishable-permutation-input} 
}

\def \RemediationDayD {}

\def \LessonDayE {Circular Permutations}

\def \LearningCompetenciesDayE {}

\def \ObjectivesDayE {
\item %cogverbstart
Illustrate
%cogverbend
the circular permutations; 
\item %psyverbstart
Calculate
%psyverbend
the circular permutations to determine whether a binomial is a factor of a given polynomial; and, 
\item %affverbstart
Demonstrate
%affverbend
%valsstart
self-reliance and independence
%valsend
in solving problems.
}

\def \PurposeDayE {The purpose of this lesson is to enable the students to solve real life problems involving the circular permutations.}  

\def \ApplicationDayE { Let the students answer the following questions: 
\begin{enumerate}[label = \arabic*. ]
%1
\item In what real life situations or problems can we observe some examples of circular permutations? 
%2
\item How can you apply your knowledge of circular permutations in solving these real life problems? 

\end{enumerate}   
}

\def \GeneralizationDayE {Let the students answer the following questions: 
\begin{enumerate}[label = \arabic*. ]
%1
\item In your own words, what is the circular permutations? 
%2
\item How do we solve problems involving circular permutations? 

\end{enumerate}   
}

\def \TeachersGuideDayE {pp. 319--324}

\def \LMPagesDayE {pp. 304--309}

\def \TextbookPagesDayE {pp. 332--337}

\def \AdditionalMaterialsDayE {}

\def \OtherResourcesDayE {Flashcards}

\def \ReviewDayE {\input{/storage/emulated/0/Documents/documents/latex/1920/Grade-10/3rd/circular-permutations/vs-circular-permutations-input-dll}}



\def \ExamplesDayE {}

\def \PracticeOneDayE {\input{/storage/emulated/0/Documents/documents/latex/1920/Grade-10/3rd/circular-permutations/ps-circular-permutations-input1-dll}}

\def \PracticeTwoDayE {\input{/storage/emulated/0/Documents/documents/latex/1920/Grade-10/3rd/circular-permutations/ps-circular-permutations-input2-dll}
}

\def \MasteryDayE {\input{/storage/emulated/0/Documents/documents/latex/1920/Grade-10/3rd/circular-permutations/ps-circular-permutations-input3-dll}}



\def \EvaluationDayE {
%\input{/storage/emulated/0/Documents/documents/latex/1920/Grade-10/3rd/circular-permutations/qz-circular-permutations-input} 
}

\def \RemediationDayE {}
