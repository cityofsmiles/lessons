\begin{questions}

%1
\question What is the leading coefficient of the polynomial function $f(x) = 2x + x^3 + 4$?

\begin{multicols}{4}
\begin{choices} 
\CorrectChoice 1
\choice 2
\choice 3
\choice 4
\end{choices}
\end{multicols}

%2
\question An angle formed by two rays whose vertex is the center of a circle is called: 

\begin{multicols}{4}
\begin{choices}  
\choice Acute angle
\CorrectChoice Central angle
\choice Inscribed angle
\choice  Obtuse angle
\end{choices}
\end{multicols}

%3
\question The points where the graph intersects the x-axis are called: 
\begin{multicols}{4}
\begin{choices}  
\choice Bounds
\choice Turning points
\CorrectChoice x-intercepts
\choice  y-intercepts 
\end{choices}
\end{multicols}

%4
\question Which of the following characteristics of the polynomial function $y = x^3 + 3x^4 - x^5 - 7x^2 + 4$ is correct? 
%\begin{multicols}{2}
\begin{choices}  
\choice The leading coefficient is positive and the degree is even. 
\choice The leading coefficient is positive and the degree is odd.
\choice The leading coefficient is negative and the degree is even.
\CorrectChoice   The leading coefficient is negative and the degree is odd.
\end{choices}
%\end{multicols}

%5
\question Which term determines how many times a particular number is a zero or root for a given polynomial? 

\begin{multicols}{4}
\begin{choices}  
\choice Bound
\choice Intercept
\CorrectChoice Multiplicity
\choice  Turning point
\end{choices}
\end{multicols} 

%6
\question What should $n$ be if $f(x) = x^n$ defines a polynomial function? 
\begin{multicols}{2}
\begin{choices}  
\choice an integer
\CorrectChoice a nonnegative integer
\choice any number
\choice any number except 0
\end{choices}
\end{multicols} 

%7
\question Which of the following occurs  when  the function  changes  from  decreasing  to  increasing  or  from  increasing  to decreasing  values? 

\begin{multicols}{4}
\begin{choices}  
\choice Bound
\choice Intercept
\choice Multiplicity
\CorrectChoice Turning point
\end{choices}
\end{multicols} 

%8 
\question What is an angle whose vertex is on a circle and whose sides contain chords of the circle?  
\begin{multicols}{2}
\begin{choices} 
\CorrectChoice inscribed angle
\choice intercepted angle
\choice central angle
\choice circumscribed angle
\end{choices}
\end{multicols} 

%9
\question An arc of a circle measures 30\degree. If the radius of the circle is 5 cm, what is the length of the arc? 
\begin{multicols}{4}
\begin{choices} 
\choice 2.62 cm 
\choice 2.3 cm 
\choice 1.86 cm
\choice 1.5 cm
\end{choices}
\end{multicols} 

%10
\question The opposite angles of a quadrilateral inscribed in a circle are \blank.  
\begin{multicols}{4}
\begin{choices}  
\choice right
\choice obtuse
\CorrectChoice supplementary
\choice complementary
\end{choices}
\end{multicols} 

%11
\question In a circle, if a central angle measures $60\degree$, what is the measure of its intercepted arc? 

\begin{multicols}{4}
\begin{choices}  
\choice $30\degree$ 
\CorrectChoice $60\degree$ 
\choice $120\degree$ 
\choice $300\degree$ 
\end{choices}
\end{multicols} 

%12
\question A dart board has a diameter of 40 cm and is divided into 20 congruent       sectors. What is the area of one of the sectors?    
\begin{multicols}{4}
\begin{choices}  
\choice $20\pi \hspce cm^2$
\choice $40\pi \hspce cm^2$
\choice $60\pi \hspce cm^2$
\choice $80\pi \hspce cm^2$
\end{choices}
\end{multicols} 

%13
\question What is the y-intercept of the graph of the polynomial function $f(x) = -2x + x^3 + 3x^5 -4$?

\begin{multicols}{4}
\begin{choices}  
\choice 4
\choice 2
\choice 0
\CorrectChoice -4
\end{choices}
\end{multicols} 

%14
\question How many turning points does the polynomial function $f(x) = -2x + x^3 + 3x^5 - 4$ have?

\begin{multicols}{4}
\begin{choices}  
\choice 2
\choice 3
\CorrectChoice 4
\choice 5
\end{choices}
\end{multicols} 

%15
\question Your classmate Linus encounters difficulties in showing a sketch of the graph of  $y = 2x^3 + 3x^2 - 4x - 6$. What hint/clue should you give? 

%\begin{multicols}{4}
\begin{choices} 
\CorrectChoice The graph falls to the left and rises to the right. 
\choice The graph rises to the left and falls to the right.
\choice The graph rises to both left and right. 
\choice The graph falls to both left and right. 
\end{choices}
%\end{multicols} 

%16
\question Which of the following could be the graph of the polynomial function $y = -x^3 + 2x^2 + x - 2$?

\begin{multicols}{4}
\begin{choices}  
\choice \begin{tikzpicture}[scale=\gridscale]

\begin{axis}[\myaxis] 

\addplot[<->, >={Latex[round]},  ultra thick, domain=-3:1.8, samples=200]{(x+2)*(x+1)*(x-1)}node[]{};

\end{axis}
 
\end{tikzpicture}  
\CorrectChoice \begin{tikzpicture}[scale=\gridscale]

\begin{axis}[\myaxis] 

\addplot[<->, >={Latex[round]},  ultra thick, domain=-1.75:2.9, samples=200]{-(x+1)*(x-1)*(x-2)}node[]{};

\end{axis}
 
\end{tikzpicture} 
\choice \begin{tikzpicture}[scale=\gridscale]

\begin{axis}[\myaxis] 

\addplot[<->, >={Latex[round]},  ultra thick, domain=-2.5:1.5, samples=200]{x*(x+2)*(x+1)*(x-1)}node[]{};

\end{axis}
 
\end{tikzpicture} 
\choice \begin{tikzpicture}[scale=\gridscale]

\begin{axis}[\myaxis] 

\addplot[<->, >={Latex[round]},  ultra thick, domain=-1.6:2.6, samples=200]{x*(-x-1)*(x-1)*(x-2)}node[]{};

\end{axis}
 
\end{tikzpicture} 
\end{choices}
\end{multicols} 

%17
\question What are the end behaviors of the graph of $f(x) = -2x + x^3 + 3x^5 - 4$?

\begin{multicols}{2}
\begin{choices}  
\choice rises to the left and falls to the right
\CorrectChoice falls to the left and rises to the right
\choice rises to both directions 
\choice falls to both directions
\end{choices}
\end{multicols} 

%18
\question Which polynomial function in factored form represents the given graph?
%\begin{multicols}{1}
\begin{choices}  
\choice $y = (x+2)(x+1)(x-1)$ 
\choice $y = (x-2)(x+1)(x-1)$ 
\CorrectChoice $y = x(x+2)(x+1)(x-1)$ 
\choice $y = x(x-2)(x+1)(x-1)$ \hspace*{8cm} \begin{tikzpicture}[scale=\gridscale, 
%baseline = (current bounding box.west), 
remember picture, overlay]

\begin{axis}[\myaxis] 

\addplot[<->, >={Latex[round]},  ultra thick, domain=-2.5:1.5, samples=200]{x*(x+2)*(x+1)*(x-1)}node[]{};

\end{axis}
 
\end{tikzpicture} 
\end{choices}
%\end{multicols} 

%19
\question An arc with a measure equal to one-half the circumference of a circle is called: 

\begin{multicols}{4}
\begin{choices}  
\choice Intercepted arc
\choice Major arc
\choice Minor arc
\CorrectChoice Semicircle
\end{choices}
\end{multicols} 

%20
\question If $m\arc{AC}=40\degree $ and $m\arc{BD}=80\degree $, find $m\angle{ATC}$.
%\begin{multicols}{4}
\begin{choices} 
\choice 40\degree 
\CorrectChoice 60\degree 
\choice 80\degree 
\choice 120\degree 
\end{choices}
%\end{multicols} 
\vspace*{-3cm}\hspace*{9cm}\begin{tikzpicture}[dot/.style={circle, fill=black, inner sep=0pt, outer sep=0pt, minimum size=3pt}]

\node(circ) at (0,0) {};

\draw[line width=0.5mm] (circ) circle (\radA);

\coordinate (d) at ($(circ) + (50:\radA)$); 

\coordinate (c) at ($(circ) + (220:\radA)$);

\coordinate (a) at ($(circ) + (175:\radA)$);

\coordinate (b) at ($(circ) + (-40:\radA)$);

\draw[name path=line1, line width=0.3mm, <->, >={Latex[round]}] ($(a)!-35pt!(b) $) -- ($(b)!-0.66*\radA!(a) $);

\draw[name path=line2, line width=0.3mm, <->, >={Latex[round]}] ($(c)!-35pt!(d) $) -- ($(d)!-0.66*\radA!(c) $);

\fill[black, name intersections={of=line1 and line2, name=point}];

\node[dot] (int) at (point-1){};

\node(t.label) at ($(int)+(90:0.22*\radA)$) {$  T$};

\node(d.label) at ($(d)+(90:0.25*\radA)$) {$  D$};

\node(c.label) at ($(c)+(-100:0.25*\radA)$) {$  C$};

\node(a.label) at ($(a)+(120:0.33*\radA)$) {$  A$};

\node(b.label) at ($(b)+(20:0.33*\radA)$) {$  B$};

\end{tikzpicture} 
 

%21
\question Find the area of the shaded region in the following figure. 
%\begin{multicols}{4}
\begin{choices} 
\choice $24\pi$
\choice $36\sqrt{3}$
\CorrectChoice $24\pi - 36\sqrt{3}$
\choice $12\pi - 18\sqrt{3}$
\end{choices}
%\end{multicols} 

\vspace*{-2.7cm}\hspace*{11cm}\begin{tikzpicture}[baseline = (current bounding box.west)]

\node[draw, circle, minimum size=2*\radB, inner sep=0pt, line width=0.5mm, outer sep=0] (circ) at (0,0) {};

\node[anchor=south, inner sep=2pt, rotate=-30] (12cm.label) at ($(circ.center)!0.5!(-30:\radB)$) {$ \text{12 cm}$};

\node[anchor=north west, inner sep=2pt, rotate=0] (60.label) at ($(circ.center)+(-60:\radB+0pt)$) {$ 60 \degree $};

\filldraw[fill=\figurefill, line width=0.3mm]  (-30:\radB) arc (-30:-90:\radB) -- cycle;

\draw[line width=0.3mm] (-30:\radB) -- (-90:\radB) -- (circ.center) -- cycle;

\end{tikzpicture} 

 
%22
%\newpage 
\question Find the value of $x$ in the following figure. 
%\begin{multicols}{4}
\begin{choices}  
\choice $\sqrt{2} $ 
\choice $10\sqrt{2} $ 
\choice $10$ 
\CorrectChoice $5 \sqrt{2} $ 
\end{choices}
%\end{multicols} 

\vspace*{-3cm}\hspace*{11cm}\begin{tikzpicture}[dot/.style={circle, fill=black, inner sep=0pt, outer sep=0pt, minimum size=3pt}, baseline = (current bounding box.west)]

\node[draw,circle,minimum size=2*\radB,inner sep=0pt, line width=0.5mm, outer sep=0] (circ) at (0,0) {};

\coordinate (g) at ($(circ) + (-80:\radB)$); 

\coordinate (n) at ($(circ) + (0:\radB)$); 

\coordinate (o) at ($(n)!-1!(g)$); 

\coordinate (s) at (tangent cs:node=circ, point={(o)}, solution=1);  

\draw[name path=line1, line width=0.3mm] (g) --(n) node[midway] (tick1) {} -- (o) node[midway] (tick2) {} -- (s) ;

\node(g.label) at ($(g)+(-90:0.25*\radB)$) {$ G$};

\node(s.label) at ($(s)+(90:0.25*\radB)$) {$ S$};

\node(o.label) at ($(o)+(30:0.25*\radB)$) {$ O$};

\tikzset{onetick/.pic={\draw[line width=0.3mm] ($(0,0)+(0,0.1*\radB)$) -- ($(0,0)-(0,0.1*\radB)$) ; }}

\pic[rotate=50] at (tick1) [pic type = onetick];  

\pic[rotate=50] at (tick2) [pic type = onetick];  

\node(x.label) at ($(tick1)+(90:0.28*\radB)$) {$ x$};

\node(10-label) at ($(s)!0.5!(o) +(90:0.25*\radB)$) {$ 10$}; 

\end{tikzpicture} 



\newpage 
%23
\question Find the value of $x$ in $\odot O$
%\begin{multicols}{4}
\begin{choices}  
\choice 12
\choice 24
\CorrectChoice 48
\choice 60
\end{choices}
%\end{multicols} 

\vspace*{-2.7cm}\hspace*{11cm}\begin{tikzpicture}[dot/.style={circle, fill=black, inner sep=0pt, outer sep=0pt, minimum size=3pt}, 
nodot/.style={circle, fill=black, inner sep=0pt, outer sep=0pt, minimum size=0pt}, 
baseline = (current bounding box.west)
]  

\node(o)[dot] at (0,0) {};

\draw[name path=circ, line width=0.5mm] (o) circle (\radfig3a);

\node(o-label) at ($(o)+(250:0.22*\radfig3a)$) {$\  O$};

\coordinate (y) at ($(o) +(\radfig3a, 0)$);

\node(y.label) at ($(y)+(0:0.22*\radfig3a)$) {$\  Y$};

\coordinate (a) at ($(o) + (0, \radfig3a)$);

\coordinate (int) at ($(o)! 0.6!(a)$);

\path[name path=line.guess, line width=0.3mm] ($(int)!2.3!-90:(o)$) -- ($(int)!2!90:(o)$);

\fill[black, name intersections={of=line.guess and circ, name=point}];

\node[nodot] (l) at (point-1){};

\node(l.label) at ($(l)+(0:0.22*\radfig3a)$) {$\  L$};

\draw[line width=0.3mm] (o) -- (y) node [midway, yshift=-0.19*\radfig3a] (25)
{$\  25$};

\draw[line width=0.3mm] (o) -- (int) node [midway, xshift=-0.2*\radfig3a, yshift=-2pt] (7)
{$\  7$};;

\draw[line width=0.3mm] (point-2) -- (l) node [midway, yshift=0.15*\radfig3a] (x)
{$\  x$};;

\begin{scope} [rotate=-90]
\draw[line width=0.3mm] (int) rectangle ++(0.22*\radfig3a,0.22*\radfig3a) node[transform shape]{};
\end{scope}

\end{tikzpicture}



%\newpage 
%24
\question If an inscribed angle of a circle intercepts a semicircle, then the angle is \blank. 
\begin{multicols}{4}
\begin{choices}  
\choice acute
\CorrectChoice right
\choice obtuse
\choice straight 
\end{choices}
\end{multicols} 

%25
\question Which of the following represents the distance $d$ between the two points $A(x_1, y_1)$ and $B(x_2, y_2)$?   
\begin{multicols}{2}
\begin{choices}  
\choice $AB = \sqrt{(x_2-x_1)^2 - (y_2-y_1)^2}$
\CorrectChoice $AB = \sqrt{(x_2-x_1)^2 + (y_2-y_1)^2}$
\choice $AB = \sqrt{(x_2+x_1)^2 + (y_2+y_1)^2}$
\choice $AB = \sqrt{(x_2+x_1)^2 - (y_2+y_1)^2}$
\end{choices}
\end{multicols} 

%26
\question Point $L$ is the midpoint of $\overline{KM}$. Which of the following is true about the distances among $K$, $L$, and $M$? 
\begin{multicols}{2}
\begin{choices}  
\choice $LM=KM$
\CorrectChoice $KL=LM$
\choice $KL=KM$
\choice $2|KM|=KL+LM$
\end{choices}
\end{multicols} 

%27
\question What is the distance between the points $P(-2, 6)$ and $Q(-7, 7)$? 
\begin{multicols}{4}
\begin{choices}  
\choice $2\sqrt{6}$
\choice $4$
\choice $2\sqrt{26}$
\CorrectChoice $\sqrt{26}$
\end{choices}
\end{multicols} 

%28
\question Which of the following equations describes a circle on the coordinate plane with a radius of 4 units? 
\begin{multicols}{2}
\begin{choices}  
\choice $(x+2)^2 - (y-2)^2 = 4^2$
\CorrectChoice $(x+2)^2 + (y-2)^2 = 4^2$
\choice $(x-4)^2 + (y-4)^2 = 2^2$
\choice $(x-4)^2 + (y-4)^2 = 16^2$
\end{choices}
\end{multicols} 

%29
\question What  is the  center  of  the  circle  $x^2+y^2-4x+10y+13=0$?  
\begin{multicols}{4}
\begin{choices} 
\choice $(–2,  –5)$
\choice $(2,  5)$
\choice $(–2,  5)$
\choice $(2,  –5)$
\end{choices}
\end{multicols} 

%30
\question The coordinates of the vertices of a square are $H(3, 8)$, $I(15, 8)$,           $J(15, -4)$, and $K(3, -4)$. What is the length of a diagonal of the square?  
\begin{multicols}{4}
\begin{choices}  
\choice $4$ 
\choice $8$ 
\choice $12$ 
\choice $12\sqrt{2} $
\end{choices}
\end{multicols} 

\end{questions}