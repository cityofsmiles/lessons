% cd /sdcard/VerbTeX/Local/exam/; pdflatex prefinal-exam-gr10.tex

  
\begin{questions}

%\vspace{\stretch{1}}
%1
\question The  median  score  is also  the  \blank. 
\begin{multicols}{4}
\begin{choices} 
\choice \ordinalnum{75}  percentile
\CorrectChoice \ordinalnum{5}  decile
\choice \ordinalnum{3}  decile
\choice \ordinalnum{1}  quartile
\end{choices}
\end{multicols}


%\vspace{\stretch{1}}
%2
\question When  a  distribution  is  divided  into  hundred  equal  parts,  each  score point  that  describes the  distribution  is called  a    \blank. 
\begin{multicols}{4}
\begin{choices} 
\choice Decile 
\choice Median 
\CorrectChoice Percentile 
\choice Quartile 
\end{choices}
\end{multicols}

%\vspace{\stretch{1}}
%3
\question The  lower quartile  is equal to \blank.  
\begin{multicols}{4}
\begin{choices} 
\choice \ordinalnum{50}  percentile
\CorrectChoice \ordinalnum{25}  percentile
\choice \ordinalnum{2}  decile 
\choice \ordinalnum{3}  quartile
\end{choices}
\end{multicols}

%\vspace{\stretch{1}}
%4
\question In the set of scores 14, 17, 10,  22,  19,  24,  8,  12,  and  19,  the  median score is  \blank. 
\begin{multicols}{4}
\begin{choices} 
\choice 17
\choice 16
\choice 15
\choice 13  
\end{choices}
\end{multicols}

%\vspace{\stretch{1}}
%5
\question In a 70-item test,   Melody  got  a  score  of  50  which  is  the  third  quartile.    This  means  that: 
%\begin{multicols}{2}
\begin{choices} 
\choice she  got  the  highest  score. 
\choice her score is  higher than  25\%  of  her classmates.
\CorrectChoice she  surpassed  75\%  of  her classmates.
\choice seventy-five  percent  of  the  class did  not  pass  the  test.  
\end{choices}
%\end{multicols}

%\vspace{\stretch{1}}
%6
\question The  first  quartile  of  the  ages  of  250  fourth  year  students  is  16  years  old. Which  of  the  following  statements  is true? 
%\begin{multicols}{2}
\begin{choices} 
\choice Most  of  the  students  are  below  16  years old.
\CorrectChoice Seventy-five  percent  of  the  students  are  16  years old  and  above.
\choice Twenty-five  percent  of  the  students  are 16  years old. 
\choice One  hundred  fifty  students  are younger than  16  years. 
\end{choices}
%\end{multicols}

%\vspace{\stretch{1}}
%7
\question In  a  100-item  test,  the  passing  mark  is  the  \ordinalnum{3}  quartile.  What  does  it imply? 
%\begin{multicols}{4}
\begin{choices} 
\choice The  students  should  answer  at  least  75  items  correctly  to  pass  the test. 
\choice The  students  should  answer  at  least  50  items  correctly  to  pass  the test.
\CorrectChoice The  students  should  surpass  75\%  of  their classmates  to  pass  the test.
\choice The  students  should  surpass  50\%  of  their classmates  to  pass  the test. 
\end{choices}
%\end{multicols}

%\vspace{\stretch{1}}

\uplevel{For items 8 to 20, refer to table A below.
%\centering
\begin{center}
%\vspace*{-2ex}
%\scalebox{1}{
\textbf{Table A} \\[1ex]

\noindent\begin{minipage}{3in}
{\begin{tabularx}{\textwidth}{YYY}
\toprule
\textbf{Scores} & \textbf{Frequency} & \textbf{Cumulative Frequency} \\
\midrule
46	--	50	&	4	&	50	\\
\midrule
41	--	45	&	8	&	46	\\
\midrule
36	--	40	&	11	&	38	\\
\midrule
31	--	35	&	9	&	27	\\
\midrule
26	--	30	&	12	&	18	\\
\midrule
21	--	25	&	6	&	6	\\
\bottomrule
\end{tabularx}}
\end{minipage}%}
\end{center} 
}

%8
\question What is the size of the class interval of the frequency distribution table? 
\begin{multicols}{4}
\begin{choices} 
\choice 3
\choice 4
\CorrectChoice 5
\choice 6
\end{choices}
\end{multicols}

%\vspace{\stretch{1}}
%9
\question What is the total frequency? 
\begin{multicols}{4}
\begin{choices} 
\choice 27
\choice 38
\choice 46
\CorrectChoice 50
\end{choices}
\end{multicols}

%\vspace{\stretch{1}}
%10
\question Which class interval is the $Q_1$ class? 
\begin{multicols}{4}
\begin{choices} 
\choice 21--25
\CorrectChoice 26--30
\choice 31-35
\choice 36--40
\end{choices}
\end{multicols}

%\vspace{\stretch{1}}
%\newpage
%11
\question Which class interval is the $Q_3$ class? 
\begin{multicols}{4}
\begin{choices} 
\choice 21--25
\choice 26--30
\choice 31-35
\CorrectChoice 36--40
\end{choices}
\end{multicols}

%\newpage
%12
\question What cumulative frequency should be used in solving for the \ordinalnum{35} percentile? 
\begin{multicols}{4}
\begin{choices} 
\choice 38
\choice 27
\choice 18
\CorrectChoice 6
\end{choices}
\end{multicols}

%13
\question What cumulative frequency should be used in solving for the \ordinalnum{8} decile? 
\begin{multicols}{4}
\begin{choices} 
\choice 6
\choice 18
\choice 27
\CorrectChoice 38
\end{choices}
\end{multicols}

%14
\question In solving for the \ordinalnum{60} percentile, the lower boundary is \blank.
\begin{multicols}{4}
\begin{choices} 
\choice 30.5
\CorrectChoice 35.5
\choice 40.5
\choice 45.5
\end{choices}
\end{multicols}

%15
\question In solving for the \ordinalnum{9} decile, the lower boundary is \blank.
\begin{multicols}{4}
\begin{choices} 
\choice 25.5
\choice 30.5
\choice 35.5
\CorrectChoice 40.5 
\end{choices}
\end{multicols}

%16
\question The \ordinalnum{45} percentile is \blank.  
\begin{multicols}{4}
\begin{choices} 
\choice 49
\CorrectChoice 360
\choice 720 
\choice 5040 
\end{choices}
\end{multicols}


%\vspace{\stretch{1}}
%17
\question What is the frequency of the $P_{35}$ class? 
\begin{multicols}{4}
\begin{choices} 
\choice 8
\choice 9
\choice 11 
\CorrectChoice 12 
\end{choices}
\end{multicols}

%\vspace{\stretch{1}}
%18
\question What is the frequency of the $D_{6}$ class? 
\begin{multicols}{4}
\begin{choices} 
\choice 8
\choice 9
\CorrectChoice 11 
\choice 12 
\end{choices}
\end{multicols}

%\vspace{\stretch{1}}
%\newpage
%19
\question The \ordinalnum{1} quartile is \blank.  
\begin{multicols}{4}
\begin{choices} 
\choice 26.2
\CorrectChoice 28.2
\choice 30.2 
\choice 32.2 
\end{choices}
\end{multicols}


%\newpage
%20
\question The \ordinalnum{7} decile is \blank.  
\begin{multicols}{4}
\begin{choices} 
\choice 36.1
\choice 37.1
\choice 38.1 
\CorrectChoice 39.1 
\end{choices}
\end{multicols}

%13









\end{questions}

