% cd /storage/emulated/0/Documents/documents/latex/1920/Grade-10/tests/math-month/ && pdflatex logic-mechanics.tex && termux-open logic-mechanics.pdf

\documentclass[11pt]{article}
\usepackage[letterpaper, portrait, margin=0.5in]{geometry}

\usepackage{xcolor}
\usepackage{anyfontsize}
\usepackage{enumitem}
\usepackage{multicol}
\usepackage{amsmath, makecell}
\usepackage{tabularx} 
\usepackage{gensymb}
\usepackage{wasysym} %for checked symbol 
\usepackage{multirow}
\usepackage{graphicx, tipa}
\usepackage{tikz}
\usetikzlibrary{angles,quotes}
\usepackage{pgfplots} 
\usetikzlibrary{calc}
\pgfplotsset{compat=newest}
\usetikzlibrary{arrows.meta}
\usetikzlibrary{intersections}
\usetikzlibrary{decorations.pathreplacing}
\usepackage{flafter}
%\usepackage{fourier} 
\usepackage{amsmath,amssymb,cancel,units}
\usepackage{microtype} % nicer output 
\usepackage{hfoldsty} % nicer output 
\usepackage{fixltx2e} 
\usepackage{mathptmx}
\usepackage{numprint}
\usepackage[T1]{fontenc}
\usepackage[utf8]{inputenc} 
\usepackage{stackengine} %to define \pesos 
\usepackage{lmodern} %scalable font
\usepackage{booktabs}
\usepackage{array}


\pagenumbering{gobble}
%\linespread{0.9}
\newcommand{\vspce}{\vspace{0.75ex}}

\newcommand{\hspce}{\hspace{0.5em}}

\newcommand{\blank}{\underline{\hspace{2em}}}%{\rule{1em}{0.15ex}}

\newcommand{\arc}[1]{{% 
\setbox9=\hbox{#1}% 
\ooalign{\resizebox{\wd9}{\height}{\texttoptiebar{\phantom{A}}}\cr#1}}}


\newcommand\pesos{\stackengine{-1.4ex}{P}{\stackengine{-1.25ex}{$-$}{$-$}{O}{c}{F}{F}{S}}{O}{c}{F}{T}{S}} 


\renewcommand\theadalign{bc} 

\renewcommand\theadfont{\bfseries} 

\renewcommand\theadgape{\Gape[4pt]} 

\renewcommand\cellgape{\Gape[4pt]} 

\pagenumbering{gobble}

\newcolumntype{Y}{>{\centering\arraybackslash}X} %for tabularx

\newcolumntype{R}{>{\raggedleft\arraybackslash}X} %for tabularx

\newcolumntype{Z}{>{\raggedleft\arraybackslash}X} %for tabularx

\newcolumntype{L}{>{\raggedright\arraybackslash}X} %for tabularx

\newcolumntype{A}[1]{>{\raggedright\arraybackslash}p{#1}} %for longtable LEFT

\newcolumntype{C}[1]{>{\centering\arraybackslash}p{#1}} %for longtable CENTER

\newcolumntype{B}[1]{>{\raggedleft\arraybackslash}p{#1}} %for longtable RIGHT 
 
\renewcommand{\tabularxcolumn}[1]{>{\small}m{#1}}

\newcolumntype{N}[1]{>{\raggedleft}p{#1}} %for tabular left 

\newcolumntype{M}[1]{>{\raggedright\arraybackslash}p{#1}} %for tabular right 

\newcommand{\myaxis}{xticklabels={}, 
yticklabels={}, 
ymin=-10, ymax=10,
xmin=-10, xmax=10,
axis lines = center, 
inner axis line style={Latex-Latex,very thick}, 
grid=both,
minor tick num=4, 
tick align=inside} % grid without labels, origin at the center, 10 units from origin

\newcommand{\axisfive}{xticklabels={}, 
yticklabels={}, 
ymin=-5, ymax=5,
xmin=-5, xmax=5,
axis lines = center, 
inner axis line style={Latex-Latex,very thick}, 
grid=both,
minor tick num=1, 
tick align=inside} % grid with labels, origin at the center, 5 units from origin 

\newcommand \redcheck {{\color{red}\checkmark}}

 

\pagenumbering{gobble}

\begin{document} 

\begin{center}
\textbf{Search for the Logic Grand Masters}
\end{center} 

\textbf{General Game Mechanics} 
%\begin{center}
%\vspace*{2ex}
%\scalebox{1}{
%\noindent\begin{minipage}{\textwidth}
{\begin{enumerate}[label = \arabic*. ]
%\begin{multicols}{2}
%1
\item Each contestant must be a bona fide student of Sauyo High School. 
%2
\item There must only be 10 contestants per year level. 
%3
\item There are three categories with different pointing systems. The Easy Round will be the elimination per grade level after which only 5 contestants from each grade level will remain to compete in the Average Round. The Average Round will be a group elimination after which only three groups will compete in the Difficult Round. The scores accumulated in the Average Round is reset before the Difficult Round begins. After the Difficult Round, the Quiz Master will determine the Logic Elites (Third Place), Logic Masters (Second Place), and the Logic Grandmasters (First Place). 

%\end{multicols} 
\end{enumerate}}
%\end{minipage}}
%\end{center} 

\textbf{Easy Round Game Mechanics} 
\begin{enumerate}[label = \arabic*. ]
%\begin{multicols}{2}
%1
\item The Easy Round consists of 10 Riddles and Verbal Puzzles to be answered by the contestants from each grade level. 
%2
\item This round will be an individual elimination wherein the 5 contestants to answer the fastest will advance to the Average Round. 
%3
\item The Game Master will attempt to read the question thrice. At the first attempt, the question will be read without choices. If a contestant answers the question incorrectly, he loses a turn and the Game Master will read the question for the second time still without choices. If another contestant answers the question incorrectly, then the Game Master will reread the question along with the choices. For this time, the contestants will have equal chances to answer. 
%4 
\item The contestants are allowed to answer the question anytime even before the question is fully read by pressing the bell situated in the middle of the contestants. 
%\end{multicols} 
\end{enumerate} 

\textbf{Average Round Game Mechanics} 
\begin{enumerate}[label = \arabic*. ]
%\begin{multicols}{2}
%1
\item The Average Round consists of 10 Mathematical and Visual Puzzles to be answered by the 5 groups determined from the Easy Round. 
%2
\item This round will be a group elimination wherein the 3 groups with the highest points will advance to the Difficult Round. 
%3
\item The grouping system in this round is determined by the sequence by which the contestants answered in the previous round. The first to answer from each grade level will belong in group 1, and so on. This process will continue until all the 5 groups are determined. 
%4
\item Each question in the Average Round is worth 2 points except for the Rebus Puzzles. The Rebus Puzzles are counted as one question only. The group with the most answers in the Rebus Puzzles will receive 4 points, the second highest gains 2 points, and the rest will get no points. 
%5 
%\item 
%\end{multicols} 
\end{enumerate} 

\textbf{Difficult Round Game Mechanics} 
\begin{enumerate}[label = \arabic*. ]
%\begin{multicols}{2}
%1
\item The Difficult Round consists of 5 Logic Puzzles with Manipulatives to be answered by the 3 groups who scored the highest in the Average Round. 
%2
\item In this round, puzzles are categorized as Hard or Difficult, depending on the time needed to solve them. The Hard Puzzles are allotted a maximum of 3 minutes to solve, while the Difficult Puzzles are given 3 to 10 minutes to finish. 
%3
\item The Hard Puzzles are worth 5 points and the Difficult ones are worth 10 points. 
%4
\item The maximum possible score in this round is 40 points. 
%5 
%\item 
%\end{multicols} 
\end{enumerate} 

\end{document} 