\begin{questions}

%1
\question What is the degree of the polynomial function $P(x) = 3x - 9x^3 + 5x^4 -5$?  

%2
\question Given the polynomial function $P(x) = 121x^2 - 5x^{11} + x^8 + 2x^5 - 50$, find its  leading term.

%3
\question If three geometric means are inserted between 1 and 256, find the third geometric mean.  


%4
\question What is the next term in the harmonic sequence $\dfrac{1}{11}, \dfrac{1}{15}, \dfrac{1}{19}, \dfrac{1}{23}, \ldots$? 

%5
\question The polynomial  function $P(x) = 4x^4  - 17x^2 + 4$ has how many  possible rational zeros?


%6
\question What is the next term in the geometric sequence  4, 12, 36? 


%7
\question Find the common difference in the arithmetic sequence $3, \dfrac{13}{4}, \dfrac{7}{2}, \dfrac{15}{4}$. 


%8
\question If $(x-1)$ is a factor of the polynomial $x^2 - 2x + 1$, what is the other factor?



%9
\question Find the equation of a quadratic function whose zeros are $5$ and $-3$. 


%10
\question Find the remainder of  $P(x) = 3x^{100} -4x^{50} + 8$  divided by $(x + 1)$.


%11
\question What do we call an angle formed by two rays whose vertex is the center of a circle? 


%12
\question What do we call the points where the graph of a function intersects the x-axis?  



%13
\question What are the end behaviors of the graph of the polynomial function $y = x^3 + 3x^4 - x^5 - 7x^2 + 4$? 


%14
\question Which term determines how many times a particular number is a zero or root for a given polynomial? 



%15
\question What should $n$ be if $f(x) = x^n$ defines a polynomial function? 


%16
\question What is an angle whose vertex is on a circle and whose sides contain chords of the circle?  


%17
\question In a circle, if a central angle measures $60\degree$, what is the measure of its intercepted arc? 




%18
\question A dart board has a diameter of 40 cm and is divided into 20 congruent       sectors. What is the area of one of the sectors?    


%19
\question What is the y-intercept of the graph of the polynomial function $f(x) = -2x + x^3 + 3x^5 -4$?




%20
\question How many turning points does the polynomial function $f(x) = -2x + x^3 + 3x^5 - 4$ have?




%21
\question Choosing  a  subset  of  a  set  is an  example  of \blank.



%22
\question What are the coordinates of the center of the circle defined by the equation $x^2 + (y-5)^2 = 8$?



%23
\question What do we call the  product  of  a  positive  integer  $n$  and  all  the  positive  integers  less than  $n$?   


%24
\question A radio signal can transmit messages up to a distance of 3 km. If the radio
signal’s origin is located at a point whose coordinates are $(4, 9)$, what is
the equation of the circle that defines the boundary up to which the messages can be transmitted?




%25
\question How many  different  4-digit  even  numbers can  be  formed  from  the  digits 1,  3,  5,  6,  8,  and  9  if  no  repetition  of  digits  is allowed? 



%26
\question What is the center of the circle $x^2 + y^2 - 4x + 10y + 13 = 0$?




%27
\question In how many ways can 8  people  be  seated  around  a  circular  table  if two  of  them  insist  on  sitting  beside  each  other? 


%28
\question On a grid map of a province, the coordinates that correspond to the
location of a cellular phone tower is $(-2, 8)$ and it can transmit signals
up to a 12 km radius. What is the equation that represents the transmission boundaries of the tower?



%29
\question In a town fiesta singing competition with 12 contestants, in how many ways can the organizer arrange the first three singers? 


%30
\question If a combination lock must contain 5 different digits, in how many ways can a code be formed from the digits 0 to 9? 


%31
%32
%33
%34
%35
%36
%37
%38
%39
%40


\end{questions}