\begin{center}
\textbf{Mode for Grouped Data}
\end{center}

\vspace*{1ex}

\begin{center}
%\vspace*{-2ex}
\scalebox{0.9}{
\noindent\begin{minipage}{\textwidth}
{
\textbf{Mode} 
\begin{itemize} 
\item[-] the measure or value which occurs most frequently in a set of data
\item[-] the value with the greatest frequency
\end{itemize}  

To find the mode of grouped data, use:  
\[
\text{Mode } (\hat{x})  =lb_{mo} + \left[\displaystyle \frac{d_1}{d_1+d_2} \right]i
\] 

\begin{center}
\begin{tabular}{lcll}
where: & $lb_{mo}$ & = & lower boundary of the modal class \\
& $d_1$ & = & \makecell[tl]{difference between the frequencies of the\\ modal class and the class preceding the modal class} \\ 
& $d_2$ & = & \makecell[tl]{difference between the frequencies of the modal\\ class and the class succeeding the modal class} \\ 
& $i$ & = & class interval\\ 
\end{tabular} 
\end{center}
}
\end{minipage}}
\end{center} 